\renewcommand{\contentsname}{Inhalt}

\newcommand{\hnone}[1]{\section*{#1}\addcontentsline{toc}{section}{#1}\markboth{#1}{}}
\newcommand{\hone}[1]{\section{#1}}
\newcommand{\htwo}[1]{\subsection{#1}\markboth{#1}{}}
\newcommand{\hthree}[1]{\subsubsection{#1}}
\newcommand{\hfour}[1]{\begin{flushleft}
    {\large \textbf{#1}} 
\end{flushleft}}

\renewcommand{\lstlistingname}{Code}
\renewcommand{\lstlistlistingname}{Codeabbildungsverzeichnis}

\newcommand{\typescript}[2]{\begin{singlespacing}\lstinputlisting[language=TypeScript,caption={#2},captionpos=b]{#1}\end{singlespacing}}
\newcommand{\html}[2]{\begin{singlespacing}\lstinputlisting[language=HTML,caption={#2},captionpos=b]{#1}\end{singlespacing}}
\newcommand{\css}[2]{\begin{singlespacing}\lstinputlisting[language=CSS,caption={#2},captionpos=b]{#1}\end{singlespacing}}
\newcommand{\sql}[2]{\begin{singlespacing}\lstinputlisting[language=SQL,caption={#2},captionpos=b]{#1}\end{singlespacing}}

\newcommand{\image}[2]{
    \begin{figure}[h]
        \centering
        \includegraphics{#1}
        \caption{#2}
    \end{figure}
}

\setlength{\parindent}{0px}
\setlength{\parskip}{0.75em}
