\htwo{Express.Js}
\sectionauthor{Mersed Kečo}

Express.Js ist ein server-seitiges Web-Framework für Node.Js. Es ergänzt Node.Js mit weiteren Funktionalitäten, womit die Entwicklung von modernen Webanwendungen durchaus erleichtert wird. Der größte Vorteil ist die Verwendung von JavaScript als Programmiersprache für die Backendprogrammierung. Daher bietet sich "Full Stack Web" Entwicklern die Möglichkeit, Front– und Backend zu programmieren, ohne die Programmiersprache ändern zu müssen. 

Die Express Philosophie ist kleine, robuste Werkezuge für HTTP Server bereitzustellen, wodurch sie eine sehr gute Option für SPAs, Webseiten und öffentliche HTTP APIs ist. \cite{Express}

\hthree{Hello World in Express}

In diesem kleinen Code-Beispiel ist wird ein "Hello World" – Programm, welches in Express.js geschrieben ist, vorgestellt:

\typescript{code/ExpressJs/HelloWorld.ts}{HelloWorld in Express.Js}\cite{Express}

