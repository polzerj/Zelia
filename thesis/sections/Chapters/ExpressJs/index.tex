\htwo{Express.js}
\sectionauthor{Mersed Kečo}

Express.js ist ein server-seitiges Web-Framework für Node.js. Es ergänzt Node.js mit weiteren Funktionalitäten, womit die Entwicklung von modernen Webanwendungen durchaus erleichtert wird. Der größte Vorteil ist die Verwendung von JavaScript als Programmiersprache für die Backendprogrammierung. Daher bietet sich "Full Stack Web" Entwicklern die Möglichkeit, Front– und Backend zu programmieren, ohne die Programmiersprache ändern zu müssen. Die "Express" Philosophie ist, kleine, robuste Werkezuge für HTTP Server bereitzustellen, wodurch sie eine sehr gute Option für SPAs, Webseiten und öffentliche HTTP APIs ist. \cite{Express}

\hthree{Full Stack Web}
Der Begriff "Full Stack Developer" stammt aus dem Englischen und ist die Berufsbezeichnung von speziellen Programmierern, welche in der Frontend- sowie Backendentwicklung tätig sind. Durch die Vielfalt an Programmiersprachen, die ein "Full Stack Developer" beherrscht sind diese sehr oft mit dem Design von Benutzer-/Webanwedungen beschäftigt. Der einzige "Nachteil", den die Entwickler meistens erleben ist, das durch das breite Fachwissen meistens ein weniger spezifisches Fachwissen vorherrscht. 

Zu den Aufgaben eines "Full Stack Developers" gehört auch das unteranderem das Testen, Überwachen und Protokollieren von entwickelter Software. Einer der wichtigen Punkte, die einen solchen Entwickler ausmachen ist aber auch, falls Fehler auftreten, die Fehlersuche in ihren Aufgabenbereich fallen. \cite{FullStack}
%TODO: Full Stack Web erklären

\pagebreak
\hthree{Hello World in Express}

In diesem kleinen Code-Beispiel ist wird ein "Hello World" – Programm, welches in Express.js geschrieben ist, vorgestellt:

%TODO: Warum Express.js verwendet wird im Vergleich zu Node.js, Express Features erwähnen, Zelia Bezug schaffen 

\typescript{code/ExpressJs/HelloWorld.ts}{HelloWorld in Express.js}\cite{Express}

