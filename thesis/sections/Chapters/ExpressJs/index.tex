\htwo{Express.js}
\sectionauthor{Mersed Kečo}

Express.js ist ein server-seitiges Web-Framework für Node.js. Es ergänzt Node.js mit weiteren Funktionalitäten, womit die Entwicklung von modernen Webanwendungen durchaus erleichtert wird. Der größte Vorteil ist die Verwendung von Javascript als Programmiersprache für die Backendprogrammierung. Daher bietet sich "Full Stack Web" Entwicklern die Möglichkeit, Front– und Backend zu programmieren, ohne die Programmiersprache ändern zu müssen. Die "Express" Philosophie ist, kleine, robuste Werkezuge für HTTP Server bereitzustellen, wodurch sie eine sehr gute Option für SPAs, Webseiten und öffentliche HTTP APIs ist. \cite{Express}

\hthree{Full Stack Web}
Der Begriff "Full Stack Developer" stammt aus dem Englischen und ist die Berufsbezeichnung von speziellen Programmierern, welche in der Frontend- sowie Backendentwicklung tätig sind. Durch die Vielfalt an Programmiersprachen, die ein "Full Stack Developer" beherrscht sind diese sehr oft mit dem Design von Benutzer-/Webanwedungen beschäftigt. Der einzige "Nachteil", den die Entwickler meistens erleben ist, das durch das breite Fachwissen meistens ein weniger spezifisches Fachwissen vorherrscht. 

Zu den Aufgaben eines "Full Stack Developers" gehört auch das unteranderem das Testen, Überwachen und Protokollieren von entwickelter Software. Einer der wichtigen Punkte, die einen solchen Entwickler ausmachen ist aber auch, falls Fehler auftreten, die Fehlersuche in ihren Aufgabenbereich fallen. \cite{FullStack}

\pagebreak
\hthree{Hello World in Express}

In diesem kleinen Code-Beispiel ist wird ein "Hello World" – Programm, welches in Express.js geschrieben ist, vorgestellt:


\typescript{code/ExpressJs/HelloWorld.ts}{HelloWorld in Express.js}\cite{Express}


\hthree{Vorteile}
Wie auch schon erwähnt erweitert Express Node.js, leider werden aber die Unterschiede und Vorteile von Express.js nicht allzu klar offengelegt. Mithilfe der folgenden Punkte wird versucht, eine kleine Übersicht zu liefern, warum Express in der Anwendung von \ZELIA\ die erste Wahl war.

\begin{enumerate}
    \item Besonderheiten
    \newline
    Während Node.js eher einer einfachen Handhabung entspricht und nicht allzuviele Besonderheiten aufweißt, implementiert Express.js eine Vielfalt an Besonderheiten und Features, welches das eigentliche Ziel von Express erzielt.\cite{NodeExp}
    \item Verwendung
    \newline
    Die Verwendung von Node.js findet meistens bei der Programmierung von serverseitigen,ereignisbasierenden Anwendungen statt. Im Vergleich dazu verwendet Express Ansätze von Node für die Entwicklung von Webanwendungen und Programmierschnittstellen (APIs).\cite{NodeExp}
    \item Kategorisierung
    \newline
    Bei dem Wort "Kategorisierung" handelt es sich um die Kategorisierung der jeweiligen Frameworks in entweder Front-,Backend und Full Stack Programmierung. In diesem Fall würde Node.js unter die Kategorie "Full Stack" und Express unter "Backend".\cite{NodeExp}
    \item Zeitaufwand
    \newline
    Aufgrund der verschiedenen Anwendungsbereiche und aber auch die Erforderung von mehr Codezeilen bei der Programmierung mithile von Node ist mehr Zeit zur Compilierung erforderlich. Im Gegenteil dazu ist Express um einiges schneller und kann mit weniger Zeilen Code dasselbe Ergebnis erreichen wie Node.\cite{NodeExp}
\end{enumerate} 

Pauschal ist nicht zu sagen, ob entweder Node.js oder Express.js besser ist, man kann nur sagen das durch die ständigen Erweiterungen und die fortschreitende Entwicklung Express auf jeden Fall langlebiger ist und in der Zukunft auch verstärkter verwendet wird. \cite{NodeExp}

\htwo{Verwendung}
Express.js wird in \ZELIA\ primär für die Implementierung von Controllern, das Routing und die Verwendung von Middleware verwendet, da dies nicht von Node.js realisiert werden kann. Die Themen und Erklärungen sind in den Kapiteln \ref{sec:csrouter} (siehe Seite \pageref{sec:csrouter}), \ref{sec:controller} (siehe Seite \pageref{sec:controller}) und \ref{sec:middleware} (siehe Seite \pageref{sec:middleware}) zu finden. \cite{NodeExp}
