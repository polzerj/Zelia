\htwo{Client Side Router}
\label{sec:csrouter}
\sectionauthor{Julian Kusternigg}
\hthree{Allgemeines}

\hfour{Einleitung}

In der Netzwerktechnik ist "Routing" der Vorgang, um einen Weg für Daten zu finden. In der Webentwicklung ist das gar nicht so anders. Jede Seite hat einen bestimmten Pfad, eine Route, z.B.:

\emph{https://zelia/home} oder \emph{.../room/S1308}

In diesen Fällen wären "/home" und "/room/S1308" 2 verschiedene Routen. Früher hat man am Webserver, also der Server, der die Daten zu Verfügung stellt, ein Verzeichnis gehabt, dass alle möglichen Pfade abbildet.

\begin{figure}[H]
    \begin{lstlisting}
www/
> home/
  > room/
    > S1308/
    > ...
    \end{lstlisting}
    \caption{Ordnerstruktur eines Webservers}
\end{figure}

Diese Art von Webserver stellt statische Websites zu Verfügung. Statisch, weil die Daten, z.B.: das "index.html", nicht geändert werden, sondern so wie sie am Server liegen dem Client geschickt werden. Dies hat den Vorteil das der Server kaum Ressourcen verbraucht. In jedem dieser Verzeichnisse kann man ein "index.html" finden, welches der Webserver schickt, wenn einer dieser Pfade vom Browser angefragt wird. Da gibt es schon das erste Problem: Es muss für jeden Raum ein fertiges "index.html" im richtigen Pfad geben, um darauf zugreifen zu können.

Als Lösung dafür gibt sogenannte Router, die wissen was zu tun ist, wenn eine Route angefragt wird. Somit kann man diese statische Hierarchie auflösen und auf jede Anfrage individuell reagieren. In der Webentwicklung werden diese Steuerungsprogramme, die auf Anfragen reagieren, meist Controller genannt. Ein Controller bekommt somit eine Anfrage und liefert eine Antwort, welche im Zusammenhang von vorher das "index.html" ist. Dabei ist aber wichtig zu erwähnen, dass der Inhalt des "index.html", also der Antworttext, modifiziert werden kann. So kann ein Router beispielsweise alle Routen die aus "/room/..."bestehen an einen Controller weitergeben, der dem Client jeweils andere Informationen zurückgibt.

Nochmal kurz zusammengefasst: Router sind in der Webentwicklung dazu da, um Routen (Pfade) aufzulösen und an den richtigen Controller weiterzureichen. Controller liefern dann individuelle Ergebnisse auf jeweilige Anfragen.

\hfour{Backend Routing}

Backend Routing wird der Vorgang genannt bei dem das Routing im Backend der Applikation, also Serverseitig, passiert. Beispiele dafür wären wie oben genannt: statische Webserver oder dynamische Webserver, welche mit Controllern arbeiten.

\hfour{Frontend Routing}

Frontend Routing, auch "Client-Side-Routing" genannt, ist, wenn das Routing zum Client verschoben wird, also im Browser abgefertigt wird. So muss ein Webserver nur eine einzige Seite zu Verfügung stellen und der Client lädt diese und zusätzlich holt er sich die notwendigen Informationen, die er braucht, um eine Seite darzustellen.

Im Beispiel von vorhin, müsste der Webserver nur ein "index.html" hosten welches der Client, egal welche Route er besucht herunterlädt. Das ist meist eine Minimale HTML Seite, die eine JavaScript Datei einbindet. Dieses Skript schaut dann auf welchem Pfad es sich befindet und holt Informationen die benötigt werden von einer API um eine Seite aufzubauen. Das "index.html" wird nur einmal zu Beginn, wenn die Website besucht wird, also wenn man die URL manuell im Browser eingibt, heruntergeladen. Bei jedem Link auf der Website werden nur die notwendigen Infos nachgeladen und angezeigt. Zusätzlich wird mit Hilfe der "History-API" ein Eintrag in den Verlauf gemacht so, dass man mit der "Zurücktaste", im Browser, zu zuvor besuchten Pfaden zurückspringen kann.

Obwohl die Seite eigentlich nie verlassen wird, ändert der Browser den Verlauf und den Pfad, der in der URL-Eingabe Box steht. Somit ist es möglich das Routing komplett in den Browser zu verschieben.

\hfour{Vorteile des Client-Side-Routers}

Einer der größten Vorteile ist, dass die eigentliche Webpage, also das "index.html", praktisch überall bereitgestellt werden kann, da es wie schon gesagt kaum Ressourcen benötigt so einfache statische Seiten zu Verfügung zu stellen. Die Browser auf den Geräten, mit denen die Seite aufgerufen wird, haben eine komplexere Aufgabe, aber mit heutigen Computern ist das auch kein Problem mehr.

Außerdem ist es oft schneller, nur notwendige Infos nachzuladen, vor allem wenn man im Browser heruntergeladene Daten in Caches zwischenspeichert. Somit müssen am Anfang meist mehr Daten geladen werden, dafür hat man dann beim Wechseln der Seite kaum Ladezeiten. Zusätzlich kann man Musterwerte verwenden, oder einen Ladebildschirm anzeigen, wenn die individuellen Werte noch nachgeladen werden. So kann der Browser schon mal Inhalte darstellen und es muss nicht gewarten werden, bis eine gesamte Seite am Server zusammengestellt und geschickt wird.

Die Daten, die angezeigt werden sollen, werden von "API-Servern" zu Verfügung gestellt. Dort kann man zentral die Schnittstellen definieren und überwachen welche Daten zugänglich sind. Somit ist Server und Client besser voneinander getrennt, anders als wenn am Server die HTML Seiten generiert werden, was bei komplexen Anwendungen schnell unübersichtlich werden kann.

\hthree{Implementierung}

Für \ZELIA\ ist "Sequelize" verwendet, da die Software in ein MariaDB Datenbanksystem schreiben, Daten ändern, oder Datensätze auslesen muss.

Durch das Einbinden der Bibliothek kann einerseits die Verbindung zum Datenbankserver hergestellt werden und andererseits auch verschiedene Methoden implementieren werden, welche auf die Daten aus der Datenbank zugreifen können. "Sequelize" stellt für zweiteres auch Methoden bereit, welche zum Beispiel bei der Suche auch durch verschiedene Eigenschaften eingeschränkt werden können (WHERE Parameter).

\hfour{Verbindung zur Datenbank aufbauen}

Um die Verbindung mit dem Datenbankserver herstzustellen ist folgendes notwendig:

Zu Beginn muss das Sequelize Framework eingebunden werden.

\typescriptsub{code/Sequelize/datenbankverb.ts}{Importieren von Sequelize}{1}{1}

Um eine Verbindung aufzubauen werden Parameter benötigt, welche bei \ZELIA\ in einer eigenen ".env"-Datei ausgelagert ist. Folgende Parameter sind für den Aufbau erforderlich:

\begin{itemize}
    \item "DB\_SERVER" -- Gibt an, über welcher Adresse der Datenbank Server erreichbar ist.
    \item "DB\_USER" -- Gibt an, über welchen User, welcher schon im Datenbanksystem angelegt sein muss, man sich zum Server verbindet.
    \item "DB\_PASSWORD" -- Gibt das passende Passwort zum angegebenen User mit.
    \item "DB\_DATABASE" -- Gibt die Datenbank an, auf welche zugegriffen werden soll. Diese muss schon im Datenbanksystem angelegt worden sein.
\end{itemize}

Nun werden die jeweiligen Variablen in eine lokal angelegte Konstante gespeichert. Zugegriffen wird auf die ".env"-Datei mit "{\ttfamily process.env}".

\typescriptsub{code/Sequelize/datenbankverb.ts}{Erstellen einer Konstante für die Umgebungsvariablen}{3}{3}

Weiters wird eine lokale Variable namens "sequelize" von der Klasse "Sequelize" angelegt. Diese Klasse wird durch das Importieren von "Sequelize" in der verwendeten Datei verfügbar.

\typescriptsub{code/Sequelize/datenbankverb.ts}{Lokale Variable von der Klasse Sequelize}{4}{4}

Diese lokale Variable wird nun mit "{\ttfamily = new Sequelize}" zu einem neuen Sequelize Objekt. Der Konstruktor von Sequelize, welcher in \ZELIA\ verwendet wird benötigt drei Parameter, nämlich einen Datenbanknamen, den Datenbankuser und das Passwort für den vorher angegebenen User. 

Nachfolgend können außerdem diverseste Optionen angegeben werden. Im Projekt \ZELIA\ werden zwei Optionen verwendet, nämlich der "Host", bei welchem die Adresse des Datenbankserver mitgegeben wird, und den "dialect". Dieser gibt an auf welches Datenbanksystem zugegriffen werden soll.

Folgender Code ist für das oben beschriebene zuständig:

\typescriptsub{code/Sequelize/datenbankverb.ts}{Erzeugen eines neuen Sequelizeobjekts}{6}{9}

Außerdem wird das Schlüsselwort "export" verwendet, da so das "Sequelize-Objekt" nun auch in anderen Dateien aufgerufen werden kann. Dies ist für den weiteren Datenbankzugriff und in weiterer folge die Verbindungsklassen wichtig.

\typescriptsub{code/Sequelize/datenbankverb.ts}{Exportieren des Sequelizeobjekts}{11}{11}

\hfive{Gesamter Code}

\typescript{code/Sequelize/datenbankverb.ts}{Verbindung zur Datenbank herstellen}


\hfour{Typen für Rauminfos in ZELIA}\label{sec:roomtype}

Damit Informationen über einen Raum korrekt angezeigt werden können, ist es wichtig, dass gewisse Eigenschaften nur gewisse Werte annehmen können. Dies gibt es in \ZELIA\ etwa für den Raumtyp, den Tafeltyp, die Art des Projektors und die Verbindungsmöglichkeiten für den Projektor.

Im Code wird das folgendermaßen realisiert:

Das Schlüsselwort "export" wird angegeben, um den Typ in allen Dateien zugreifbar zu machen. Weiters muss dieser Typ einen Namen bekommen und die einzelnen Werte des Typen eingetragen werden. 

Das nachfolgende Beispiel zeigt, dass es ein Raum nur einen der eingetragenen Raumtypen annehmen darf.

\typescript{code/Sequelize/einfach.ts}{Einfachauswahl eines Raumtypen}

Bei einem Projektor gibt es in der Regel mehrere Möglichkeiten, eine Verbindung aufzubauen. Deshalb gibt es hier eine Mehrfachauswahl.

Die Auswahl mehrerer Werte wird durch ein Array zum Schluss ermöglicht.

\typescript{code/Sequelize/mehrfach.ts}{Mehrfachauswahl einer Verbindungsmöglichkeit für den Projektor}

\hfour{Entities}

In \ZELIA\ gibt es im Ordner "data" einen Unterordner namens "entities". In diesem sind für jede Tabelle in der Datenbank eine "Entity" mit all ihren Eigenschaften enthalten. Abgewickelt wird es über ein "Interface", welches die jeweiligen Eigenschaften enthält.

Außerdem kommen hierbei die Raumtypen (siehe Kapitel "Raumtypen" \ref{sec:roomtype}, Seite \pageref{sec:roomtype}) zum Einsatz, indem sie für einzelne Eigenschaften im Interface verwendet werden.

Beispiel für eine Entity:

\typescript{code/Sequelize/entity.ts}{Entity Raum}
\hfour{Verbindungsklassen für die Datenbank}

Die Verbindungsklassen sind für die Kommunikation mit der Datenbank und deren unterschiedlichen Tabellen zuständig. Nach einem gewissen Initialisierungsprozess, welcher später beschrieben wird, folgen am Ende der Klasse Methoden. Diese Methoden sind für den Zugriff nach verschiedenen Kriterien zuständig (Daten nach einem gewissen Parameter auslesen, oder alle Daten auslesen). Im Projekt ZELIA gibt es Vier unterschiedliche Arten von Zugriffsmethoden auf die Datenbank.

\begin{itemize}
    \item Das Auslesen aller Daten
    \item Das Auslesen von Daten nach gewissen Kriterien (zum Beispiel nach mitgegebener Raumnummer)
    \item Das Verändern von vorhandenen Datensätzen in der Datenbank
    \item Das hinzufügen von Datensätzen in die Datenbank
\end{itemize}

Um die Struktur einer Verbindungsklasse besser zu verstehen, kann man sich das Klasse "RoomReportConnection" ansehen.

Zu Beginn muss man aus der Sequelize-Bibliothek "Models" und "DataTypes" importieren. Folgender Code ist dafür zuständig:

\typescript{code/Sequelize/importVerbindung.ts}{Importieren von "Models" und "DataTypes"}

Im zweiten Schritt werden diverseste Datentypen aus ZELIA in die Klasse importiert, welche in weiteren Schritten benötigt werden.

In weiterer Folge müssen nun die Werte, welche in der Tabelle auf die zugegriffen werden soll definiert werden. Hierzu erstellt man eine neue Klasse, welche das "RoomReportEntity" Interface beinhaltet und implementiert. In dieser Klasse werden nun die Spalten Eigenschaften mit Zugriffsoperatoren und Datentypen versehen. Außerdem werden an alle Eigenschaften außer "Id" ein "!" angehängt. Dieses gibt da, dass es sich um ein Pflichtfeld handelt. Bei der "Id" steht ein "?". Dieses steht für eine nicht zwangläufig auszufüllende Eigenschaft.

\typescript{code/Sequelize/class.ts}{Klasse "RoomReport"}

Diese Klasse, welche das "Entityinterface" nun implementiert muss nun nach den Sequelizestandard initialisiert werden. Hierbei wird auf die gewünschte Klasse die von Sequelize zur Verfügung gestellte Methode "init" angewandt. Nachfolgend wird nun jeder Eigenschaft aus der Klasse (in diesem Fall RoomReport) mehrere Attribute mitgegeben (zum Beispiel ein "type" oder "allow null").

Außerdem werden Eigenschaften, welche in der Datenbank über einen Fremdschlüssel referenziert werden mit dem Schlüsselwort "references" auf eine Spalte einer anderen Tabelle referenziert. Hierbei muss nachher noch das "Model" angegeben werden auf welches Bezug genommen wird und der Schlüssel, welcher referenziert werden soll.

Zum Schluss wird noch der Name der Tabelle angegeben, auf welche Bezug genommen wird und das Sequelizeobjekt wird mitgegeben.

\typescript{code/Sequelize/init.ts}{Initialisierung einer Zugriffsklasse auf die Datenbank}

Der Prozess zum Setzten eines Fremdschlüssels ist jedoch noch nicht abgeschlossen, da man angeben muss, dass die Tabelle auf die referenziert wird viele unserer Objekte beinhalten. Umgekehrt aber nur eines der anderen Objekte auf unseres zutrifft. Im vorliegenden Beispiel bedeutet das: Ein "Room" kann mehrere "RoomReports" beinhalten. Ein "RoomReport" ist aber immer einem gewissen "Room" zugeordnet.

\typescript{code/Sequelize/fremdSchluessel.ts}{Finalisieren einer Fremdbeziehung zwischen Zwei Tabellen}

Bei den tatsächlichen Zugriffsmethoden gibt es wie oben schon erwähnt 4 unterschiedliche Arten

\hfive{Auslesen aller Datensätze}

Mit dieser Methode werden alle Datensätze einer Tabelle ausgelesen und in eine Konstante gespeichert. Dies ist beispielsweise für die Administrator*innen relevant, da diese eine Übersicht aller Meldungen für alle Räume benötigen um zu sehen, wie viele Schäden noch offen sind und welche schon abgearbeitet sind.

Hierbei ist es auch nicht erforderlich, dass der Server eine Raumnummer als Parameter mitgibt. Die von Sequelize implementierte Methode "findAll" greift dann auf die Tabelle zu. Diese kann jedoch auch mit mehreren Attributen versehen werden. Als Beispiel kann das nicht zurückliefern von bestimmten Spalten nennen. Sollte ein Fremdschlüssel gesetzt sein, was in diesem Beispiel der Fall ist, ist es außerdem notwendig das "Model" der zu referenzierenden Tabelle anzugeben und die Eigenschaft "required" auf "true" zu setzten. Zum Schluss wird die Konstante mit den Werten zurückgegeben.

\typescript{code/Sequelize/getAll.ts}{Zurückliefern aller Datensätze der Tabelle "RoomeReport"}

\hfive{Auslesen von Datensätzen nach Raumnummer gefiltert}

Mit dieser Methode werden nur einzelne Datensätze für einen bestimmten Raum ausgelesen. Dies ist beispielsweise für das hinzufügen von Warnungen vor Schäden zu einem Raum relevant. 

Die Struktur ist der Methode zum Auslesen aller "RoomReports" ähnlich, jedoch mit dem Unterschied, dass ein Parameter mit der Raumnummer mitgegeben werden muss. Nun wird als zusätzliches Attribut ein "where" hinzugefügt um die Datensätze nach einer gewissen Raumnummer zu filtern.

\typescript{code/Sequelize/getRoomNumber.ts}{Auslesen von bestimmten Datensätzen (nach Raumnummer gefiltert)}

\hfive{Verändern von vorhandenen Datensätzen}

Gerade beim Melden von Schäden in Räumen ist es wichtig, Datensätze welche schon vorhanden sind wieder zu verändern, da beim Eintrag in die Datenbank der Status einer Meldung darauf eingestellt ist, dass er noch nicht in Arbeit ist. Sobald er in Arbeit ist, kann eine Administrator*in den Status auf "in Arbeit" ändern, damit die Schüler*innen erkennen, dass sich um ihr gemeldetes Problem gekümmert wird. Zum Abschluss wird der Status auf "Erledigt" gesetzt.

Hierbei müssen Zwei Parameter mitgegeben werden, nämlich einerseits die "id" des Datensatzes als Zahl und der neue "ReportStatus" als Text. Nun kann mit der von Sequelize bereitgestellten Methode "update" der Datensatz verändert werden. Hierbei muss zuerst der Spaltenname und der neue Status, welcher als Parameter mitgegeben wurde angegeben werden und nachher mit einem "where" der richtige Datensatz anhand der mitgegebenen "id" gefunden werden.

\typescript{code/Sequelize/change.ts}{Ändern der Spalte "ReportDescription" in der Tabelle "RoomReport"}

\hfive{Hinzufügen von Datensätzen}

Wenn nun von einer Person, welche sich im Schulgebäude aufhält ein Schaden in einem Raum entdeckt wird bietet ZELIA dich Möglichkeit diesen zu melden. Der neu gemeldete Schaden muss dann in der Datenbank gespeichert werden können um später darauf zugreifen zu können und ihn als verantwortliche Person bearbeiten zu können.

Sequelize stellt zum Schrieben in eine Tabelle die Methode "create" zur Verfügung, welche direkt auf das "Model" angewandt werden kann. Nachfolgend muss jeder Spalte ein Wert zugeordnet werden. Die Daten hierfür werden in dem Beispiel aus dem mitgegebenen "RoomReport" Objekt entnommen.

\typescript{code/Sequelize/new.ts}{Hinzufügen eines neuen Datensatzes}

\hfour{Datenbankservice}

Durch die verschiedenen Verbindungsklassen ermöglicht man einen direkten Zugriff auf die Datenbank und die Tabellen in ihr. Der Datenbankservice ist nun für die Kommunikation mit dem restlichen Server zuständig. Das Zeil hinter diesem eigenen Services war es die Interaktion des restlichen Servers mit der internen Datenbankkommunikation so gering wie möglich zu halten. Durch den externen Service muss nur dieser aufgerufen werden. Der Datenbankservice übernimmt ab diesem Zeitpunkt die gesamte Aufrufshierarchie der Datenbank.

Außerdem wird über den Datenbankservice das Auftreten von Fehler gehandhabt. beispielsweise wird überprüft ob die Datenbank verfügbar ist, oder ob beim Verbindungsaufbau etwas nicht geklappt hat, oder der ausgewählte Raum in der Datenbank nicht vorhanden ist. Dies wird über verschiedene "Exceptions" realisiert welche einerseits den Entwicklern beim Programmieren helfen die Übersicht zu behalten und andererseits den Benutzer*innen im fertigen Produkt ein Feedback liefern, ob ihre Aktion erfolgreich war, oder nicht.

Beispiel für eine Exception:

\typescript{code/Sequelize/exception.ts}{Datenbank nicht verfügbar Exception}

Je nachdem, welche Aktion vom Server angefragt wird muss ein Parameter mitgegeben werden oder nicht. Wenn ein Parameter mitgegeben wird, wird dieser in die gewünschte Zugriffsmethode weitergereicht.

\typescript{code/Sequelize/all.ts}{Beispiel für eine Datenbankservicemethode}
