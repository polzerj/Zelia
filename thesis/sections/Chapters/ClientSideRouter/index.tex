\htwo{Client Side Router}
\label{sec:csrouter}
\sectionauthor{Julian Kusternigg}
\hthree{Allgemeines}

\hfour{Einleitung}

In der Netzwerktechnik ist "Routing" der Vorgang um einen Weg für Daten zu finden. In der Webentwicklung ist das recht ähnlich. Jede Seite hat einen bestimmten Pfad, eine Route, \zb:

{\ttfamily https://zelia/home} oder {\ttfamily .../room/S1308}

In diesen Fällen wären "{\ttfamily /home}" und "{\ttfamily /room/S1308}" zwei verschiedene Routen. Früher hat man am Webserver, also der Server, der die Daten zur Verfügung stellt, ein Verzeichnis gehabt, das alle möglichen Pfade abbildet.

\begin{figure}[H]
  
  \begin{singlespace}
  \begin{lstlisting}
www/
  > home/
  > room/
    > S1308/
    > ...
    \end{lstlisting}
  \end{singlespace}
    \caption{Ordnerstruktur eines Webservers}
\end{figure}

Diese Art von Webserver stellt statische Webseiten zur Verfügung. Statisch, weil die Daten, \zb: das "{\ttfamily index.html}", nicht geändert werden, sondern so wie sie am Server liegen dem Client geschickt werden. Dies hat den Vorteil, dass der Server kaum Ressourcen verbraucht. In jedem dieser Verzeichnisse kann man ein "{\ttfamily index.html}" finden, welches der Webserver schickt, wenn einer dieser Pfade vom Browser angefragt wird. Das Problem welches dabei auftritt ist, dass es für jeden Raum ein fertiges "{\ttfamily index.html}" im richtigen Pfad geben muss.

Als Lösung für dieses Problem gibt es sogenannte Router, die wissen was zu tun ist, wenn eine Route angefragt wird. Somit kann man die statische Hierarchie auflösen und auf jede Anfrage individuell reagieren. In der Webentwicklung werden diese Steuerungsprogramme, die auf Anfragen reagieren, meist "Controller" genannt. Ein "Controller" bekommt somit eine Anfrage und liefert eine Antwort, welche im Zusammenhang von vorher das "{\ttfamily index.html}" ist. Dabei ist aber wichtig zu erwähnen, dass der Inhalt des "{\ttfamily index.html}", also der Antworttext, modifiziert werden kann. So kann ein Router beispielsweise alle Routen die aus "{\ttfamily /room/...}" bestehen an einen "Controller" weitergeben, der dem Client jeweils andere Informationen zurückgibt.

Nochmal kurz zusammengefasst: Router sind in der Webentwicklung dazu da, um Routen (Pfade) aufzulösen und an den richtigen "Controller" weiterzureichen. "Controller" liefern dann individuelle Ergebnisse auf jeweilige Anfragen.

\hfour{Backend Routing}

Backend Routing wird der Vorgang genannt bei dem das Routing im Backend der Applikation, also Serverseitig, gemacht wird. Beispiele dafür wären wie oben genannt, statische Webserver oder dynamische Webserver, welche mit "Controllern" arbeiten.

\hfour{Frontend Routing}

Frontend Routing, auch "Client-Side-Routing" genannt, ist, wenn das Routing zum Client verschoben wird, also wenn die Anfragen im Browser verarbeitet werden. So muss ein Webserver nur eine einzige Seite zur Verfügung stellen die der Client lädt.

Im Beispiel von vorhin müsste der Webserver nur ein "{\ttfamily index.html}" hosten welches der Client, egal welche Route er besucht, herunterlädt. Das ist meist eine minimale HTML Seite, die eine Javascript-Datei einbindet. Dieses Skript schaut dann auf welchem Pfad es sich befindet und holt falls nötig Informationen von einer API, um eine Seite aufzubauen. Das "{\ttfamily index.html}" wird nur einmal zu Beginn heruntergeladen, wenn die Webseite besucht wird bzw. wenn man die URL manuell im Browser eingibt. Bei jedem Link auf der Webseite werden nur die notwendigen Infos nachgeladen und angezeigt. Zusätzlich wird mit Hilfe der "History-API" ein Eintrag in den Verlauf gemacht, sodass man mit der "Zurücktaste", auf die zuvor besuchten Pfade zurückspringen kann.

Obwohl die Seite eigentlich nie verlassen wird, ändert der Browser den Verlauf und den Pfad, der in der URL-Eingabe Box steht. Somit ist es möglich das Routing komplett in den Browser zu verschieben.

\hfour{Vorteile des Client-Side-Routers}

Einer der größten Vorteile ist, dass die eigentliche Webseite, also das "{\ttfamily index.html}", praktisch überall bereitgestellt werden kann. Es braucht, wie schon gesagt, kaum Ressourcen solche statische Seiten zur Verfügung zu stellen. Die Browser auf den Geräten, mit denen die Seite aufgerufen wird, haben eine komplexere Aufgabe die Seite richtig dazustellen. Mit heutigen Computern und Smartphones ist das allerdings auch kein Problem mehr.

Außerdem ist es oft schneller, nur notwendige Infos nachzuladen. Vor allem wenn man im Browser heruntergeladene Daten in Caches zwischenspeichert. Somit müssen am Anfang meist mehr Daten geladen werden, dafür hat man dann beim Wechseln der Seite kaum Ladezeiten. Zusätzlich kann man Musterwerte verwenden oder einen Ladebildschirm anzeigen, wenn die individuellen Werte noch nachgeladen werden. So kann der Browser schon mal Inhalte darstellen. Es muss nicht gewartet werden, bis eine gesamte Seite am Server zusammengestellt und geschickt wird.

Die Daten, die angezeigt werden sollen, werden von "API-Servern" zur Verfügung gestellt. Dort kann man zentral die Schnittstellen definieren und überwachen auf welche Daten zugänglich sind. Somit ist Server und Client besser voneinander getrennt. Wenn am Server die HTML Seiten generiert werden, kann dies bei komplexen Anwendungen schnell unübersichtlich werden.

\hthree{Implementierung}

Wie gerade erwähnt ist "Frontend Routing" ein bisschen komplizierter zu implementieren da, wenn man kein Framework wie z.B.: Angular verwendet, muss man praktisch alles selbst implementieren. Wir haben uns aber trotzdem dazu entschieden kein Framework zu verwenden und unsere eigenes kleines "Frontend Service" zu schreiben da wir im Rahmen von Tests  in Erfahrung gebracht haben, dass die meisten Frontend-Frameworks viel zu groß und komplex sind für unsere Anforderungen.

Aufteilen kann man unsere Client Bibliothek wie schon gesagt in 2 Teile: Das Komponentensystem und den Client-Side-Router. Das Komponentensystem wurde oben schon beschrieben (siehe Kapitel WebComponents \ref{sec:webcomponents}). Beide Teile der Bibliothek haben keine Abhängigkeiten und können getrennt voneinander laufen. Sie sind beide für den Anwendungsfall in ZELIA entwickelt, können aber auch problemlos einzeln für andere Web-Applikationen verwendet werden.

Wie vorhin erwähnt wird auch bei ZELIA ein einfaches "index.html" verwendet, um das Skript zu laden, welches sich um die gesamte Seite kümmert. Zusätzlich ist ein leerer Container mit dem Namen "app" definiert, der sich auf der Webpage befindet. Dieser Container wird vom Client-Side-Router verwendet um Seiten dynamisch in die Hauptseite, die nie wirklich verlassen wird, hineinzuladen. Wenn das Skript vom Browser gestartet wird, erstellt ZELIA ein Router Objekt mit folgenden Schnittstellen:

\typescript{code/CSRouter/interface.ts}{Schnittstellen der "Router" Klasse}

Als "Root-Element" wird im Konstruktor der "app"-Container mitgegeben. Der Inhalt von diesem Container wird jedes Mal, wenn eine neue Seite geladen wird, gelöscht und mit den neuen Daten befüllt. Somit gibt es die Möglichkeit statische Inhalte einzubauen, wie zum Beispiel eine Kopfzeile oder Impressum, ohne sie in jede Seite hinzufügen zu müssen.

Nachdem der Router erstellt wurde, werden die notwendigen Routen initialisiert:

\typescript[initRoutes]{code/CSRouter/init.ts}{Initialisierung der Routen}


Der erste Parameter ist der Pfad und der Zweite ist der Controller, welcher ausgeführt wird, wenn der Pfad aufgerufen wird. Ein Controller ist in unserem Fall eine einfache Methode, welche den App-Container befüllt.

\typescript{code/CSRouter/hw_example.ts}{Beispiel einer "Hello World"-Seite}

In dem Codebeispiel (Code \ref{initRoutes}) wird in den Zeilen 5. bis 7. der Pfadeintrag "/room/:roomNumber" verwendet. Der Router leitet somit alle "/room/..." auf den verknüpften Controller um und wenn dieser Aufgerufen wird, bekommt er als Parameter die Werte von dem Platzhaltern mitgeliefert:

\typescript{code/CSRouter/placeholder.ts}{Beispiel "Controller" mit Platzhalter}

Am Ende, nachdem alle Routen initialisiert wurden, sagen wir dem Router, wohin er Routen soll, also welcher Controller ausgeführt werden soll, um die nötigen Inhalte darzustellen. Dafür wird der Pfad verwendet, welcher in der Eingabezeile des Browsers mitgegeben wird. Somit ist es möglich das man den Pfad manuell eingeben kann und nicht immer auf der Anfangsroute "/" startet.

Wenn ein Link ausgewählt wird, fängt ZELIA die Anfrage an dem Server ab und verwaltet das Umleiten selbst:

\typescript{code/CSRouter/redirect.ts}{Umleitung in einem Event}

Im  Hintergrund wird jedes Mal beim Umleiten ein Eintrag in den Verlauf gemacht. Bei diesem Eintrag wird der besuchte Pfad gespeichert. Der Browser speichert den Verlauf einer Seite als Stapel, sprich man kann nur oben hinzufügen und das oberste Element runternehmen. Somit kann man das Event "onPop" welches der Browser feuert, abfangen und selbst mit dem Router Umleiten:

\typescript[code:routerhistory]{code/CSRouter/history.ts}{Abfangen und Umleiten wenn der Verlauf geändert wird}

Im Codebeispiel \ref{code:routerhistory} steht das Zweite Argument für "Silent History", womit kein Eintrag in den Verlauf gemacht wird.

Mit all den oben beschriebenen Maßnahmen kann der ZELIA Client-Service zwischen verschiedenen Seiten logisch wechseln ohne jemals die eigentliche Seite zu verlassen.
