\hthree{Entities in ZELIA}

In ZELIA gibt es im Ordner "data" einen Unterordner namens "entities". In diesem sind für jede Tabelle in der Datenbank eine "Entity" mit all ihren Eigenschaften enthalten. Abgewickelt wird das Ganze über ein "Interface", welches jeweiligen Eigenschaften enthält.

In dem Entity "Room" werden außerdem die Typen für Rauminfos eingebunden, da beispielweiße der Raumtyp als Datentyp für eine Eigenschaft verwendet wird.

Beispiel für eine Entity:

// Code Snippet RoomEntity
