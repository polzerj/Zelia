\hfour{Entities}

In \ZELIA\ gibt es im Ordner "data" einen Unterordner namens "entities". In diesem sind für jede Tabelle in der Datenbank eine "Entity" mit all ihren Eigenschaften enthalten. Abgewickelt wird es über ein "Interface", welches die jeweiligen Eigenschaften enthält.

Außerdem kommen hierbei die Raumtypen (siehe Kapitel "Raumtypen" \ref{sec:roomtype}, Seite \pageref{sec:roomtype}) zum Einsatz, indem sie für einzelne Eigenschaften im Interface verwendet werden.

Beispiel für eine Entity:

\typescript{code/Sequelize/entity.ts}{Entity Raum}