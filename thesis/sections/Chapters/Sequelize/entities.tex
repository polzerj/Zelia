\hfour{Entities in ZELIA}

In \ZELIA\ gibt es im Ordner "data" einen Unterordner namens "entities". In diesem sind für jede Tabelle in der Datenbank eine "Entity" mit all ihren Eigenschaften enthalten. Abgewickelt wird es über ein "Interface", welches die jeweiligen Eigenschaften enthält.

In dem Entity "Room" werden außerdem die Typen für Rauminfos eingebunden, da beispielsweise die Bezeichnung des Raumes als eigens definierter Datentyp für die Eigenschaft "RoomType" verwendet wird.

Beispiel für eine Entity:

\typescript{code/Sequelize/entity.ts}{Entity Raum}