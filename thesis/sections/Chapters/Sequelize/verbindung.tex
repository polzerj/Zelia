\hfour{Verbindung zur Datenbank aufbauen}

Um die Verbindung mit dem Datenbankserver herstzustellen ist folgendes notwendig:

Zu Beginn muss das Sequelize Framework eingebunden werden.

\typescriptsub{code/Sequelize/datenbankverb.ts}{Importieren von Sequelize}{1}{1}

Um eine Verbindung aufzubauen werden Parameter benötigt, welche bei \ZELIA\ in einer eigenen ".env"-Datei ausgelagert ist. Folgende Parameter sind für den Aufbau erforderlich:

\begin{itemize}
    \item "DB\_SERVER" -- Gibt an, über welcher Adresse der Datenbank Server erreichbar ist.
    \item "DB\_USER" -- Gibt an, über welchen User, welcher schon im Datenbanksystem angelegt sein muss, man sich zum Server verbindet.
    \item "DB\_PASSWORD" -- Gibt das passende Passwort zum angegebenen User mit.
    \item "DB\_DATABASE" -- Gibt die Datenbank an, auf welche zugegriffen werden soll. Diese muss schon im Datenbanksystem angelegt worden sein.
\end{itemize}

Nun werden die jeweiligen Variablen in eine lokal angelegte Konstante gespeichert. Zugegriffen wird auf die ".env"-Datei mit "{\ttfamily process.env}".

\typescriptsub{code/Sequelize/datenbankverb.ts}{Erstellen einer Konstante für die Umgebungsvariablen}{3}{3}

Weiters wird eine lokale Variable namens "sequelize" von der Klasse "Sequelize" angelegt. Diese Klasse wird durch das Importieren von "Sequelize" in der verwendeten Datei verfügbar.

\typescriptsub{code/Sequelize/datenbankverb.ts}{Lokale Variable von der Klasse Sequelize}{4}{4}

Diese lokale Variable wird nun mit "{\ttfamily = new Sequelize}" zu einem neuen Sequelize Objekt. Der Konstruktor von Sequelize, welcher in \ZELIA\ verwendet wird benötigt drei Parameter, nämlich einen Datenbanknamen, den Datenbankuser und das Passwort für den vorher angegebenen User. 

Nachfolgend können außerdem diverseste Optionen angegeben werden. Im Projekt \ZELIA\ werden zwei Optionen verwendet, nämlich der "Host", bei welchem die Adresse des Datenbankserver mitgegeben wird, und den "dialect". Dieser gibt an auf welches Datenbanksystem zugegriffen werden soll.

Folgender Code ist für das oben beschriebene zuständig:

\typescriptsub{code/Sequelize/datenbankverb.ts}{Erzeugen eines neuen Sequelizeobjekts}{6}{9}

Außerdem wird das Schlüsselwort "export" verwendet, da so das "Sequelize-Objekt" nun auch in anderen Dateien aufgerufen werden kann. Dies ist für den weiteren Datenbankzugriff und in weiterer folge die Verbindungsklassen wichtig.

\typescriptsub{code/Sequelize/datenbankverb.ts}{Exportieren des Sequelizeobjekts}{11}{11}

\hfive{Gesamter Code}

\typescript{code/Sequelize/datenbankverb.ts}{Verbindung zur Datenbank herstellen}

