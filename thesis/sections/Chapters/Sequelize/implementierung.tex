\hthree{Implementierung}

Für \ZELIA\ wird "Sequelize" verwendet, da die Software in ein MariaDB Datenbanksystem schreiben, Daten ändern, oder Datensätze auslesen können soll.

Durch das Einbinden der Bibliothek kann einerseits die Verbindung zum Datenbankserver hergestellt werden und andererseits auch verschiedene Methoden implementieren werden, welche auf die Daten aus der Datenbank zugreifen können. "Sequelize" stellt für zweiteres auch Methoden bereit, welche zum Beispiel bei der Suche auch durch verschiedene Eigenschaften eingeschränkt werden können (WHERE Parameter).

\hfour{Verbindung zur Datenbank aufbauen}

Um die Verbindung mit dem Datenbankserver herstzustellen, ist folgendes notwendig:

Zu Beginn muss das Sequelize Framework eingebunden werden.

\typescriptsub{code/Sequelize/datenbankverb.ts}{Importieren von Sequelize}{1}{1}

Um eine Verbindung aufzubauen werden Parameter benötigt, welche bei \ZELIA\ in einer eigenen ".env"-Datei ausgelagert ist. Folgende Parameter sind für den Aufbau erforderlich:

\begin{itemize}
    \item "DB\_SERVER" -- Gibt an, über welcher Adresse der Datenbank Server erreichbar ist.
    \item "DB\_USER" -- Gibt an, über welchen User, welcher schon im Datenbanksystem angelegt sein muss, man sich zum Server verbindet.
    \item "DB\_PASSWORD" -- Gibt das passende Passwort zum angegebenen User mit.
    \item "DB\_DATABASE" -- Gibt die Datenbank an, auf welche zugegriffen werden soll. Diese muss schon im Datenbanksystem angelegt worden sein.
\end{itemize}

Nun werden die jeweiligen Variablen in eine lokal angelegte Konstante gespeichert. Zugegriffen wird auf die ".env"-Datei mit "{\ttfamily process.env}".

\typescriptsub{code/Sequelize/datenbankverb.ts}{Erstellen einer Konstante für die Umgebungsvariablen}{3}{3}

Weiters wird eine lokale Variable namens "sequelize" von der Klasse "Sequelize" angelegt. Diese Klasse wird durch das Importieren von "Sequelize" in der verwendeten Datei verfügbar.

\typescriptsub{code/Sequelize/datenbankverb.ts}{Lokale Variable von der Klasse Sequelize}{4}{4}

Diese lokale Variable wird nun mit "{\ttfamily = new Sequelize}" zu einem neuen Sequelize Objekt. Der Konstruktor von Sequelize, welcher in \ZELIA\ verwendet wird, benötigt drei Parameter, nämlich einen Datenbanknamen, den Datenbankuser und das Passwort für den vorher angegebenen User. 

Nachfolgend können außerdem diverseste Optionen angegeben werden. Im Projekt \ZELIA\ werden zwei Optionen verwendet, nämlich der "Host", bei welchem die Adresse des Datenbankservers mitgegeben wird und den "dialect". Dieser gibt an, auf welches Datenbanksystem zugegriffen werden soll.

Folgender Code ist für das oben Beschriebene zuständig:

\typescriptsub{code/Sequelize/datenbankverb.ts}{Erzeugen eines neuen Sequelizeobjekts}{6}{9}

Außerdem wird das Schlüsselwort "export" verwendet, da so das "Sequelize-Objekt" nun auch in anderen Dateien aufgerufen werden kann. Dies ist für den weiteren Datenbankzugriff und in weiterer Folge für die Verbindungsklassen wichtig.

\typescriptsub{code/Sequelize/datenbankverb.ts}{Exportieren des Sequelizeobjekts}{11}{11}

\hfive{Gesamter Code}

\typescript{code/Sequelize/datenbankverb.ts}{Verbindung zur Datenbank herstellen}


\hfour{Typen für Rauminfos in ZELIA}

Um Informationen über einen Raum korrekt anzeigen zu können ist es wichtig, dass gewisse Eigenschaften nur gewisse Werte annehmen können. So gibt es dies etwas für den Raumtyp, den Tafeltyp, die Art des Beamers und die Verbindungsmöglichkeiten für den Beamer.

Im Code wird das folgendermaßen realisiert:

Zuerst wird wieder das Schlüsselwort "export" angegeben, um den Typ in allen File zugreifbar zu haben. Weiters muss diesem Typ einen Name bekommen. Die einzelnen Werte des Typen eingetragen werden. Im nachfolgenden Beispiel wird dadurch festgelegt, dass ein Raum nur einen der folgenden Werte annehmen darf.

\typescript{code/Sequelize/einfach.ts}{Einfachauswahl eines Raumtypen}

Eine Mehrfachauswahl wird durch ein Array zum Schluss ermöglicht.

\typescript{code/Sequelize/mehrfach.ts}{Mehrfachauswahl einer Verbindungsmöglichkeit für den Beamer}

\hthree{Entities in ZELIA}

In ZELIA gibt es im Ordner "data" einen Unterordner namens "entities". In diesem sind für jede Tabelle in der Datenbank eine "Entity" mit all ihren Eigenschaften enthalten. Abgewickelt wird das Ganze über ein "Interface", welches jeweiligen Eigenschaften enthält.

In dem Entity "Room" werden außerdem die Typen für Rauminfos eingebunden, da beispielweiße der Raumtyp als Datentyp für eine Eigenschaft verwendet wird.

Beispiel für eine Entity:

// Code Snippet RoomEntity

\hfour{Verbindungsklassen für die Datenbank}

Die Verbindungsklassen sind für die Kommunikation mit der Datenbank und deren unterschiedlichen Tabellen zuständig. Nach einem gewissen Initialisierungsprozess, welcher später beschrieben wird, folgen am Ende der Klasse Methoden. Diese Methoden sind für den Zugriff nach verschiedenen Kriterien zuständig (Daten nach einem gewissen Parameter oder alle Daten auslesen). Im Projekt \ZELIA\ gibt es vier unterschiedliche Arten von Zugriffsmethoden auf die Datenbank.

\begin{itemize}
    \item Das Auslesen aller Daten
    \item Das Auslesen von Daten nach mitgegebener Raumnummer
    \item Das Verändern von vorhandenen Datensätzen in der Datenbank
    \item Das Hinzufügen von Datensätzen in die Datenbank
\end{itemize}

Um die Struktur einer Verbindungsklasse besser zu verstehen, kann die Klasse "RoomReportConnection" angesehen werden.

Zu Beginn muss aus der Sequelize-Bibliothek "Models" und "DataTypes" importiert werden, um verschiedenste Methoden und Funktionen von "Sequelize" nutzen zu können. 

Folgender Code ist dafür zuständig:

\typescriptsub{code/Sequelize/RoomReport.ts}{Importieren von "Models" und "DataTypes"}{1}{8}

Im zweiten Schritt werden diverseste Datentypen aus \ZELIA\ in die Klasse importiert, welche in weiteren Schritten benötigt werden.

In weiterer Folge müssen nun die Werte, auf welche in der Tabelle, zugegriffen werden soll, definiert werden. Hierzu wird eine neue Klasse erstellt, welche das "RoomReportEntity" Interface beinhaltet und implementiert. In dieser Klasse werden nun die Eigenschaften mit Zugriffsoperatoren und Datentypen versehen. 

Außerdem werden an alle Eigenschaften außer "Id" ein "!" angehängt. Dieses gibt an, dass es sich um ein Pflichtfeld handelt. Bei der "Id" steht ein "?". Dieses steht für eine nicht zwangläufig auszufüllende Eigenschaft.

\typescriptsub{code/Sequelize/RoomReport.ts}{Klasse "RoomReport"}{10}{24}

Diese Klasse, welche das "Entityinterface" nun implementiert, muss nun nach den Sequelizestandard initialisiert werden. Hierbei wird auf die gewünschte Klasse, die von Sequelize zur Verfügung gestellte Methode "init", angewandt. Nachfolgend wird nun jeder Eigenschaft aus der Klasse (in diesem Fall RoomReport) mehrere Attribute mitgegeben (zum Beispiel ein "type" oder "allow null").

Außerdem werden Eigenschaften, welche in der Datenbank über einen Fremdschlüssel referenziert werden, mit dem Schlüsselwort "references" auf eine Spalte einer anderen Tabelle referenziert. Hierbei muss nachher noch das "Model" angegeben werden,
auf welches Bezug genommen wird und der Schlüssel, welcher referenziert werden soll.

Zum Schluss wird noch der Name der Tabelle angegeben, um festzulegen auf welche Tabelle Bezug genommen wird. Außerdem wird das "Sequelizeobjekt" mitgegeben.

\typescriptsub{code/Sequelize/RoomReport.ts}{Initialisierung einer Zugriffsklasse auf die Datenbank}{26}{76}

Der Prozess zum Setzten eines Fremdschlüssels ist jedoch noch nicht abgeschlossen, da angeben werden muss, dass die Tabelle, auf die referenziert wird, viele unserer Objekte beinhaltet. Umgekehrt aber nur eines der anderen Objekte auf unser Objekt referenziert. Im vorliegenden Beispiel bedeutet das: Ein "Room" kann mehrere "RoomReports" beinhalten. Ein "RoomReport" ist aber immer einem gewissen "Room" zugeordnet.

\typescriptsub{code/Sequelize/RoomReport.ts}{Finalisieren einer Fremdbeziehung zwischen Zwei Tabellen}{78}{79}

Bei den tatsächlichen Zugriffsmethoden gibt es wie oben schon erwähnt vier unterschiedliche Arten.

\hfive{Auslesen aller Datensätze}

Mit dieser Methode werden alle Datensätze einer Tabelle ausgelesen und in eine Konstante gespeichert. Dies ist beispielsweise für die Administrator*innen relevant, da diese eine Übersicht aller Meldungen für alle Räume benötigen, um zu sehen, wie viele Schäden noch offen und welche schon abgearbeitet sind.

Hierbei ist es auch nicht erforderlich, dass der Server eine Raumnummer als Parameter mitgibt. Die von Sequelize implementierte Methode "findAll" greift dann auf die Tabelle zu. Diese kann jedoch auch mit mehreren Attributen versehen werden. Als Beispiel kann das nicht-zurückliefern von bestimmten Spalten genannt werden. Sollte ein Fremdschlüssel gesetzt sein, was in diesem Beispiel der Fall ist, ist es außerdem notwendig, das "Model" der zu referenzierenden Tabelle anzugeben und die Eigenschaft "required" auf "true" zu setzen. Zum Schluss wird die Konstante mit den Werten zurückgegeben.

\typescriptsub{code/Sequelize/RoomReport.ts}{Zurückliefern aller Datensätze der Tabelle "RoomReport"}{109}{130}

\hfive{Auslesen von Datensätzen nach Raumnummer gefiltert}

Mit dieser Methode werden nur einzelne Datensätze für einen bestimmten Raum ausgelesen. Dies ist beispielsweise für das Hinzufügen von Warnungen vor Schäden zu einem Raum relevant. 

Die Struktur ist der Methode zum Auslesen aller "RoomReports" ähnlich, jedoch mit dem Unterschied, dass ein Parameter mit der Raumnummer mitgegeben werden muss. Nun wird als zusätzliches Attribut ein "where" hinzugefügt, um die Datensätze nach einer gewissen Raumnummer zu filtern.

\typescriptsub{code/Sequelize/RoomReport.ts}{Auslesen von bestimmten Datensätzen nach Raumnummer gefiltert}{81}{92}

\hfive{Verändern von vorhandenen Datensätzen}

Gerade beim Melden von Schäden in Räumen ist es wichtig, dass Datensätze, welche schon vorhanden sind, auch verändern zu können. In diesem Beispiel wird ein neuer Datensatz zu Beginn also noch nicht in Arbeit abgespeichert. Sobald der Schaden sich in Bearbeitung befindet, kann eine Administratorin oder ein Administrator den Status auf "in Arbeit" ändern, damit die Schüler*innen erkennen, dass sich um ihr gemeldetes Problem gekümmert wird. Zum Abschluss wird der Status auf "Erledigt" gesetzt.

Hierbei müssen zwei Parameter mitgegeben werden, nämlich einerseits die "id" des Datensatzes als Zahl und der neue "ReportStatus" als Text. Nun kann mit der von "Sequelize" bereitgestellten Methode "update" der Datensatz verändert werden. Hierbei muss zuerst der Spaltenname und der neue Status, welcher als Parameter mitgegeben wurde, angegeben werden und nachher mit einem "where" der richtige Datensatz anhand der mitgegebenen "id" gefunden werden.

\typescriptsub{code/Sequelize/RoomReport.ts}{Ändern von Spalten in der Tabelle "RoomReport"}{132}{144}

\hfive{Hinzufügen von Datensätzen}

Wenn nun von einer Person, welche sich im Schulgebäude aufhält, ein Schaden in einem Raum entdeckt wird, bietet \ZELIA\ die Möglichkeit diese zu melden. Der neu gemeldete Schaden muss dann in der Datenbank gespeichert werden können, um später darauf zugreifen zu können und ihn als verantwortliche Person bearbeiten zu können.

"Sequelize" stellt zum Schreiben in eine Tabelle die Methode "create" zur Verfügung, welche direkt auf das "Model" angewandt werden kann.

\typescriptsub{code/Sequelize/RoomReport.ts}{Hinzufügen eines neuen Datensatzes}{94}{107}

\hfour{Datenbankservice}

Durch die verschiedenen Verbindungsklassen wird ein direkter Zugriff auf die Datenbank und die Tabellen in ihr ermöglicht. Der Datenbankservice ist nun für die Kommunikation mit dem restlichen Server zuständig. 

Das Ziel hinter diesem eigenen Services war es die Interaktion des restlichen Servers mit der internen Datenbankkommunikation so gering wie möglich zu halten. Durch den externen Service muss nur dieser aufgerufen werden. Der Datenbankservice übernimmt ab diesem Zeitpunkt die gesamte Aufrufhierarchie der Datenbank.

Außerdem wird über den Datenbankservice das Auftreten von Fehler gehandhabt. beispielsweise wird überprüft ob die Datenbank verfügbar ist, oder ob beim Verbindungsaufbau etwas nicht geklappt hat, oder der ausgewählte Raum in der Datenbank nicht vorhanden ist. Dies wird über verschiedene "Exceptions" realisiert welche einerseits den Entwicklern beim Programmieren helfen die Übersicht zu behalten und andererseits den Benutzer*innen im fertigen Produkt ein Feedback liefern, ob ihre Aktion erfolgreich war, oder nicht.

Beispiel für eine Exception:

\typescript{code/Sequelize/exception.ts}{Datenbank nicht verfügbar Exception}

Je nachdem, welche Aktion vom Server angefragt wird muss ein Parameter mitgegeben werden oder nicht. Wenn ein Parameter mitgegeben wird, wird dieser in die gewünschte Zugriffsmethode weitergereicht.

\typescript{code/Sequelize/all.ts}{Beispiel für eine Datenbankservicemethode}
