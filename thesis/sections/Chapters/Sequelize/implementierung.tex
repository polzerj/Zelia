\hthree{Implementierung}

Für ZELIA ist Sequelize optimal, da wir in ein MariaDB Datenbanksystem schreiben müssen, Daten ändern können, oder eben Datensätze auslesen müssen. Mit Sequelize ist dies problemlos gewährleistet. Durch das Einbinden des Frameworks kann man einerseits die Verbindung zum Datenbankserver herstellen und andererseits auch verschiedene Methoden implementieren, welche auf die Daten aus der Datenbank zugreifen können. Sequelize stellt für zweiteres auch Methoden bereit, welche zum Beispiel bei der Suche auch durch verschiedene Eigenschaften eingeschränkt werden können (WHERE Parameter).

\hfour{Verbindung zur Datenbank aufbauen}

Um die Verbindung mit dem Datenbankserver herstzustellen, ist folgendes notwendig:

Zu Beginn muss das Sequelize Framework eingebunden werden.

\typescriptsub{code/Sequelize/datenbankverb.ts}{Importieren von Sequelize}{1}{1}

Um eine Verbindung aufzubauen werden Parameter benötigt, welche bei \ZELIA\ in einer eigenen ".env"-Datei ausgelagert ist. Folgende Parameter sind für den Aufbau erforderlich:

\begin{itemize}
    \item "DB\_SERVER" -- Gibt an, über welcher Adresse der Datenbank Server erreichbar ist.
    \item "DB\_USER" -- Gibt an, über welchen User, welcher schon im Datenbanksystem angelegt sein muss, man sich zum Server verbindet.
    \item "DB\_PASSWORD" -- Gibt das passende Passwort zum angegebenen User mit.
    \item "DB\_DATABASE" -- Gibt die Datenbank an, auf welche zugegriffen werden soll. Diese muss schon im Datenbanksystem angelegt worden sein.
\end{itemize}

Nun werden die jeweiligen Variablen in eine lokal angelegte Konstante gespeichert. Zugegriffen wird auf die ".env"-Datei mit "{\ttfamily process.env}".

\typescriptsub{code/Sequelize/datenbankverb.ts}{Erstellen einer Konstante für die Umgebungsvariablen}{3}{3}

Weiters wird eine lokale Variable namens "sequelize" von der Klasse "Sequelize" angelegt. Diese Klasse wird durch das Importieren von "Sequelize" in der verwendeten Datei verfügbar.

\typescriptsub{code/Sequelize/datenbankverb.ts}{Lokale Variable von der Klasse Sequelize}{4}{4}

Diese lokale Variable wird nun mit "{\ttfamily = new Sequelize}" zu einem neuen Sequelize Objekt. Der Konstruktor von Sequelize, welcher in \ZELIA\ verwendet wird, benötigt drei Parameter, nämlich einen Datenbanknamen, den Datenbankuser und das Passwort für den vorher angegebenen User. 

Nachfolgend können außerdem diverseste Optionen angegeben werden. Im Projekt \ZELIA\ werden zwei Optionen verwendet, nämlich der "Host", bei welchem die Adresse des Datenbankservers mitgegeben wird und den "dialect". Dieser gibt an, auf welches Datenbanksystem zugegriffen werden soll.

Folgender Code ist für das oben Beschriebene zuständig:

\typescriptsub{code/Sequelize/datenbankverb.ts}{Erzeugen eines neuen Sequelizeobjekts}{6}{9}

Außerdem wird das Schlüsselwort "export" verwendet, da so das "Sequelize-Objekt" nun auch in anderen Dateien aufgerufen werden kann. Dies ist für den weiteren Datenbankzugriff und in weiterer Folge für die Verbindungsklassen wichtig.

\typescriptsub{code/Sequelize/datenbankverb.ts}{Exportieren des Sequelizeobjekts}{11}{11}

\hfive{Gesamter Code}

\typescript{code/Sequelize/datenbankverb.ts}{Verbindung zur Datenbank herstellen}


\hfour{Typen für Rauminfos in ZELIA}

Um Informationen über einen Raum korrekt anzeigen zu können ist es wichtig, dass gewisse Eigenschaften nur gewisse Werte annehmen können. So gibt es dies etwas für den Raumtyp, den Tafeltyp, die Art des Beamers und die Verbindungsmöglichkeiten für den Beamer.

Im Code wird das folgendermaßen realisiert:

Zuerst wird wieder das Schlüsselwort "export" angegeben, um den Typ in allen File zugreifbar zu haben. Weiters muss diesem Typ einen Name bekommen. Die einzelnen Werte des Typen eingetragen werden. Im nachfolgenden Beispiel wird dadurch festgelegt, dass ein Raum nur einen der folgenden Werte annehmen darf.

\typescript{code/Sequelize/einfach.ts}{Einfachauswahl eines Raumtypen}

Eine Mehrfachauswahl wird durch ein Array zum Schluss ermöglicht.

\typescript{code/Sequelize/mehrfach.ts}{Mehrfachauswahl einer Verbindungsmöglichkeit für den Beamer}

\hthree{Entities in ZELIA}

In ZELIA gibt es im Ordner "data" einen Unterordner namens "entities". In diesem sind für jede Tabelle in der Datenbank eine "Entity" mit all ihren Eigenschaften enthalten. Abgewickelt wird das Ganze über ein "Interface", welches jeweiligen Eigenschaften enthält.

In dem Entity "Room" werden außerdem die Typen für Rauminfos eingebunden, da beispielweiße der Raumtyp als Datentyp für eine Eigenschaft verwendet wird.

Beispiel für eine Entity:

// Code Snippet RoomEntity
