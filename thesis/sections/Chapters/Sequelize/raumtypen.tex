\hthree{Typen für Rauminfos in ZELIA}

Um Informationen über einen Raum korrekt anzeigen zu können ist es wichtig, dass gewisse Eigenschaften nur gewisse Werte annehmen können. So gibt es dies etwas für den Raumtyp, den Tafeltyp, die Art des Beamers und die Verbindungsmöglichkeiten für den Beamer.

Im Code wird das folgendermaßen realisiert:

Zuerst wird wieder das Schlüsselwort "export" angegeben, um den Typ in allen File zugreifbar zu haben. Danach wird der Name angegeben und nach einem Ist gleich und einer runden Klammer auf werden nun die einzelnen Werte angegeben und dabei unter doppelte Hochkommer gestellt. Sollte man mehrere Werte auswählen können, so ist es erforderlich, nach dem Schließen der runden Klammer eckige Klammern zu machen, um dadurch ein Array zu generieren, welches eine Mehrauswahl zulässt.

//Code Snippet Einfachauswahl

//Code Snippet Mehrfachauswahl
