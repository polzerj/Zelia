\hfour{Typen für Rauminfos in ZELIA}\label{sec:roomtype}

Damit Informationen über einen Raum korrekt angezeigt werden können, ist es wichtig, dass gewisse Eigenschaften nur gewisse Werte annehmen können. Dies gibt es in \ZELIA\ etwa für den Raumtyp, den Tafeltyp, die Art des Projektors und die Verbindungsmöglichkeiten für den Projektor.

Im Code wird das folgendermaßen realisiert:

Das Schlüsselwort "export" wird angegeben, um den Typ in allen Dateien zugreifbar zu machen. Weiters muss dieser Typ einen Namen bekommen und die einzelnen Werte des Typen eingetragen werden. 

Das nachfolgende Beispiel zeigt, dass es ein Raum nur einen der eingetragenen Raumtypen annehmen darf.

\typescript{code/Sequelize/einfach.ts}{Einfachauswahl eines Raumtypen}

Bei einem Projektor gibt es in der Regel mehrere Möglichkeiten, eine Verbindung aufzubauen. Deshalb gibt es hier eine Mehrfachauswahl.

Die Auswahl mehrerer Werte wird durch ein Array zum Schluss ermöglicht.

\typescript{code/Sequelize/mehrfach.ts}{Mehrfachauswahl einer Verbindungsmöglichkeit für den Projektor}
