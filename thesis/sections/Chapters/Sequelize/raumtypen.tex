\hfour{Typen für Rauminfos in ZELIA}

Um Informationen über einen Raum korrekt anzeigen zu können ist es wichtig, dass gewisse Eigenschaften nur gewisse Werte annehmen können. So gibt es dies etwas für den Raumtyp, den Tafeltyp, die Art des Beamers und die Verbindungsmöglichkeiten für den Beamer.

Im Code wird das folgendermaßen realisiert:

Zuerst wird wieder das Schlüsselwort "export" angegeben, um den Typ in allen File zugreifbar zu haben. Weiters muss diesem Typ einen Name bekommen. Die einzelnen Werte des Typen eingetragen werden. Im nachfolgenden Beispiel wird dadurch festgelegt, dass ein Raum nur einen der folgenden Werte annehmen darf.

\typescript{code/Sequelize/einfach.ts}{Einfachauswahl eines Raumtypen}

Eine Mehrfachauswahl wird durch ein Array zum Schluss ermöglicht.

\typescript{code/Sequelize/mehrfach.ts}{Mehrfachauswahl einer Verbindungsmöglichkeit für den Beamer}
