\hthree{Installation von Sequelize}

Um "Sequelize" in einem Softwareprojekt nutzen zu können, ist es erforderlich die Bibliothek zu installieren. Hierfür kann der "Node Package Manager" (siehe Kapitel \ref{sec:npm} NPM) verwendet werden.

\typescript{code/Sequelize/install.bash}{Installation von Sequelize}

\cite{Sequelize}

Außerdem müssen Treiber für das jeweils verwendete Datenbanksystem nachinstalliert werden. Im nachfolgenden Beispiel werden Treiber für "MariaDB" installiert. "Sequelize" würde aber mehrere Datenbanksysteme unterstützen, unter anderem auch den Microsoft SQL-Server.

\typescript{code/Sequelize/installMariaDb.bash}{Installation der MariaDB Treiber \label{code:SeqMaria}}

\cite{SequInstall}
