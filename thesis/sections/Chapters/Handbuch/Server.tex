\htwo{Server}
\hthree{Installation}

Der ZELIA Server ist notwendig um den ZELIA Client zu verwenden. Da der Client eine Web-Applikation ist, muss diese von einem Server bereit gestellt werden. Diese Aufgabe übernimmt der ZELIA Server. Zusätzlich stellt dieser Server die ZELIA-API zu Verfügung, welche vom Client verwendet wird um Informationen der Räume abzufragen und zu schreiben. Die Datenbank, in der Informationen über die Räume gespeichert sind, kann auch mittels Container direkt auf diesem Server gestartet werden.

\hfour{Abhänigkeiten}

\hfive{NodeJS}

Bevor man den Server aufsetzen und verwenden kann, gibt es ein paar Abhänigkeiten die man installieren sollte. Die wichtigste Abhängigkeit, die man nur zum Entwickeln braucht, ist NodeJS, eine Javascript Engine welche ohne ein Browser-Frontend läuft (siehe Kapitel NodeJS \ref{sec:nodejs}). Installieren kann man NodeJS einfach von der offiziellen Webseite \emph{https://nodejs.org/} (siehe Abbildung \ref{fig:nodejsdownload}). Verison 16.14.0 bietet sich an, da die meißten Pakete diese Version unterstützen. Für ZELIA macht es aber keinen Unterschied, da der Server auch auf der neusten Version von Node läuft.

\begin{figure}[H]
    \centering
    \includegraphics[width=120mm]{media/Handbuch/nodejs.png}
    \caption{Installation von NodeJS}
    \label{fig:nodejsdownload}
\end{figure}

\hfive{Git}

Durch die Installation von Git, kann man sich das Abreiten mit dem Quellcode erleichtern, notwendig ist es aber nicht. Verwenden kann man das Programm als Befehl in der Konsole. Unter Linux den Meisten Linux Distributionen ist Git bereits vorinstalliert. Unter MacOS kann man Git herunterladen, wenn man die Apple Entwicklungswerkzeuge installiert, welche mit der Entwicklerumgebung Xcode mitgeliefert werden. Um Git auf Windows laufen zu lassen, kann man das Programm von der offiziellen Website \emph{https://git-scm.com/downloads} herunterladen.

GitHub stellt "GitHub Desktop" zur verfügung, welches auch ein integrietes Git mitbringt. Zusätzlich hat "GitHub Desktop" eine grafische  Benutzeroberfläche, die den Einstieg in GitHub und Git vereinfachen kann (\emph{https://desktop.github.com}). 

\hfive{Docker}

Docker baut, wie man in Kapitel \ref{sec:ContainerAndDocker} erfährt, auf Containertechnologien auf, die es möglich machen unabhängig von der Hardware ausgeführt zu werden. Rein theoretisch bräuchte man Docker nicht, da man die einzelnen Komponenten auch nebeneinander auf einem Computer mittels mehrerer NodeJS instanzen zum laufen bringen könnte. Um den Startvorgang zu vereinfachen und Hardware unabhängig zu machen, sollte man allerdings Docker verwenden. Herunterladen kann man "Docker Desktop", die Benutzeroberfläche von Docker, von \emph{https://www.docker.com/get-started}.

\begin{figure}[H]
    \centering
    \includegraphics[height=90mm]{media/Handbuch/dockerdesktop.png}
    \caption{Installation von Docker}
\end{figure}

\hfour{Quellcode}

Entwickelt wurde ZELIA mit Hilfe von Git und gelagert von GitHub. GitHub ist vereinfacht gesagt eine Platform auf der man Quellcode hosten kann um gemeinsam daran mit Versions Controlle (Git) arbeiten zu können. 

Von GitHub kann man sich einfach mit Git oder "GitHub Desktop" den Quellcode lokal als Arbeitskopie auf einen Computer kopieren. Wenn man kein Git verwenden kann oder will kann man den gesammten Code auch als ZIP-File herunterladen. Druch die Verwendung von Git sieht man allerding auf einem Blick ob und was sich an dem Quellcode bei Updates verändert hat.

\begin{figure}[H]
    \centering
    \includegraphics[width=100mm]{media/Handbuch/GitHub_Download.png}
    \caption{Herunterladen des Quellcodes von GitHub}
\end{figure}

\hfive{Git in der Komandozeile}

In der Komandoeingabe des jeweiligen Betriebsystems kann man sich mit dem einfachen Befehl \emph{git clone https://github.com/polzerj/Zelia.git} eine Arbeitskopie herunterladen. Dabei wird in dem Verzeichnis in dem man sich befindet ein Unterorder mit dem Namen "ZELIA" angelegt in dem der gesammte Code liegt.

\hfive{GitHub Desktop}

Nachdem man das Programm geöffnet hat wird man gefragt von wo man eine Arbeitskopie holen will. Hier muss man auswähen \emph{Aus dem Internet kopieren} und dann den URL von dem Repository angeben (siehe Abbildung \ref{fig:clonewithdesktop}). Nachdem man auf "Kopieren" gedrückt hat, wird der Quellcode in das ausgewählte Verzeichniss geladen. 

\begin{figure}[H]
    \centering
    \includegraphics[width=100mm]{media/Handbuch/clone_gh.png}
    \caption{Quellcode mit "GitHub Desktop" herunterladen}
    \label{fig:clonewithdesktop}
\end{figure}

\hthree{Starten des Servers}

Wenn der Quellcode nun auf dem beliebigen Server herunterladen wurde, muss man noch eine \emph{.env-Datei} im Unterverzeichniss "api" anlegen, in der die notwendigen Umgebungsvariablen stehen. Diese werden benötigt um zu defineren auf welchem Port der Server läuft, welche Anmeldedaten für die Abfrage von WebUntis verwendet werden und noch vieles mehr (siehe Abbildung \ref{fig:envexample}).

\begin{figure}[H]
    \begin{lstlisting}
PORT=3001
DB_SERVER=db
DB_USER=***
DB_PASSWORD=***
DB_DATABASE=Zelia
WEBUNTIS_SCHOOL=szu
WEBUNTIS_USERNAME=***
WEBUNTIS_PASSWORD=***
WEBUNTIS_BASE_URL=aoide.webuntis.com
EMAIL_NAME=***
EMAIL_PASSWORD=***
JWT_SECRET=ZEL1A
    \end{lstlisting}
    \caption{Beispiel einer .env-Datei}
   \label{fig:envexample}
\end{figure}

\hfour{Entwicklung}

Wenn der Server im Entwicklungsmodus gestartet wird, kann bei einer Änderung im Quellcode das veränderte Programmteil automatisch neugestartet werden. Um den Server in diesem Zustand zu starten muss man in dem ZELIA-Verzeichnis in den beiden Unterorder "pwa" und "api" jeweil folgendes Komando in der Konsole ausführen: \emph{npm install} oder verkürzt \emph{npm i}. Dadurch werden vom "Node Packet Manager" (siehe Kapitel "NPM" \ref{sec:npm}) alle Bibliotheken heruntergeladen die benötigt sind um den Server zum Laufen zu bringen.

Danach muss man unter MacOS und Windows "Docker Desktop" öffnen, damit die "Docker-Engine" gestartet wird. Auf Linuxsystemen ist dieser Vorgang nicht notwendig (siehe Kapitel \ref{sec:dockerexplained}). Nachdem die "Docker-Engine" gestartet wurde, kann man die Docker-Container starten mit folgendem Befehl starten:

\emph{docker-compose -f docker-compose.dev.yaml --env-file api/.env up}  

Beim ersten mal wird der Startvorgang etwas länger dauern da, im Hintergrund die Abbilder der Docker-Container heruntergeladen werden.

\hfour{Produktion}

So wie im Entwicklungsmodus muss man auch hier "Docker Desktop" öffnen, um Docker verwenden zu können. Statt \emph{npm i} in den beiden Unterverzeichnissen ausführen zu müssen, startet man nur die Docker-Container mit folgendem Befehl:

\emph{docker-compose -f docker-compose.yaml --env-file api/.env up}

Beim ersten mal wird es auch bei diesem Startvorgang etwas länger dauern, da erstens Docker die notwendigen Abbildungen herunterlädt und zweitens der NPM-Installationsvorgang durchläuft.