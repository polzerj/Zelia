\hfour{Cookies}

Web-Cookies sind Dateien auf einem Computer, die von einem Webserver geschrieben werden, um ein Client zu identifizieren. 
Bei der ersten Anfrage sendet ein Server die Webseite selbst und zusätzlich ein Cookie zurück. 
Dieses Cookie wird bei jeder weiteren Anfrage automatisch an den Server gesendet und kann dort aktualisiert werden. 
Da Cookies nicht nach Beenden der Sitzung gelöscht werden, werden diese auch bei erneuten Aufrufen mitgeschickt, wodurch \zb\ ermöglicht wird, dass der Benutzer auf einer Webapp angemeldet bleibt. 

Allerdings ist es auch möglich Cookies mit Javascript zu speichern, welches vom Server mit der Webseite mitgesendet wird. 
\cite{wikiCookies}
