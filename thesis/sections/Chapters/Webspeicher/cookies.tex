\hfour{Coockies}

Web-Cookies sind einfach kleine Dateien auf dem Rechner des Nutzers, die dort vom Webserver gespeichert werden, um den Client zu identifizieren. Bei der ersten Anfrage sendet der Server die Website selbst und zusätzlich das Cookie zurück. Dieses Cookie wird bei jeder weiteren Anfrage an den Server gesendet und kann dort aktualisiert werden. 

Es gibt drei Arten von Cookies:

\begin{itemize}
    \item Transient Cookies: Sie werden auch als Sitzungscookies bezeichnet. Sie werden im Arbeitsspeicher gespeichert und gelöscht, sobald der Browser geschlossen wird.
    \item Persistent Cookies: Sie werden auf der Festplatte des Nutzers gespeichert und bleiben dort so lange gespeichert, wie der Nutzer dies zulässt.
    \item Flash Cookies: Diese Cookies werden als persistente Cookies dauerhaft gespeichert, können aber vom Browser nicht gelöscht werden. 
\end{itemize}
\cite{w3Cookies}