\hfour{Indexed-DB}\label{sec:indexeddb}

Eine neue Entwicklung zur Speicherung von großen Datenmengen ist die Indexed-DB. 
Diese ist eine Datenbank, welche den Entwicklern ermöglicht, Daten in der Form von Objekten abzuspeichern.

Jede Datenbank enthält sogenannte Object-Stores. Das sind Sammlungen von Daten, welche als JSON-Objekte abgespeichert sind. 
Im Gegensatz zu relationalen Datenbanken ist bei der Indexed-DB kein festes Schema erforderlich. 
Dennoch ist immer ein Primärschlüssel zu definieren, welcher in diesem Object-Store für jedes Objekt einzigartig sein muss.
Dieser kann entweder ein Attribut des Objektes sein oder durch eine automatisch inkrementierende Ganzzahl generiert werden. 

Damit Datensätze nach einer Eigenschaft durchsucht werden können, wird ein Index erstellt. 
Ein indiziertes Attribut muss in jedem Objekt des Object-Stores vorhanden sein.

Im Vergleich zu Session-Storage und Local-Storage ist die API der Indexed-DB komplexer zu implementieren.

Um mit Indexed-DB arbeiten zu können, wird zunächst ein sogenannter "open request" erstellt:

\typescript{code/Webspeicher/openRequest.ts}{Erstellen eines Indexed-DB Open-Requests}

\begin{figure}[H]
    \centering
    \includegraphics[width=0.8\textwidth]{media/Webspeicher/openDb.png}
    \caption{Öffnen einer Indexed-DB \cite{fig:openDB}}
\end{figure}

Auf dem Open-Request-Objekt gibt es mehrere Events. 
Das Event "{\ttfamily upgradeneeded}" wird ausgelöst, wenn die geöffnete Datenbank noch nicht initialisiert wurde oder wenn die Version erhöht wurde. 
Die Version der Datenbank wird beim Erstellen des Open-Requests angegeben. 
Wenn sich das Schema der Datenbank ändern soll, wird diese Versionsnummer erhöht, damit das "{\ttfamily upgradeneeded}"-Event ausgelöst wird.
Innerhalb dieses Ereignisses werden die Object-Stores und die Indizes erstellt (siehe Code \ref{code:createDB}). 
Beim Erstellen des Object-Stores müssen der Name des Object-Stores und der Primärschlüssel definiert werden. Wenn der Primärschlüssel automatisch generiert werden soll, muss dies durch die Eigenschaft "autoIncrement", welche auf "true" gesetzt werden muss, angegeben werden.
Wenn der Object-Store erstellt ist, können Attribute indiziert werden. 
Dabei muss jedem Index ein Name gegeben und ein Attribut angegeben werden, nach welchem der Object-Store durchsucht werden soll. 
Außerdem kann definiert werden, ob dieses Attribut einzigartig (unique) sein soll.

\typescript[code:createDB]{code/Webspeicher/createDB.ts}{Erstellen und Ändern des Datenbankschemas}

Das Event "{\ttfamily success}" wird nach der vollständigen Initialisierung der Datenbank ausgelöst -- 
das heißt, nach dem "{\ttfamily upgradeneeded}"-Event, falls dieses erforderlich war. 
Wenn die Datenbank bereits initialisiert wurde, wird das "{\ttfamily success}"-Event sofort ausgelöst. 
Nach dem "{\ttfamily success}"-Event sind alle Arten von Manipulationen und Abfragen der Daten möglich. 
Zum Abrufen von Daten aus der Indexed-DB, wird eine Transaktion unter Verwendung des "Indexed Database Objekts" erstellt. 
Das "Indexed Database Objekt" ist als Ergebnis des "Open Request" abrufbar.

\typescript{code/Webspeicher/getByPrimaryKey.ts}{Abfragen von Datensätzen mit einem Primärschlüssel}

Es ist auch möglich, Daten über einen Index abzurufen. Da ein indiziertes 
% Zusammenhang methoden First/all Match
Attribut nicht eindeutig sein muss, gibt es zwei Methoden, um die Daten zu erhalten. Die Methode "{\ttfamily get}" liefert die erste Übereinstimmung und die Methode "{\ttfamily getAll}" alle Über\-ein\-stimmungen als Array.

\typescript{code/Webspeicher/getByIndex.ts}{Abfragen von Datensätzen mit einem Index}

Die Add-Methode des Object Stores erlaubt es Objekte hinzuzufügen. Die Put-Methode überschreibt bestehende Objekte.
\cite{MDNIndexedDB}
\cite{MDNUsingIndexedDB}