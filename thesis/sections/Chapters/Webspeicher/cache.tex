\hfour{Cache API}

Die Cache-API ermöglicht das Zwischenspeichern von Webseiten, damit diese offline genutzt werden können.
Sie wird meist in PWAs (Progressive Web Apps) verwendet. Eine PWA ist eine spezielle Webanwendung, welche anhand ihrer Eigenschaft erkannt werden kann. Dazu gehören die folgenden Eigenschaften \cite{datacodedesignPWA}:
\begin{itemize}
    \item Offline-Fähigkeit (Es wird auch eine Seite geladen, wenn der Browser keine Internetverbindung hat)
    \item Installierbarkeit 
    \item Ein Icon für die Seite
    \item Ein Responsive Design
    \item Sicherheit (Die Webseite verwendet HTTPS)
\end{itemize}
% zu viel PWA? 

Der Code für die Zwischenspeicherung wird in einen Service Worker geschrieben, der ein Javascript Skript ist, welches in einem separaten Thread läuft. 
Die Verwendung der Cache-API wird im Beispiel (Code \ref{code:cache}) vom Service Worker, Kapitel \ref{sec:cacheImpl} gezeigt.