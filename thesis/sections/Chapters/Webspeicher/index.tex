\htwo{Webspeicher}
\label{sec:webstorage}
\sectionauthor{Johannes Polzer}

\hthree{Einleitung}

Webspeicher (im Englischen "Web Storage" genannt) ermöglichen Entwicklern % tofix: Gendern? 
Daten im Webbrowser zu speichern.
Es gibt verschiedene Arten von Webspeichern, welche für unterschiedlicher Anwendungsfälle konzipiert wurden. Dazu zählen:

\begin{itemize}
    \item Local Storage -- vergleiche \ref{sec:localstorage}
    \item Session Storage -- vergleiche \ref{sec:sessionstorage}
    \item Cookies -- vergleiche \ref{sec:cookies}
    \item Indexed-DB -- vergleiche \ref{sec:indexeddb}
    \item Cache -- vergleiche \ref{sec:cache}
\end{itemize}

Damit die Daten nicht von anderen Web-Apps gelesen und manipuliert werden können, kommt eine  "Same-Origin-Policy" zum Einsatz (siehe Kapitel \ref{sec:sameorigin}).
% Damit die Session gespeichert und die Website offline aufgerufen werden kann, ist es wichtig, Daten im Webbrowser abzuspeichern. Es gibt verschiedene Arten von Speichern in einem modernen Webbrowser Um riesige Webanwendungen zu entwickeln, wurde die Menge an Speicherplatz in den letzten Jahren erhöht. 

\hfour{Speichermengen}

Damit eine Webanwendung nicht die gesamte Festplatte vollschreiben kann, wird die Speichermenge einer Webseite limitiert. Diese Limitierungen sind davon abhängig, welcher Webbrowser verwendet wird. \cite{WebDevStorage}

Der Google Chrome Browser erlaubt es, Webseiten bis zu achtzig Prozent der Fest\-platten\-kapazität zu nutzen. 

Firefox hat ein Limit von zwei Gigabyte pro Website. Insgesamt dürfen Webseiten bis zu 50 \% des freien Speicherplatzes nutzen. 

Apples Safari-Browser schränkt die Nutzung von Webanwendungen auf maximal ein Gigabyte ein. 
Wenn diese Grenze erreicht ist, wird der Benutzer alle 200 MB gefragt, ob er weitere 200 MB zulassen möchte. 
Dieses Verhalten ist von Apple nicht dokumentiert. 

\hfour{Same-Origin-Policy} \label{sec:sameorigin}

Damit nicht jede Webseite auf die gespeicherten Ressourcen der anderen Webseiten zugreifen kann, gibt es eine "Same-Origin-Policy".
Dadurch soll verhindert werden, dass Webseiten andere Webseiten ausspionieren, indem sie auslesen, mit welchen Accounts der Benutzer auf anderen Seiten angemeldet ist. Außerdem wird die Manipulation der Daten anderer Seiten unterbunden. \cite{MDNSame-origin-policy}

Dennoch kann Tracking mit Hilfe von Cookies durchgeführt werden.
Dabei wird auf einer Webseite \zb\ ein Script von Google Analytics eingebunden. 
Dieses Script kann dann auf die Daten der anderen Webseiten zugreifen und somit mit Tracking-Cookies das Benutzer\-verhalten verfolgen. \cite{TrackingCookies}


\hthree{Allgemeines}

\hfour{Local Storage}
Im Local Storage können Strings gespeichert werden. Dieser Speicher wird nie gelöscht, es sei denn, die Benutzer tun es selbst. 
\cite{w3LocalStorage}

\typescript{code/Webspeicher/localstorage.ts}{Schreiben und Lesen des Local-Storage}
\hfour{Session Storage}
Der Session Storage ist dem Local Storage sehr ähnlich. Der einzige Unterschied ist, dass die Daten nach dem Schließen des Tabs gelöscht werden. 

\typescript{code/Webspeicher/sessionstorage.ts}{Schreiben und Lesen des Session-Storage}
\hfour{Cookies}

Web-Cookies sind Dateien auf einem Computer, die von einem Webserver geschrieben werden, um ein Client zu identifizieren. 
Bei der ersten Anfrage sendet ein Server die Webseite selbst und zusätzlich ein Cookie zurück. 
Dieses Cookie wird bei jeder weiteren Anfrage automatisch an den Server gesendet und kann dort aktualisiert werden. 
Da Cookies nicht nach Beenden der Sitzung gelöscht werden, werden diese auch bei erneuten Aufrufen mitgeschickt, wodurch \zb\ ermöglicht wird, dass der Benutzer auf einer Webapp angemeldet bleibt. 

Allerdings ist es auch möglich Cookies mit Javascript zu speichern, welches vom Server mit der Webseite mitgesendet wird. 
\cite{wikiCookies}

\hfour{Indexed-DB}

Eine ganz neue Art der Speicherung ist Indexed DB. Indexed DB ist eine Datenbank, die in den Webbrowser eingebaut ist, um eine große Menge an Daten als Objekte zu speichern. Jede Datenbank enthält so genannte Object-Stores. Dies sind Sammlungen von Daten. Innerhalb der Object-Stores können die Objekte gespeichert werden. Für jeden Object-Store muss ein Primärschlüssel definiert sein. Dies kann entweder ein Attribut des Objekts oder eine automatisch inkrementierende Ganzzahl sein. Um Datensätze durch die Suche nach anderen Eigenschaften schneller zu finden, kann ein Index erstellt werden. Diese indizierten Attribute müssen in jedem Eintrag des Object-Stores gesetzt werden.

Die Indexed DB API ist allerdings nicht so einfach wie die für die Session- oder den Local-Storage. 

Um mit Indexed DB arbeiten zu können muss zunächst ein sogenannter open request erstellt werden:

\typescript{code/Webspeicher/openRequest.ts}{Erstellen eines Indexed-DB Open-Requests}

Auf dem Open Request Objekt gibt es mehrere Events. Das Event "upgradeneeded" wird ausgelöst, wenn die geöffnete Datenbank noch nicht erstellt wurde oder wenn die Version erhöht wurde. Innerhalb dieses Ereignisses müssen die Object Stores und die Indizes erstellt werden: 

\typescript{code/Webspeicher/createDB.ts}{Erstellen und ändern des Datenbankschemas}

Das Event "Success" wird nach der Einrichtung der Datenbank ausgelöst, falls erforderlich. Nach dem Success Event sind alle Arten von Manipulationen und Abfragen der Daten möglich. Um Daten aus der Indexed DB abzurufen, muss eine Transaktion unter Verwendung des Indexed Database Objekts erstellt werden. Das Indexed Database Objekt ist als Ergebnis des Open Request abrufbar.

\typescript{code/Webspeicher/getByPrimaryKey.ts}{Abfragen von Datensätzen mit einem Primärschlüssel}

Es ist auch möglich, Daten über einen Index abzurufen. Da ein indiziertes 
% Zusammenhang methoden First/all Match
Attribut nicht eindeutig sein muss, gibt es zwei Methoden, um die Daten zu erhalten. Die Methode get liefert die erste Übereinstimmung oder getAll liefert alle Übereinstimmungen als Array.

\typescript{code/Webspeicher/getByIndex.ts}{Abfragen von Datensätzen mit einem Index}

Die Add-Methode des Object Stores erlaubt es Objekte hinzuzufügen und die Put-Methode überschreibt bestehende.
\cite{MDNIndexedDB}
\cite{MDNUsingIndexedDB}
\clearpage
\hfour{Cache API}

Die Cache-API ermöglicht das Zwischenspeichern von Webseiten, damit diese offline genutzt werden können.
Sie wird meist in PWAs (Progressive Web Apps) verwendet. Eine PWA ist eine spezielle Webanwendung, welche anhand ihrer Eigenschaft erkannt werden kann. Dazu gehören die folgenden Eigenschaften \cite{datacodedesignPWA}:
\begin{itemize}
    \item Offline-Fähigkeit (Es wird auch eine Seite geladen, wenn der Browser keine Internetverbindung hat.)
    \item Installierbarkeit 
    \item Ein Icon für die Seite
    \item Ein Responsive Design
    \item Sicherheit (Die Webseite verwendet HTTPS.)
\end{itemize}
% zu viel PWA? 

Der Code für die Zwischenspeicherung wird in einen Service Worker geschrieben, der ein Javascript Skript ist, welches in einem separaten Thread läuft. 
Die Verwendung der Cache-API wird im Beispiel (Code \ref{code:cache}) vom Service Worker, Kapitel \ref{sec:cacheImpl} auf Seite \pageref{code:cache}, gezeigt.
\clearpage

\hfour{Speicherung des JWT}

% TODO

\hfour{Chache für schnelle Ladezeiten}

Damit die Anwendung alle Voraussetzungen erfüllt um als PWA installiert zu werden, muss die Webseite im Cache gespeichert werden. Somit kann die Webseite bzw. die installierte PWA offline genauso geöffnet werden. Allerdings wird bei der offline-Nutzung eine Meldung angezeigt, dass keine Rauminformationen verfügbar sind.

Ein Beispiel für die Implementierung des Caches befindet sich im Kapitel \ref{sec:cacheImpl}.
 