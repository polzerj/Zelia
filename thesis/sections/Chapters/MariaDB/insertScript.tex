\hfour{Script zum Erzeugen von Daten zu Testzwecken}

Um unterschiedliche Softwaretests des Produkts durchführen zu können, wurden Testdaten in das Datenbanksystem eingebunden.

\sql{code/MariaDb/DatabaseInsertScript.sql}{Script zum Einfügen von Dummydaten}

Die Datenbank wird erneut in Verwendung gebracht um die Daten in die gewünschten Tabellen, welche sich in der Datenbank befinden, zu erstellen. Hierfür wird der Befehl "USE" verwendet.

\sql{code/MariaDb/use.sql}{Verwenden einer Datenbank}

Durch den Befehl "INSERT INTO" können Datensätze in einer Tabelle angelegt werden. Die Tabelle sollte im Vorhinein schon durch "CREATE TABLE" erstellt worden sein.

\sql{code/MariaDb/insertAdminUser.sql}{Aufrufen von "INSERT INTO"}

Mit dem Befehl "VALUES" werden dann die Datenwerte zum dazugehörigen Datensatz angegeben.

\sql{code/MariaDb/insertValues.sql}{Einfügen von Werten in eine Tabelle}
