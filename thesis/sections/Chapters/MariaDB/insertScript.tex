\hfour{Script zum Erzeugen von Daten zu Testzwecken}

Um unterschiedliche Softwaretests des Produkts durchführen zu können, wurden Testdaten in das Datenbanksystem eingebunden.

Die Datenbank wird erneut in Verwendung gebracht, um die Daten in die gewünschten Tabellen, welche sich in der Datenbank befinden, zu erstellen. Hierfür wird der Befehl "USE" verwendet.

\sqlsub{code/MariaDb/DatabaseInsertScript.sql}{Verwenden einer Datenbank}{1}{1}

Durch den Befehl "INSERT INTO" können Datensätze in einer Tabelle angelegt werden. Die Tabelle sollte im Vorhinein schon durch "CREATE TABLE" erstellt worden sein.

\sqlsub{code/MariaDb/DatabaseInsertScript.sql}{Aufrufen von "INSERT INTO"}{3}{4}

Mit dem Befehl "VALUES" werden dann die Datenwerte zum dazugehörigen Datensatz angegeben.

\sqlsub{code/MariaDb/DatabaseInsertScript.sql}{Einfügen von Werten in eine Tabelle}{5}{7}

\pagebreak

\hfive{Gesamtes Script}

\sql{code/MariaDb/DatabaseInsertScript.sql}{Script zum Einfügen von Dummydaten}
