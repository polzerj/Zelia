\hthree{Geschichte zu MariaDb und MySQL}

Der Startschuss zum älteren der beiden Systeme (MySQL) wurde bereits im Jahr 1980 gelegt. 15 Jahre vergingen und MySQL wurde als fertig entwickeltes Datenbanksystem herausgebracht. 2001 wurde das System dann auch offiziell zu einem Unternehmen, der MySQL AB. Im Jahr 2008 wurde das Unternehmen von der Firma Sun Microsystems, welche damals eine der bedeutendsten IT-Unternehmen war, gekauft und übernommen. Ab 2010 gliederte sich Sun Microsystems und damit auch MySQL in die Unternehmensgruppe von Oracle Corporation ein. Zwei Namen, welche im Zusammenhang mit dieser Historie erwähnt werden müssen, sind jene von oben schon genannten Michael Widenius und Kaj Arnö. \cite{MariaMy}

Das Projekt MariaDB wurde 2009 gestartet und wurde damals als Systemunterzweig von MySQL geführt, aber als eigenes Unternehmen, nämlich der MariaDB AB geführt. Es legte primär den Fokus auf die Verwaltbarkeit und die Betreuung von MySQL aber auch MariaDB Datenbanksystemen. Die MariaDB AB bietet deshalb beispielsweiße eine Remote-Datenbankbetreuung oder einen Remote Support an. \cite{MariaMy}

Das besondere an beiden Unternehmen, oder Datenbanksystemen ist, dass beides Open-Source Projekte sind und damit einen Meilenstein in der Verfügbarkeit und vor Allem der Weiterentwicklung von Datenbanksystemen gemacht haben und weiterhin machen. Die MariaDB AB finanziert sich zum Beispiel über die MariaDB Foundation. Diese gemeinnützige Stiftung macht es möglich, dass Alle, welche das Open-Source Projekt weiterentwickeln und voranbringen wollen das auch können. \cite{MariaMy}

Zu der Gründung von MariaDB und damit der Abspaltung von MySQL kam es, da sich die geplante Übernahme von Sun Microsystems durch die Oracle Corporation abzeichnete. Oracle wird von vielen Personen in der Open-Source Community äußerst kritisch gesehen und ist durchaus umstritten. Michael Widenius, welcher MariaDB mit anderen Kernentwicklern von MySQL gründete, hatte Angst, dass Oracle in Konflikt mit den Open-Source lastigen Visionen und Träumen von Ihm gerät und war der Ansicht, dass diese Zusammenarbeit nicht gut enden würde. Deshalb formulierte er drei entscheidende Ziele, welche das neue Datenbanksystem auf jeden Fall erfüllen muss: \cite{MariaMy}

\begin{itemize}
    \item MariaDB soll die Grundfunktionalitäten von MySQL beibehalten.
    \item Die Entwicklung, welche durch die Open-Source Community passierte, weiter zu forcieren und voranzutreiben.
    \item Die Sicherstellung, dass eine frei zugängliche und Open-Source basierte Version von MySQL für Alle erhältlich ist.
\end{itemize}

\cite{MariaMy}