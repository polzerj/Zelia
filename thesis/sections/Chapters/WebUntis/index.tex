\htwo{WebUntis}
\sectionauthor{Mersed Kečo}

\hthree{Einleitung}

WebUntis ist der zentrale Ort, wo die Planung für eine Schule vorgenommen werden kann. Entscheidungen sind mithilfe von WebUntis leichter zu treffen und Informationen sind immer direkt griffbereit. Dies ist nicht nur für Lehrer*innen und die Administration der Schule, sondern auch für Schüler*innen und deren Eltern abrufbar. Da das Schulzentrum Ungargasse WebUntis jetzt schon einige Zeit verwendet, bietet es eine solide Grundlage für den Informationsgewinn- und Verarbeitung. Damit das \ZELIA-Team überhaupt eine Möglichkeit besitzt, auf Stundenpläne von Klassen und auf bestimmte Räume zugreifen zu können, benötigt es eine "REST-API". In folgenden Fall gibt es einen bereits bestehenden "Node.js"-Wrapper der WebUntis-API mit einer weiterhelfenden Dokumentation zu den darin bestehenden Methoden und den zu verwendenden Parametern. Dieser wird nicht direkt von WebUntis bereitgestellt, sondern von einer kleinen Community auf GitHub. \cite{WebUntisWrapper}

\begin{figure}[H]
    \centering
    \includegraphics{media/WebUntis/WebUntisLogo.png}
    \caption{WebUntis Logo \cite{WebUntisLogo}}
\end{figure}

\hfour{Geschichte}

Untis wurde am 7.Jänner 1970 von den beiden Programmierern, Bernhard Gruber und Heinz Petters gegründet. Der Hintergrund, warum sich die beiden vor allem für einen Online-Stundenplan entschieden haben in einer Zeit, wo noch nicht einmal der Gedanke eines Online-Klassenbuches im Raum stand, ist nicht ersichtlich. Dennoch prägt ihre bahnbrechende Entwicklung nicht nur österreichische, sondern auch über 26.000 Schulen international mit einem sehr großen Supportnetz. \cite{Untis}

\hthree{Implementierung}

Um den Wrapper verwenden zu können, benötigt man zuerst das WebUntis-Package, welches sich wie folgt herunterladen und installiert lässt:



\emph{npm i webuntis}

Durch den Wrapper besteht die Möglichkeit, nicht nur auf den Stundenplan und gewisse Informationen zuzugreifen, sondern auch zum Beispiel Informationen zur besuchten Fachabteilung oder das Anzeigen von zu erledigenden Hausübungen. In diesem Fall benötigt \ZELIA\ folgende Methoden, die teilweise mit anderen Methoden verwendet werden, um auf das Wunschergebnis zu kommen:


\typescript[code:ConWebAPI]{code/WebUntis/constructor.ts}{Konstruktor von der WebUntis API}

Der Konstruktor (in Code \ref{code:ConWebAPI}) muss so verwendet werden, da sonst nicht auf das WebUntis-Objekt zugregriffen werden kann. Bei \ZELIA\ wird es genauso verwendet. Der einzige Unterschied ist aber hierbei, dass die verwendeten Variablen und die dazu entsprechenden Werte in einer ".env"-Datei abgespeichert sind, damit erstens, die Daten alle gesammelt an einer Stelle liegen und zweitens, da somit ein wenig Sicherheit hineingebracht werden kann.

\typescript{code/WebUntis/.env.ts}{".env"-Datei von \ZELIA}

getRooms wird verwendet, um alle Räume für ein Schulgebäude herauszufinden.

\typescript{code/WebUntis/getRooms.ts}{Rückgabe aller Raumnummern}
\begin{minipage}{\textwidth}
    Dies ist nur ein kleiner Ausschnitt der gesamten Räume des SZU:
    
    \typescriptsub{code/WebUntis/Rooms.json}{Ausgabe der getRooms Methode}{10}{30}
\end{minipage}

\begin{minipage}\textwidth
    Mithilfe der Methode "getTimetableFor" und den mitzugebenden Parametern kann der Stundeplan für einen speziellen Typen angezeigt werden lassen.
    
    \typescript{code/WebUntis/getTimetableFor.ts}{Methode um Stundenplan einer Klasse abzufragen}
\end{minipage}


Der WebUntisElementType wird als Wert übergeben und je nachdem, welcher Wert das ist, wird der Stundenplan ausgegeben. 

\typescript{code/WebUntis/WebUntisElementType.ts}{WebUntisElementType}

In der Ausgabe werden also alle wichtigen Parameter angezeigt, sei es das Fach, die Klasse die momentan unterrichtet wird oder aber auch der Lehrer, der die Stunde abhält.

\typescript{code/WebUntis/getTimetableForAusgabe.ts}{Ausgabe des Stundenplans}

\begin{minipage}\textwidth
    Die Methode "getTimeGrid" wird verwendet, um die Struktur des Stundenplans anzuzeigen. Mit der Struktur sind die Unterrichtsstunden, deren Anfangs– und Endzeiten gemeint.

\typescript{code/WebUntis/getTimegrid.ts}{Methode um die Stundenplanstruktur abzufragen}

Das Ergebnis sieht bei \ZELIA\ wie folgt aus:

\typescript{code/WebUntis/getTimegridAusgabe.ts}{Ausgabe des Standardstundenplans}

\end{minipage}

\pagebreak
\hfour{Probleme}

Mithilfe der API ist einiges möglich, dennoch sind in dem Anwendungsfall \ZELIA\ einige "Probleme" aufgefallen, die abgefangen beziehungsweise ausgebessert werden mussten.

Das allererste Problem ist, dass gewisse Räume mit mehreren Namen beziehungsweise mehreren IDs eingetragen sind und anfänglich die Frage im Raum stand, nach welchen Kriterien der Klassenraum auffindbar ist. Jedoch wurde bei \ZELIA\ die Entscheidung getroffen, nach der ID des Raumes zu suchen.

\typescript{code/WebUntis/getIDByRoomNumber.ts}{Raum über ID zurückbekommen}


Die Methode bekommt einen Wert des Typen "RoomNumber" übergeben. Gleichzeitig liefert die Methode getRoomList() alle eingetragenen Klassen, bei der in einer for-Schleife durch jedes Element durchgegangen wird und passende Übereinstimmungen ausgegeben werden. Die Vorgehensweise scheint zwar unkonventionell, doch während der Implementierung ist aufgefallen, dass der vollständige Klassenname nicht immer im Longname oder im Shortname abgespeichert, sondern eher rein zufällig in einem der beiden Parametern beinhaltet ist. Deswegen wird auf Nummer sicher gegangen und beide Parameter geprüft und geben bei einer Übereinstimmung die dazu vorhandene Raum-ID mit, diese ist einzigartig dennoch können, wie schon zuvor gesagt, mehrere IDs auf einen Raum zeigen, was in dem Fall kein allzu großes Problem darstellt.

Des Weiteren gab es das Problem, dass WebUntis einen Token eingebaut hat, welcher nach einer gewissen Zeit abläuft und die momentane Sitzung als "ungültig" deklariert. In diesem Fall weitergehend mit der Methode "validateSession()" überprüft, ob der Token zum Zeitpunkt der Abfrage schon abgelaufen oder noch gültig ist. 

\typescript{code/WebUntis/validateSession.ts}{Abfrage eines "gültigen" Tokens}


Falls der Token zum Zeitpunkt der Abfrage ungültig ist, wird die Variable "valid" auf "false" gesetzt und springt nach der if-Abfrage in den else-Zweig. Dort wird automatisch eine Neuanmeldung mit den Informationen des Benutzers getätigt und es wird erneut versucht, die gewünschte Methode auszuführen.


Während dem Exception-Handling ist \ZELIA\ folgender Fehler seitens WebUntis aufgefallen:

\typescript[code:ExceptionHandling]{code/WebUntis/ExceptionHandling.ts}{ExceptionHandling}

Wenn der WebUntis-Server kein Ergebnis auf eine Anfrage zurückgesendet, schickt dieser eine Exception-Message, die in diesem Fall abfangen wird. Leider ist aber die Nachricht falsch geschrieben und der Fehler ist nicht direkt ersichtlich, außer es wird sehr genau gelesen. Dieser Fehler wird aber mit dem Code in der Abbildung \ref{code:ExceptionHandling} abgefangen und macht weitergehend keine Probleme im Code.