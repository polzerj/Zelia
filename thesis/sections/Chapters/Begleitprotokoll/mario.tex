\hthree{Mario Naunović}
\sectionauthor{Mario Naunović}

\begin{description}
    \item[20.09.2021-26.09.2021] Auffrischen der "Node.js" Kenntnisse. Kennenlernen der WebUntis-API und erste Gedanken zur Verwendung der WebUntis-API.
    \item[27.09.2021-03.10.2021] Erste Experimente mit den verschiedenen Funktionen der WebUntis-API, um Informationen zu bestimmten Lehrräumen zu bekommen. Erste Arbeiten an der Zelia-API. 
    \item[04.10.2021-10.10.2021] Umschreiben des Backend-Codes von JavaScript in Typescript. Weitere Planung der Struktur der Zelia-API.
    \item[11.10.2021-17.10.2021] Erste Implementierungen von Funktionen zur Beschaffung von Rauminformationen. Meeting bezüglich der „Alpha Version“. Designen der Benutzeroberfläche. 
    \item[18.10.2021-24.10.2021] Implementierung von Funktionen zur Beschaffung der Raumnamen. Finale Arbeiten an der „Alpha Version“.
    \item[25.10.2021-31.10.2021] Implementierung von Funktionen zur Beschaffung der Stundenpläne. Schreiben von Funktionen, die die Präfixe der Raumnamen entfernt. Meeting mit dem Team vor den Herbstferien. 
    \item[01.11.2021-07.11.2021] Recherche zu REST und OpenAPI. Erste Arbeiten an der Dokumentation. Untersuchen des Fehlers bei der API durch auslaufenden Token. 
    \item[08.11.2021-14.11.2021] Meeting mit Diplomarbeitsbetreuern.
    \item[15.11.2021-21.11.2021] Weiterschreiben an Dokumentation
    \item[22.11.2021-28.11.2021] Erste Implementierung des Errorhandlings in der Middleware schreiben. Meeting mit Diplomarbeitsbetreuern bezüglich der „Beta“-Version. Zusammen mit Mersed Kečo Implementierung des Room Booking Controllers.
    \item[29.11.2021-05.12.2021] Schreiben an der Dokumentation. Anpassen der http-Errorcodes, die zurückgeliefert werden. Weiteres Planen bezüglich der „Beta“-Version.
    \item[06.12.2021-12.12.2021] Implementierung des Login-Controllers. Implementierung des AdminRequest-Controllers. Schreiben an der Dokumentation. 
    \item[13.12.2021-19.12.2021] Zusammen mit Mersed Kečo Fehler durch auslaufenden Token abfangen und erneuern. Weiterschreiben an der Dokumentation. 
    \item[20.12.2021-26.12.2021] Weiterschreiben an der Dokumentation.
    \item[27.12.2021-02.01.2022] Weiterschreiben an der Dokumentation.
    \item[03.01.2022-09.01.2022] Weiterschreiben an der Dokumentation.
    \item[10.01.2022-16.01.2022] Weiterschreiben an der Dokumentation.
    \item[17.01.2022-23.01.2022] Weiterschreiben an der Dokumentation.
    \item[24.01.2022-30.01.2022] Weiterschreiben an der Dokumentation.
    \item[31.01.2022-06.02.2022] Weiterschreiben an der Dokumentation.
    \item[07.02.2022-13.02.2022] Weiterschreiben an der Dokumentation.
    \item[14.02.2022-20.02.2022] Weiterschreiben an der Dokumentation. Erlernen der LaTeX-Syntax.
    \item[21.02.2022-27.02.2022] Weiterschreiben an der Dokumentation. Installieren von LaTeX. 
    \item[28.02.2022-06.03.2022] Weiterschreiben an der Dokumentation. Umstellung von Word auf LaTeX.
    \item[07.03.2022-13.03.2022] Weiterschreiben an der Dokumentation. Korrekturlesen von Mersed Kečo’s Teil.
    \item[14.03.2022-20.03.2022] Korrekturlesen durch Diplomarbeitsbetreuer.
    \item[21.03.2022-27.03.2022] Verbesserung der Dokumentation.
    \item[28.03.2022-01.04.2022] Brennen der Repository-CD. Abgabe der Diplomarbeit.     
\end{description}