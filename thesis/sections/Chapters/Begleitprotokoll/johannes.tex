\hthree{Johannes Polzer}
\sectionauthor{Johannes Polzer}

\begin{description}
    \item[20.09.2021-26.09.2021] Beginn der Entwicklung des Komponentensystems und des "Client-Side-Routers". Erste Schritte für OCR Implementierung. Beides mit Julian Kusternigg. Unterstützung beim Verbinden der WebUntis API.
    \item[27.09.2021-03.10.2021] Planung aller notwendigen Arbeitsschritte, um \ZELIA\ zu realisieren. Entwurf und Implementierung der Server API-Architektur. Erstellung der Logging Middleware am Server.
    \item[04.10.2021-10.10.2021] Erstellung der 404 Fehler Komponente und Seite im Frontend. Unterstützung bei Verbindung zwischen API-Server und Datenbank. Entwicklung eines WebUntis-Wrapper um die WebUntis API in TypeScript vereinfacht verwenden zu können.
    \item[11.10.2021-17.10.2021] Entwicklung der Komponenten, um Rauminformationen und Links darzustellen. Implementierung der API-Abfragen für Raumnummern und Stundenpläne vom Client. Überarbeitung des Frontend-Designs. CORS Middleware erstellt und Funktion zum Abfragen der Stundenpläne am Server implementiert. Testwerte gesetzt, um Abfragen testen zu können. Planung für Alpha\-version. Beginn der Dokumentation. 
    \item[18.10.2021-24.10.2021] Komponente, um Stundenplan anzuzeigen. Unter\-stützung \linebreak bei Erstellung des Datenbank\-schemas. Fehler\-behandlung bei Fehlern von Web\-Untis API. Implementierung eines Web\-Untis-Cache. Endpunkt erstellt, um alle Raumnamen und Nummern abzufragen. Fertig\-stellung und Nach\-besprechung der Alpha\-version.
    \item[25.10.2021-31.10.2021] Prüfung, ob Raum richtig eingegeben wurde, implementiert. Pathvariablenfehler behoben und Überarbeitung des Designs im Frontend (mit Julian Kusternigg). Stundenplankomponente fertiggestellt. Gemeinsames Planen für die Betaversion. 
    \item[01.11.2021-07.11.2021] Implementierung eines optimalen "ShadowRoot" der Komponenten. Erste Verwendung von Docker. Recherchieren über SSL Zertifikate und HTTP/S Proxys mit Nginx und Einführung von flexiblem CSS im Frontend (mit Julian Kusternigg).
    \item[08.11.2021-14.11.2021] Meeting mit Vorgesetzten. Implementierung der Debugkomponente, um auf Mobilgeräten den Logger anzuzeigen. Überarbeitung des RegEx Musters für das OCR Ergebnis (beides mit Julian Kusternigg). Recherche über Docker.
    \item[15.11.2021-21.11.2021] Überprüfung der Fehlerbehandlung am API-Server. Meeting für Betaversion und Nachbesprechung mit den Vorgesetzten. Unterstützung bei Entwicklung der Raummeldungskomponente. Überarbeitung der Docker Konfiguration.
    \item[22.11.2021-28.11.2021] Fertigstellung der Komponente, um Räume zu melden (mit Julian Kusternigg). Implementierung der Buchungskomponente. Überprüfung des Fehlerabfang-Experiments in der Middleware.
    Erstellung der OpenAPI-Spezi\-fikation. Weiterarbeiten an der Dokumentation.
    \item[29.11.2021-05.12.2021] Weitere Planungen für Betaversion
    \item[06.12.2021-12.12.2021] Admin Login implementiert (mit Julian Kusternigg). Docker Konfiguration optimiert. Betaversion Troubleshooting-Meeting
    \item[13.12.2021-19.12.2021] Filter für Admin-Dashboard eingebaut (mit Julian Kusternigg). Überarbeitung der "Booking-Handle"-Komponente. DB-Docker Fehler gesucht und behoben (mit Julian Kusternigg). Präsentierung des neuen Stands.
    \item[27.12.2021-02.01.2022] Dokumentationsarbeit 
    \item[17.01.2022-23.01.2022] Dokumentationsarbeit
    \item[24.01.2022-30.01.2022] Dokumentationsarbeit. Umstieg von Word auf LaTeX und Einarbeitung des Feedbacks von Vorgesetzten (mit Julian Kusternigg). 
    \item[31.01.2022-06.02.2022] Dokumentationsarbeit
    \item[07.02.2022-13.02.2022] Dokumentation überarbeiten
    \item[14.02.2022-20.02.2022] Dokumentationsarbeit 
\end{description}