\hthree{Mersed Kečo}
\sectionauthor{Mersed Kečo}

\begin{description}
    \item[20.09.2021-26.09.2021] Kennenlernen von "Node.js" und das Verständnis der WebUntis API. Des Weiteren die ersten Schritte mithilfe von GitHub. Erste Inbetriebnahme der WebUntis API. 
    \item[27.09.2021-03.10.2021] Modellierung des Datenbankschemas.Als Gruppe: Erstellen von Arbeitspaketen um einen Überblick zu behalten.
    \item[04.10.2021-10.10.2021] Verwaltung von Arbeitspaketen.Verwendung des WebUntis Wrapper's und alle dafür notwendigen Grundlagen. Umschreiben des Code von "Node.js" in TypeScript.
    \item[11.10.2021-17.10.2021] Helfen bei dem Designen des Frontends. Designen der \ZELIA\ API. Planen der zu erreichenden Ziele und Arbeitspakete für die Alphaversion. Meeting mit den Diplomarbeitsbetreuung um die Ziele der Alphaversion festzuhalten. Code für die Beschaffung der Rauminformationen über WebUntis API schreiben. 
    \item[18.10.2021-24.10.2021] Implementieren des GetRoomInfo Controller. Error-Hand\-ling und Zurückliefern eines HTTP-Statuscodes beim Abfangen eines Fehlers von WebUntis. FInale Abgaben für die Alpha-Version. DA-Meeting mit Betreuern bezüglich Alpha-Version. Beginn der Dokumentation.
    \item[25.10.2021-31.10.2021] Kickoff der Planung für die Beta-Version und das Zusammenschreiben von Arbeitspaketen. Entfernen der Präfixe der Klassenräume. Weiterschreiben an der Dokumentation.
    \item[01.11.2021-07.11.2021] Den Ablauf der WebUntis-Session behandeln und den automatischen Re-Login einbauen. Diplomarbeitsdokumentation erweitern. 
    \item[08.11.2021-14.11.2021] Meeting mit Diplomarbeitsbetreuern und Feedback zum momentanen Stand.
    \item[15.11.2021-21.11.2021] Diplomarbeitsdokumentation erweitern, Error-Handling der Middleware
    \item[22.11.2021-28.11.2021] Meeting für die Beta-Version und Feedback der Diplomarbeitsbetreuer. Errorhandling für die \ZELIA-API einbauen. Die grundlegenden Fehler der Middleware verstehen und die versuchen, abzufangen. Implementierung des Room Booking Controllers. 
    \item[29.11.2021-05.12.2021] Implementierung von wichtigen WebUntis Interfaces. Planen der noch zusätzlich erforderlichen Änderungen für die Beta-Version. Implementierung des Room-Information-Services. Beheben von Importing-Fehlern. Weiterschreiben an der Dokumentation.
    \item[06.12.2021-12.12.2021] Implementierung der AuthenticationMiddleware und des AdminRequestController für die \ZELIA-API. Meeting für das Troubleshooting der Fehler in der Betaversion von \ZELIA. Dokumentation weiterschreiben. 
    \item[13.12.2021-19.12.2021] Verbinden des AdminRequestController zur Datenbank. Den auslaufend WebUntis Token abfangen und erneuern. Präsentation der Veränderungen der Betaversion im Vergleich zum Meeting am 22.11.2021.
    \item[20.12.2021-26.12.2021] Diplomarbeitsdokumentation weiterschreiben.
    \item[10.01.2022-16.01.2022] Coronabedingt nicht anwesend/Diplomarbeitsdokumentation weiterschreiben.
    \item[17.01.2022-23.01.2022] Coronabedingt nicht anwesend/Diplomarbeitsdokumentation weiterschreiben.
    \item[24.01.2022-30.01.2022] Coronabedingt nicht anwesend/Diplomarbeitsdokumentation weiterschreiben. Installieren von LaTeX und umschreiben der Dokumentation in LaTeX.
    \item[07.02.2022-13.02.2022] Erweiterung der Dokumentation
    \item[14.02.2022-20.02.2022] Erweiterung und Verbesserung der Dokumentation, Korrekturlesen und Themenneufindung. Neuinstallation von LaTex. Designen des Werbebanners für \ZELIA. Projekttagebuch schreiben
    \item[21.02.2022-27.02.2022] Erweiterung der Dokumentation, Einholen von Angeboten für den Druck der Diplomarbeitsdokumentation
    \item[28.02.2022-06.03.2022] Erweiterung der Dokumentation, Korrekturlesen von bestim\-mten Kapiteln(Kapitel von Richard Panzer und Mario Naunović), Ausbessern von Fehlern in der Dokumentation
    \item[07.03.2022-13.03.2022] Erweiterung de Dokumentation, Korrekturlesen und Ausbessern der eigenen Dokumentation
    \item[14.03.2022-20.03.2022] Hinzufügen der Zeiterfassung in LaTeX, Einfügen von Kapiteln von Hr. Naunović in LaTeX
    \item[21.03.2022-27.03.2022] Endversion der Dokumentation, weitere Ausbesserung von Fehlern, Abgabe der Dokumentation in der Druckerei
    \item[28.03.2022-01.04.2022] Brennen des Repositories auf die CD, Abgabe des Diplomarbeitsbuches
\end{description}