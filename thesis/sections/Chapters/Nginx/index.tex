\htwo{Nginx}
\sectionauthor{Johannes Polzer}

\hthree{Allgemeines}

NGINX ist ein Webserver, "Reverse"-Proxy und E-Mail-Proxy, der von Igor Sysoev entwickelt und unter der BSD-Lizenz veröffentlicht wurde. 
Er kann auch für "Load Balancing" und Media Streaming verwendet werden.
\cite{WikiNginx}
\cite{nginx}

\hthree{Anwendung}
NGINX wird in der Datei "nginx.conf" konfiguriert, die sich im Verzeichnis \linebreak \mbox{\ttfamily "/etc/nginx/"} befinden muss. 

\css[code:nginxConf]{code/Nginx/example.conf}{Beispiel einer "nginx.conf" Datei}

Im Folgenden wird der Server-Teil des Codebeispiels \ref{code:nginxConf} im Abschnitt "{\ttfamily http}" genauer betrachtet: \\
Dort wird definiert, auf welchem Port der Server Verbindungen erwartet und ob er SSL verwenden soll oder nicht. 
Wenn NGINX für die Verwendung von SSL konfiguriert ist, müssen das SSL-Zertifikat und der Zertifikatsschlüssel dort konfiguriert sein.

Innerhalb der "Locations" wird definiert, wie NGINX die Anfragen verarbeiten soll. In diesem Beispiel werden alle Anfragen innerhalb des "{\ttfamily /api}"-Pfads an den API-Server, welcher unter \mbox{"{\ttfamily http://api:3001}"} läuft, weitergeleitet.

Für alle anderen Anfragen werden die Dateien aus dem Verzeichnis "{\ttfamily /www}" bereitgestellt. 

Da das Frontend von \ZELIA\ als eine SPA ("Single Page Application") mit dynamischen Komponenten entwickelt wurde, ist die einzige Datei, die für jeden Pfad ausgeliefert werden muss, die Datei "{\ttfamily index.html}". 
Dieses Verhalten wird mit dem Schlüsselwort "{\ttfamily try\_files}" konfiguriert. 
In diesem Fall wird versucht, die vom Client angeforderte Datei zu erhalten. 
Wenn diese nicht existiert, wird die "{\ttfamily index.html}"-Datei ausgeliefert. 
Daher können Dateien wie CSS- oder JS-Dateien wie üblich abgerufen werden.

