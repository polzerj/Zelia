\htwo{Authentifizierung}
\sectionauthor{Johannes Polzer}

\hthree{Allgemeines}

Bei der Webentwicklung werden Daten meist am Server gespeichert, da Daten, die im Browser gespeichert werden, vom Benutzer beim Bereinigen der Browserdaten gelöscht werden können. 
Dabei werden auch benutzerspezifische Daten am Server gespeichert, wodurch Authentifizierung ein wichtiges Sicherheits-Thema ist.
Damit diese Daten nicht öffentlich eingesehen werden können, muss sich der Benutzer beim Server authentifizieren.

Es gibt verschiedene \textbf{Authentifizierungsmethoden}. Zwei davon sind:
\begin{itemize} 
    \item \textbf{Username / Passwort} Ein Benutzer kann sich mit seinem Benutzernamen und Passwort anmelden.
    \item \textbf{OAuth2} Durch das OAuth2-Protokoll kann ein Benutzer sich mit einem externen Server authentifizieren. \cite{OAuth2}
\end{itemize}

Damit der Server nach der Anmeldung die Session immer dem Benutzer zuordnen kann gibt es mehre Möglichkeiten:
Einerseits gibt es die "Session-Cookie"-Methode. Auf der anderen Seite kann man auch ein sogenanntes "JSON Web Token" (JWT) nutzen.

Bei der Anmeldung auf einer Webseite werden meistens folgende Schritte durchgeführt:
\begin{enumerate}
    \item Der Benutzer gibt seinen Benutzernamen und sein Passwort ein.
    \item Benutzername und Kennwort werden durch HTTPS verschlüsselt an den Server gesendet.
    \item Server sendet ein "Session-Cookie" (Abschnitt \ref{sec:sessioncookie}) oder ein JWT (Abschnitt \ref{sec:jwt}) zurück an den Client.
\end{enumerate}

\pagebreak
\hfour{Session Cookie}\label{sec:sessioncookie}

Die ursprüngliche Idee der "session"-basierten Authentifizierung bestand darin, ein so genanntes "Session-Cookie" auf dem Web-Client zu speichern, das bei jeder Anfrage mitgeschickt wird. Mit diesem Cookie als Schlüssel kann der Server auf die Sitzung mit den Benutzerdaten zugreifen. 

Ein Hauptproblem, welches bei "Session-Cookie" basierenden Ansätzen entsteht, ist das "Load-Balancing". 
Da die Sitzungsdaten am Server gespeichert werden, müssen diese zwischen den verschiedenen Servern synchronisiert werden, damit der Benutzer jedes Mal dieselbe Antwort erhält, egal welcher Server antwortet. 

Ein zweiter Ansatz, um "Load-Balancing" zu ermöglichen, ist, dass der "Load-Balancer" speichert, zu welchem Server die Sitzung gehört. Dadurch kann es passieren, dass beim Ausfall des Servers die Sitzung abbricht. Außerdem ist es somit schwieriger, die Lasten gleichmäßig auf alle Server zu verteilen.

\hfour{JSON Web Token (JWT)}\label{sec:jwt}

Der JWT speichert die Benutzerdaten verschlüsselt im Token selbst. Dieser Token wird wie ein Sitzungs-"Cookie" bei jeder Anfrage mitgeschickt. 
Da die Sitzungsdaten jedoch im Token gespeichert sind, müssen die Server die Sitzungen nicht synchronisieren. 
Es ist außerdem möglich, einen anderen Server nur für die Authentifizierung zu verwenden. 
Die einzige Voraussetzung ist, dass sowohl der Authentifizierungsserver als auch der eigentliche Webserver das gleiche JWT-"Secret" verwenden. 
Um sicherzustellen, dass der Token nicht manipuliert wird, haben nur die Server den Schlüssel, um den Token zu signieren und die darin enthaltenen Daten zu manipulieren. \cite{Auth0JWT}

Das JWT ist im RFC7519 definiert. Jedes JWT ist in drei Abschnitte\cite{Auth0JWT} unterteilt. 

\begin{itemize}
    \item \textbf{"Header"} -- Informationen über den Token
    \item \textbf{"Payload"} -- Informationen über den Benutzer
    \item \textbf{"Signature"} -- Signatur des Tokens
\end{itemize}

Im Header befinden sich meist zwei Felder - der Typ des Tokens ({\ttfamily typ}) und der\linebreak"Signing"-Algorithmus ({\ttfamily alg}). 
Diese standardisierten Felder sind in der Regel drei Zeichen lang, um den Token möglichst kompakt zu halten. \cite{Auth0JWT}

Die "Payload" besteht in der Regel aus Benutzerdaten. Diese nennt man "Claims". Dabei unterscheidet man drei Arten. \cite{Auth0JWT}: 

\begin{itemize}
    \item "Registered claims" sind vordefinierte "Claims", wie
    \begin{itemize}
        \item "iss" -- is user
        \item "exp" --  expiration time
        \item ...
    \end{itemize}
    \item "Public claims" können vom Entwickler definiert werden.
    \item "Private claims" werden verwendet, um Informationen zwischen Servern aus\-zu\-tauschen.
\end{itemize}

Sowohl Header als auch "Payload" werden mit Base64 kodiert.

Die "Signature" wird durch das Verschlüsseln der Base64 Url kodierten Abschnitte "Header", "Payload" und einem "Secret" erstellt. 
Dabei wird der im Header spezifizierte Verschlüsselungsalgorithmus verwendet. \cite{RFC7519} \cite{WdsJWT}

Die Abschnitte des JWT werden durch einen Punkt getrennt zusammengeführt. 

{\ttfamily Header.Payload.Signature}

Wenn geschützte Ressourcen angefordert werden, wird das JWT in der Regel im Authorization-Header als sogenannter Bearer Token gesendet. Dieser Header würde wie folgt aussehen\cite{Auth0JWT}:

{\ttfamily Authorization: Bearer \textless JWT Token\textgreater}




\clearpage


\hfour{Speicherung des JWT}

Um den JSON Web Token sicher zu speichern, ist es wichtig, dass dieser nicht mit unbegrenzter Gültigkeit gespeichert wird. Darüber hinaus soll der Administrator beim Schließen des "Admin Dashboards" ausgeloggt werden. Deshalb wird dieser Token im Session-Storage abgelegt. Dadurch wird der Token nach der Sitzung gelöscht und der Administrator muss sich beim nächsten Besuch des "Admin Dashboards" neu anmelden.

\hfour{Cache für schnelle Ladezeiten}

Damit die Anwendung alle Voraussetzungen erfüllt um als PWA installiert zu werden, muss die Webseite im Cache gespeichert werden. Somit kann die Webseite bzw. die installierte PWA offline genauso geöffnet werden. Allerdings wird bei der offline-Nutzung eine Meldung angezeigt, dass keine Rauminformationen verfügbar sind, da die API-Anfragen eine Aktive Internetverbindung benötigen.

Ein Beispiel für die Implementierung des Cache (Code \ref{code:cache}) befindet sich im Kapitel \ref{sec:cacheImpl} auf Seite \pageref{code:cache}.
 