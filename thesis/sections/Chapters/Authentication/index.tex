\htwo{Authentifizierung}
\sectionauthor{Johannes Polzer}

\hthree{Allgemeines}

Authentifizierung ist ein sehr wichtiges Thema, wenn es um Webentwicklung geht, da es keine Möglichkeit gibt, wichtige Benutzerdaten auf dem Client-Rechner zu speichern. Daher ist es notwendig, diese Daten serverseitig zu speichern. Um diese Daten geheim zu halten, muss sich der Webclient beim Server authentifizieren. Es gibt zwei häufig verwendete Arten der Authentifizierung. Die erste ist die "Session-Cookie"-Methode und die zweite ist das sogenannte "JSON Web Token" (JWT). Das Anmeldeverfahren ist bei beiden sehr ähnlich. 

\begin{enumerate}
    \item Der Benutzer gibt seinen Benutzernamen und sein Passwort ein
    \item Benutzername und Kennwort werden an den Server gesendet
    \item Server sendet ein "Sitzungs-Cookie" oder einen JWT zurück an den Client
\end{enumerate}

\hfour{Session Cookie}

Die ursprüngliche Idee der sitzungs-basierten Authentifizierung bestand darin, ein so genanntes Sitzungs-"Cookie" auf dem Web-Client zu speichern, das bei jeder Anfrage mitgeschickt wird. Mit diesem Cookie als Schlüssel kann der Server auf die Sitzung mit den Benutzerdaten zugreifen. 

Ein Hauptproblem der auf Sitzungs-"Cookies" basierenden Ansätze besteht darin, dass der "Load-Balancing"-Prozess aufgrund der Sitzungen, die zwischen den verschiedenen Servern synchronisiert werden müssen, um sicherzustellen, dass der Benutzer dieselbe Antwort erhält, egal welcher Server antwortet, recht schwierig zu handhaben ist.

\hfour{JSON Web Token (JWT)}\label{sec:jwt}

Der JWT speichert die Benutzerdaten verschlüsselt im Token selbst. Dieser Token wird wie ein Sitzungs-"Cookie" bei jeder Anfrage mitgeschickt. 
Da die Sitzungsdaten jedoch im Token gespeichert sind, müssen die Server die Sitzungen nicht synchronisieren. 
Es ist außerdem möglich, einen anderen Server nur für die Authentifizierung zu verwenden. 
Die einzige Voraussetzung ist, dass sowohl der Authentifizierungsserver als auch der eigentliche Webserver das gleiche JWT-"Secret" verwenden. 
Um sicherzustellen, dass der Token nicht manipuliert wird, haben nur die Server den Schlüssel, um den Token zu signieren und die darin enthaltenen Daten zu manipulieren. \cite{Auth0JWT}

Das JWT ist im RFC7519 definiert. Jedes JWT ist in drei Abschnitte\cite{Auth0JWT} unterteilt. 

\begin{itemize}
    \item \textbf{"Header"} -- Informationen über den Token
    \item \textbf{"Payload"} -- Informationen über den Benutzer
    \item \textbf{"Signature"} -- Signatur des Tokens
\end{itemize}

Im Header befinden sich meist zwei Felder - der Typ des Tokens ({\ttfamily typ}) und der\linebreak"Signing"-Algorithmus ({\ttfamily alg}). 
Diese standardisierten Felder sind in der Regel drei Zeichen lang, um den Token möglichst kompakt zu halten. \cite{Auth0JWT}

Die "Payload" besteht in der Regel aus Benutzerdaten. Diese nennt man "Claims". Dabei unterscheidet man drei Arten. \cite{Auth0JWT}: 

\begin{itemize}
    \item "Registered claims" sind vordefinierte "Claims", wie
    \begin{itemize}
        \item "iss" -- is user
        \item "exp" --  expiration time
        \item ...
    \end{itemize}
    \item "Public claims" können vom Entwickler definiert werden.
    \item "Private claims" werden verwendet, um Informationen zwischen Servern aus\-zu\-tauschen.
\end{itemize}

Sowohl Header als auch "Payload" werden mit Base64 kodiert.

Die "Signature" wird durch das Verschlüsseln der Base64 Url kodierten Abschnitte "Header", "Payload" und einem "Secret" erstellt. 
Dabei wird der im Header spezifizierte Verschlüsselungsalgorithmus verwendet. \cite{RFC7519} \cite{WdsJWT}

Die Abschnitte des JWT werden durch einen Punkt getrennt zusammengeführt. 

{\ttfamily Header.Payload.Signature}

Wenn geschützte Ressourcen angefordert werden, wird das JWT in der Regel im Authorization-Header als sogenannter Bearer Token gesendet. Dieser Header würde wie folgt aussehen\cite{Auth0JWT}:

{\ttfamily Authorization: Bearer \textless JWT Token\textgreater}




\pagebreak


\hfour{Speicherung des JWT}

Um den JSON Web Token sicher zu speichern, ist es wichtig, dass dieser nicht mit unbegrenzter Gültigkeit gespeichert wird. Darüber hinaus soll der Administrator beim Schließen des "Admin Dashboards" ausgeloggt werden. Deshalb wird dieser Token im Session-Storage abgelegt. Dadurch wird der Token nach der Sitzung gelöscht und der Administrator muss sich beim nächsten Besuch des "Admin Dashboards" neu anmelden.

\hfour{Cache für schnelle Ladezeiten}

Damit die Anwendung alle Voraussetzungen erfüllt um als PWA installiert zu werden, muss die Webseite im Cache gespeichert werden. Somit kann die Webseite bzw. die installierte PWA offline genauso geöffnet werden. Allerdings wird bei der offline-Nutzung eine Meldung angezeigt, dass keine Rauminformationen verfügbar sind, da die API-Anfragen eine Aktive Internetverbindung benötigen.

Ein Beispiel für die Implementierung des Cache (Code \ref{code:cache}) befindet sich im Kapitel \ref{sec:cacheImpl} auf Seite \pageref{code:cache}.
 