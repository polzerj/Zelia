\hthree{Json Web Token (JWT)}\label{sec:jwt}

Der JWT speichert die Benutzerdaten verschlüsselt im Token selbst. Dieses Token wird wie ein Sitzungs-"Cookie" bei jeder Anfrage mitgeschickt. Da die Sitzungsdaten jedoch im Token gespeichert sind, müssen die Server die Sitzungen nicht synchronisieren. Es ist außerdem möglich, einen eigenen Server nur für die Authentifizierung zu verwenden. 
Die einzige Voraussetzung ist, dass sowohl der Authentifizierungsserver als auch der eigentliche Webserver das gleiche JWT-"Secret" verwenden. Um sicherzustellen, dass das Token nicht manipuliert wird, haben nur die Server den Schlüssel, um das Token zu signieren und die darin enthaltenen Daten zu manipulieren. 

Das JWT ist im RFC7519 definiert. Jedes JWT ist in drei Abschnitte unterteilt. 

\begin{itemize}
    \item "Header"
    \item "Payload"
    \item "Signature"
\end{itemize}

Im Header befinden sich meist zwei Felder. 
Der Typ des Tokens (typ) und der\linebreak"Signing"-Algorithmus (alg). 
Diese standardisierten Felder sind in der Regel drei Zeichen lang, um das "Token" möglichst kompakt zu halten.

Die "Payload" besteht in der Regel aus Benutzerdaten. Diese nennt man "Claims". Es gibtes drei Arten von "Claims": 

\begin{itemize}
    \item "Registered claims" sind vordefinierte "Claims" wie
    \begin{itemize}
        \item "iss": is user
        \item "exp":  expiration time
        \item …
    \end{itemize}
    \item "Public claims" können vom Entwickler definiert werden
    \item "Private claims" werden verwendet, um Informationen zwischen Servern auszutauschen
\end{itemize}

Sowohl Header als auch "Payload" werden mit Base64 kodiert.

Die "Signature" wird erstellt durch Verschlüsseln der Base64Url kodierten Abschnitte "Header" und "Payload" und einem "Secret". 
Dabei wird der im Header spezifizierten Verschlüsselungsalgorithmus verwendet.

Die Abschnitte des JWT werden durch einen Punkt getrennt zusammengeführt. 

{\ttfamily Header.Payload.Signature}

Wenn geschützte Ressourcen angefordert werden, wird das JWT in der Regel im Authorization-Header als sogenanntes Bearer Token gesendet. Dieser Header würde wie folgt aussehen:

\emph{Authorization: Bearer \textless JWT Token\textgreater}

\cite{Auth0JWT}
\cite{RFC7519}
\cite{WdsJWT}