\hthree{Controller}\label{sec:controller}

Ein Controller behandelt alle "Requests" auf einem Pfad. 
Controller sind Klassen, welche von der "ControllerBase" erben. Im Super-Konstruktor des Controllers muss der Pfad der Requests angegeben werden, welche vom Controller verarbeitet werden sollen. Zusätzlich kann ein Objekt mit zusätzlichen Middlewares übergeben werden. Um eine HTTP-Methode zu implementieren, müssen lediglich die Methoden "get", "post", "put", "patch" oder "delete" überschrieben werden. Diese nehmen die von "ExpressJS" bekannten Parameter "Request" und "Response". 

Im folgenden ist ein Beispiel eines Controllers, welcher auf dem Pfad "/hello" registriert wird und beim Aufruf der Methode "get" ein Objekt zurückgibt, welches in der Eigenschaft "hello" den Text "world" beinhaltet:

\typescript{code/APITemplate/helloWorldController.ts}{Beispiel-Controller}

Wie ein Controller registriert wird ist im Abschnitt "App" \ref{sec:app} beschrieben. 