\hthree{Controller}\label{sec:controller}

Ein "Controller" verarbeitet alle "Requests" auf einem Pfad. 
"Controller" sind Klassen, welche von der "ControllerBase"-Klasse erben. 
Im Basis-Konstruktor (in Javascript "{\ttfamily super()}") des "Controllers" muss der Pfad der Requests an die Basisklasse, "ControllerBase", übergeben werden, welche vom "Controller" verarbeitet werden sollen. 
Zusätzlich kann ein Objekt mit "Middlewares" (siehe Kapitel "Middleware" \ref{sec:middleware}, Seite \pageref{sec:middleware}) übergeben werden, welche nur vor diesem "Controller" aufgerufen werden. 
Um eine HTTP-Methode zu implementieren, müssen lediglich folgende Methoden in der Klasse überschrieben werden:

\begin{itemize}
    \item {\ttfamily get} -- wird aufgerufen, wenn ein GET-Request auf den Pfad des "Controllers" gestellt wird.
    \item {\ttfamily post} -- wird aufgerufen, wenn ein POST-Request auf den Pfad des "Controllers" gestellt wird.
    \item {\ttfamily put} -- wird aufgerufen, wenn ein PUT-Request auf den Pfad des "Controllers" gestellt wird.
    \item {\ttfamily patch} -- wird aufgerufen, wenn ein PATCH-Request auf den Pfad des "Controllers" gestellt wird.
    \item {\ttfamily delete} -- wird aufgerufen, wenn ein DELETE-Request auf den Pfad des "Controllers" gestellt wird.
\end{itemize}

Diese Methode nehmen die von "ExpressJS" bekannten Parameter "Request" und "Response". 
Im "Request"-Objekt (oft mit "req" abgekürzt) kann auf alle Informationen des Requests zugegriffen werden. Dazu zählen \zb\ die Header, Query-Parameter oder der Body.
Das "Response"-Objekt (oft mit "res" abgekürzt) ermöglicht es, eine Antwort zu erstellen. Dafür können mit diesem Objekt der Statuscode, die Header und der Body gesetzt werden.


Im Folgenden ist ein Beispiel eines "Controllers", welcher auf dem Pfad "/hello" registriert wird und beim Aufruf der Methode "get" ein JSON-Objekt zurückgibt, welches in der Eigenschaft "hello" den Text "world" beinhaltet:

\typescript{code/APITemplate/helloWorldController.ts}{Beispiel-Controller}

Wie ein "Controller" registriert wird, ist im Abschnitt "App" \ref{sec:app} beschrieben. 