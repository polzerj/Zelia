\htwo{API-Architektur}
\sectionauthor{Johannes Polzer}

Um den Servercode besser zu strukturieren, wurde für \ZELIA\ ein spezieller Wrapper um das Modul "ExpressJs" (siehe Kapitel \ref{sec:express} auf Seite \pageref{sec:express}) entwickelt. 
Diese Grundstruktur erlaubt es, alle Endpunkte in einzelne sogenannte "Controller"-Klassen zu implementieren. 
Die Architektur der API besteht aus einigen Verzeichnissen, welche in der Abbildung \ref{fig:apiStructure} dargestellt werden. 
Um diesen Wrapper zu nutzen, muss eine Instanz der "App"-Klasse erstellt werden.

\begin{figure}[H]
    \centering
    \includegraphics{media/APITemplate/ProjectStructure.png}
    \caption{Projektstruktur}
    \label{fig:apiStructure}
\end{figure}

\clearpage
\htwo{"App"-Klasse}

Im Konstruktor der App-Klasse können verschiedene Optionen mitgegeben werden:

\hfour{"Port"}

Durch die Port-Option kann angegeben werden, auf welchem Port die API HTTP-Requests annehmen soll. Standardmäßig ist dies auf die Umgebungsvariable "PORT" gesetzt. Falls diese nicht gesetzt ist, wird der Port 3001 verwendet.

\hfour{"Middlewares"}

Dies ist ein Array an Middlewares, welche bei jedem Request ausgeführt werden. Siehe Unterpunkt 3.8.3 Middleware

\hfour{"Controllers"}

Die Controller werden als Array von Controller Objekten übergeben und repräsentieren einen Endpoint. Siehe Unterpunkt 3.8.2 Controller

\hfour{"Base-URL"}

Durch diesen String kann der Root-Pfad definiert werden. Standardmäßig ist diese auf "/" gesetzt. Bei Zelia ist diese auf "/api" konfiguriert worden, da alle API bezogenen Anfragen auf den Pfad "/api" gehen.

Die "listen"-Methode muss aufgerufen werden, um die "App" zu starten.
Verwendung der "App"
Um eine Instanz der App-Klasse zu erstellen, muss diese importiert werden. Danach kann die "App" mit ihren registrierten "Controllern" und "Middlewares" instanziiert werden. Damit die "App” auf dem angegebenen Port gestartet wird, wir die listen Methode aufgerufen.

\typescript{code/APITemplate/appusage.ts}{Verwendung der App-Klasse}
\hthree{Types}

Das "Types"-Verzeichnis enthält die Interfaces und abstrakten Klassen, wie die "ControllerBase"-Klasse die benötigt werden, um diesen ExpressJs Wrapper zu nutzen.

\hthree{Die "ControllerBase"-Klasse}

Die abstrakte "ControllerBase"-Klasse ist die Basisklasse eines jeden "Controllers" und stellt die grundlegenden Funktionen bereit. 

Der Konstruktor der "ControllerBase" nimmt zwei Parameter entgegen:

\begin{description}
    \item["path"] Der Pfad, unter dem der "Controller" registriert werden soll.
    \item["middlewareParam"] Ein Objekt, welches das Interface "MiddlewareParam" implementiert und dabei die folgenden Parameter enthält:
    \begin{description}
        \item["all"] Wird bei allen Requests aufgerufen
        \item["get"] Wird bei GET-Requests aufgerufen
        \item["post"] Wird bei POST-Requests aufgerufen
        \item["put"] Wird bei PUT-Requests aufgerufen
        \item["patch"] Wird bei PATCH-Requests aufgerufen
        \item["delete"] Wird bei DELETE-Requests aufgerufen
    \end{description}
\end{description}

\typescript{code/APITemplate/MiddlewareParam.ts}{Interface "MiddlewareParam"}

"Middlewares" sind im Kapitel \ref{sec:middleware} beschrieben.

Um die HTTP-Methoden zu implementieren, gibt es in der "ControllerBase"-Klasse Methoden, welche überschrieben werden können. 
Wenn eine Methode nicht implementiert wird, wird der HTTP Status-Code 405 (Method Not Allowed) zurückgegeben.

Die Namen der Methoden entsprechen deren HTTP-Methoden:
\begin{itemize}
    \item get
    \item post
    \item put
    \item path
    \item delete
\end{itemize}

Der Zusammenhang zwischen der "App"-Klasse und der "ControllerBase"-Klasse ist in der Abbildung \ref{fig:apiUML} dargestellt.

\typescript{code/APITemplate/ControllerBase.ts}{"ControllerBase"-Klasse}
\hthree{Controller}\label{sec:controller}

Ein "Controller" verarbeitet alle "Requests" auf einem Pfad. 
"Controller" sind Klassen, welche von der "ControllerBase"-Klasse erben. 
Im Super-Konstruktor des "Controllers" muss der Pfad der Requests angegeben werden, welche vom "Controller" verarbeitet werden sollen. 
Zusätzlich kann ein Objekt mit Middlewares übergeben werden, welche nur vor diesem Controller aufgerufen werden. 
Um eine HTTP-Methode zu implementieren, müssen lediglich die Methoden "get", "post", "put", "patch" oder "delete" überschrieben werden. 
Diese nehmen die von "ExpressJS" bekannten Parameter "Request" und "Response". 

Im folgenden ist ein Beispiel eines "Controllers", welcher auf dem Pfad "/hello" registriert wird und beim Aufruf der Methode "get" ein Objekt zurückgibt, welches in der Eigenschaft "hello" den Text "world" beinhaltet:

\typescript{code/APITemplate/helloWorldController.ts}{Beispiel-Controller}

Wie ein Controller registriert wird ist im Abschnitt "App" \ref{sec:app} beschrieben. 
\hthree{"Middleware"}\label{sec:middleware}

Bei \ZELIA\ ist eine "Middleware" eine Funktion, die eine bestimmte Signatur hat (siehe Code \ref{code:middlewareSignature}), welche von "express" vor der Handler-Methode des "Controllers" aufgerufen wird. 
Diese benötigt die aus dem NPM-Modul "express" bekannten Parameter:

\begin{itemize}
    \item \textbf{Request-Objekt} ({\ttfamily req}) -- beinhaltet alle Informationen des Requests. Dazu zählen \zb\ die Header, Query-Parameter oder der Body.
    \item \textbf{Response-Objekt} ({\ttfamily res}) -- ermöglicht das Erstellen einer Antwort. Dafür können mit diesem Objekt der Statuscode, die Header und der Body gesetzt werden.
    \item \textbf{Next-Funktion} ({\ttfamily next}) -- muss aufgerufen werden, wenn die Middleware fertig ist, ohne dass dieser die Anfrage durch eine Antwort beendet hat.
\end{itemize}

"Middlewares" können entweder für einzelne Methoden im "Controller", für einzelne "Controller" oder für die gesamte API registriert werden. Soll die Middleware für die gesamte API zur Verfügung stehen, muss diese im App-Konstruktor registriert werden.

\typescript[code:middlewareSignature]{code/APITemplate/middleware.ts}{Signatur einer Middleware}

\hfour{Anwendungsfälle}

Eine "Middleware" könnte zum Beispiel Informationen über die eingehenden "Requests" loggen, damit die Entwickler leichter debuggen können. 
Dabei wird sie im \linebreak App-Konstruktor registriert, um bei jeder Anfrage mitzuloggen. 

\begin{figure}[H]
    \centering
    \includegraphics{media/APITemplate/LogMiddleware.svg.pdf}
    \caption{Aufrufreihenfolge der "Middlewares" vor dem "Controller" anhand der "LoggerMiddleware"} 
\end{figure}

\typescript{code/APITemplate/loggerMiddleware.ts}{Beispiel der "LoggerMiddleware"}

"Middlewares" können auch benutzt werden, um den Benutzer beim "Request" zu \mbox{authentifizieren}.
Dabei wird in der "Middleware" geprüft, ob der Nutzer auf diesen Endpunkt zugreifen darf. Wenn ja, wird die "Next"-Funktion aufgerufen, um die nächste "Middleware" bzw. die Handler-Methode des "Controllers" aufzurufen. Wenn nicht, wird mittels des "Response"-Objektes eine "Unauthorised-Response" geschickt.

Wenn die "Logger"-"Middleware" in der App-Klasse und die "Authentication"-"Middleware" in einem "Controller" registriert sind, wird bei einer Anfrage auf diesen "Controller" zunächst die "Logger"-"Middleware" ausgeführt. Danach wird in der "Authentication"-"Middleware" geprüft, ob der Nutzer berechtigt ist, auf diesen Endpunkt zuzugreifen. Wenn ja, wird die "Next"-Funktion aufgerufen, um die nächste "Middleware" bzw. die Handler-Methode des "Controllers" aufzurufen. Wenn nicht, wird mittels des "Response"-Objektes eine "Unauthorised-Response" geschickt (siehe Abbildung \ref{fig:authMiddleware}).

\begin{figure}[H]
    \centering
    \includegraphics[width=15cm]{media/APITemplate/AuthMiddleware.svg.pdf}
    \caption{Aufrufreihenfolge der "Middlewares" vor dem "Controller" anhand der "LoggerMiddleware" und der "AuthenticationMiddleware" } 
    \label{fig:authMiddleware}
\end{figure}

\typescript{code/APITemplate/authenticationMiddleware.ts}{Beispiel der "Authentication"-"Middleware" }

\hthree{Models}

Im "Models"-Verzeichnis werden die Models der REST-API abgelegt, welche die Struktur des "Response-Body" der REST-API beschreiben.
\hthree{Services}

Im "Services"-Verzeichnis befindet die gesamte Geschäftslogik. Diese beinhaltet die Anbindung an die WebUntis-API und die Verifizierung der Buchungen und Meldungen.
\hthree{Data}

Im "Data"-Verzeichnis werden alle Skripte, die für die Datenbank relevant sind, gespeichert.
Unter anderem befinden sich dort die Datenbank-"Entities", welche jeweils eine Tabelle der Datenbank repräsentieren.
Außerdem sind dort die Scripte für die Verbindung zur Datenbank und die Mapper zur Konvertierung der verschiedenen Datentypen vorhanden. 
Weitere Informationen über die Datenbank befinden sich in den Kapiteln \ref{sec:MariaDB} MariaDB und \ref{sec:sequelize} Sequelize.

