\hthree{"Middleware"}\label{sec:middleware}

Eine "Middleware" ist eine Funktion, die eine bestimmte Signatur hat (siehe Code \ref{code:middlewareSignature}), welche vor dem "Controller" aufgerufen wird. 
Diese nimmt die aus "ExpressJS" bekannten Parameter "Request", "Response" und "Next". 
"Middlewares" können sowohl für die gesamte API im App-Konstruktor registriert werden als auch für einzelne "Controller", bzw. einzelne Methoden im "Controller". 
Wichtig ist, dass die "Next"-Funktion aufgerufen wird, wenn die "Middleware" fertig ist, da ansonsten die nachfolgenden "Middlewares" und der eigentliche "Handler" des Endpunktes nicht aufgerufen werden.

\typescript[code:middlewareSignature]{code/APITemplate/middleware.ts}{Signatur einer Middleware}

Eine "Middleware" könnte zum Beispiel Informationen über die "Requests" loggen. 

\begin{figure}[H]
    \centering
    \includegraphics{media/APITemplate/LogMiddleware.svg.pdf}
    \caption{Aufrufreihenfolge Beispiel Log-Middleware} 
\end{figure}

Allerdings können "Middlewares" auch benutzt werden, um Nutzer auf bestimme "Controller" zu Autorisieren. Dabei wird in der "Middleware" geprüft, ob der Nutzer auf diesen "Endpoint" zugreifen darf. Wenn ja, wird die "Next"-Funktion aufgerufen. Wenn nicht, wird eine "Unauthorised-Response" geschickt.

\begin{figure}[H]
    \centering
    \includegraphics[width=15cm]{media/APITemplate/AuthMiddleware.svg.pdf}
    \caption{Aufrufreihenfolge Beispiel Authentication-Middleware} 
\end{figure}