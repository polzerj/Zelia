\hthree{"Middleware"}\label{sec:middleware}

Eine "Middleware" ist eine Funktion, die eine bestimmte Signatur hat (siehe Code \ref{code:middlewareSignature}), welche von "express" vor der Handler-Methode des "Controllers" aufgerufen wird. 
Diese nimmt die aus dem NPM-Modul "express" bekannten Parametern:

    \begin{itemize}
        \item \textbf{Request-Objekt} ({\ttfamily req}) - beinhaltet alle Informationen des Requests. Dazu zählen \zb\ die Header, Query-Parameter oder der Body.
        \item \textbf{Response-Objekt} ({\ttfamily res}) - ermöglicht das Erstellen einer Antwort. Dafür können mit diesem Objekt der Statuscode, die Header und der Body gesetzt werden.
        \item \textbf{Next-Funktion} ({\ttfamily next}) - muss aufgerufen werden, wenn die Middleware fertig ist, ohne dass dieser die Anfrage durch eine Antwort beendet hat.
    \end{itemize}


"Middlewares" können sowohl für die gesamte API im App-Konstruktor registriert werden als auch für einzelne "Controller", bzw. einzelne Methoden im "Controller". 

\typescript[code:middlewareSignature]{code/APITemplate/middleware.ts}{Signatur einer Middleware}

\hfour{Anwendungsfälle}

Eine "Middleware" könnte zum Beispiel Informationen über die eingehenden "Requests" loggen, damit die Entwickler leichter debuggen können. Dabei wird sie im \linebreak App-Konstruktor registriert, um bei jeder Anfrage mitzuloggen. 

\begin{figure}[H]
    \centering
    \includegraphics{media/APITemplate/LogMiddleware.svg.pdf}
    \caption{Aufrufreihenfolge der "Middlewares" vor dem Controller anhand der "Logger"-"Middleware"} 
\end{figure}

\typescript{code/APITemplate/loggerMiddleware.ts}{Bispiel der "Logger"-"Middleware"}

"Middlewares" können auch benutzt werden, um den Benutzer beim "Request" zu Authentifizieren. Dabei wird in der "Middleware" geprüft, ob der Nutzer auf diesen Endpunk zugreifen darf. Wenn ja, wird die "Next"-Funktion aufgerufen, um die nächste "Middleware", bzw. die Handler-Methode des "Controllers" aufzurufen. Wenn nicht, wird mittels des "Response"-Objektes eine "Unauthorised-Response" geschickt.

Wenn die "Logger"-"Middleware" in der App-Klasse und die "Authentication"-"Middleware" in einem "Controller" registriert sind, wird bei einer Anfrage auf diesen Controller zunächst die "Logger"-"Middleware" ausgeführt. Danach wird in der "Authentication"-"Middleware" geprüft, ob der Nutzer berechtigt ist, auf diesen Endpunkt zuzugreifen. Wenn ja, wird die "Next"-Funktion aufgerufen, um die nächste "Middleware", bzw. die Handler-Methode des "Controllers" aufzurufen. Wenn nicht, wird mittels des "Response"-Objektes eine "Unauthorised-Response" geschickt (siehe Abbildung \ref{fig:authMiddleware}).

\begin{figure}[H]
    \centering
    \includegraphics[width=15cm]{media/APITemplate/AuthMiddleware.svg.pdf}
    \caption{Aufrufreihenfolge der "Middlewares" vor dem Controller anhand der "Logger"-"Middleware" und der "Authentication"-"Middleware" } 
    \label{fig:authMiddleware}
\end{figure}

\typescript{code/APITemplate/authenticationMiddleware.ts}{Bispiel der "Authentication"-"Middleware" }
