\hthree{Die "ControllerBase"-Klasse}

Die abstrakte "ControllerBase"-Klasse ist die Basisklasse eines jeden "Controllers" und stellt die grundlegenden Funktionen bereit. 

Der Konstruktor der "ControllerBase" nimmt zwei Parameter entgegen:

\begin{description}
    \item["path"] Der Pfad, unter dem der Controller registriert werden soll.
    \item["middlewareParam"] Ein Objekt, welches das Interface "MiddlewareParam" implementiert und dabei die folgenden Parameter enthält:
    \begin{description}
        \item["all"] Wird bei allen Requests aufgerufen
        \item["get"] Wird bei GET-Requests aufgerufen
        \item["post"] Wird bei POST-Requests aufgerufen
        \item["put"] Wird bei PUT-Requests aufgerufen
        \item["patch"] Wird bei PATCH-Requests aufgerufen
        \item["delete"] Wird bei DELETE-Requests aufgerufen
    \end{description}
\end{description}

\typescript{code/APITemplate/MiddlewareParam.ts}{Interface "MiddlewareParam"}

"Middlewares" sind im Kapitel \ref{sec:middleware} beschrieben.

Um eine HTTP-Methode zu implementieren, gibt es in der "ControllerBase" Methoden, welche überschrieben werden müssen. Wenn eine Methode nicht implementiert wird, wird der HTTP Status-Code 405 (Method Not Allowed) zurückgegeben.

Die Namen der Methoden entsprechen deren HTTP-Methoden:
\begin{itemize}
    \item get
    \item post
    \item put
    \item path
    \item delete
\end{itemize}


Der Zusammenhang zwischen der "App"-Klasse und der "ControllerBase"-Klasse ist in der Abbildung \ref{fig:apiUML} dargestellt.

\typescript{code/APITemplate/ControllerBase.ts}{"ControllerBase"-Klasse}