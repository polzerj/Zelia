\hthree{Linuxcontainer}

Da die moderne Computer- und Entwicklungswelt immer komplexer und anspruchsvoller wird, bietet die Containertechnologie mithilfe von Linux-Containern die Möglichkeit unterschiedliche Anwendungen samt ihren gesamten Daten quasi zu paketieren und vom Rest des Systems zu isolieren. So bleiben die einzelnen Programme und Anwendungen uneingeschränkt funktionsfähig und können in einzelne Umgebungen verschoben werden, wie zum Beispiel in die Entwicklung oder die Produktion. \cite{Container}

Die Entscheidung, auf die Entwicklung mit Containertechnologie zu setzten, basiert darauf, dass so Konflikte zwischen den Entwicklern und dem Operationsteam getrennt werden können, da die Zuständigkeiten dieser zwei Teams auch völlig unterschiedlich sind. So arbeiten Systemapps und Systeminfrastruktur voneinander getrennt in separaten Containern. Der wohl wichtigste Vorteil Linux-Container als Technologie zu wählen liegt darin, dass diese Technologie Open-Source basiert ist und so Änderungen und Neuerung sofort und kostenlos für alle zur Verfügung gestellt werden. \cite{Container}
