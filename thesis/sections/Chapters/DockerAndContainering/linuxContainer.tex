\hthree{Linuxcontainer}

Da die moderne Computer- und Entwicklungswelt immer komplexer und anspruchsvoller wird, bietet die Containertechnologie mithilfe von Linux-Containern die Möglichkeit unterschiedliche Anwendungen samt ihren gesamten Daten quasi zu paketieren und vom Rest des Systems zu isolieren.  So blieben die einzelnen Programme und Anwendungen uneingeschränkt funktionsfähig und können in einzelnen Umgebungen verschoben werden, wie die Entwicklung oder die Produktion.

Der entscheidende Grund sich für die Entwicklung mit Containertechnologie zu entscheiden, liegt darin, dass man so Konflikte zwischen den Entwicklern und Operationsteam trennen kann, da die Zuständigkeiten dieser Zwei Teams auch völlig unterschiedlich sind. So arbeiten Systemapps und Systeminfrastruktur voneinander getrennt und in separaten Containern. Der wohl wichtigste Vorteil Linux-Container als gewünscht Technologie zu wählen, liegt daran, dass diese Technologie open Source basiert ist und so Änderungen und Erneuerung sofort und kostenlos für alle zur Verfügung gestellt werden. \cite{Container}
