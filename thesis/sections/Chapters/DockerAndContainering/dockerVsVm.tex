\hthree{Docker vs Virtuelle Maschinien}

Die Technologie und der Sinn von Docker unterscheiden sich in vielen Punkten von herkömmlichen Virtuellen Maschinen. Jedoch kann man zusammenfassend und vereinfacht sagen, dass virtuelle Maschinen für die meisten Benutzer einmal erstellt und nachher nicht mehr geändert werden. Die VM muss höchstens aktualisiert, umkonfiguriert oder repariert werden. Jedoch bleibt sie immer ein sich geschlossenes System.

Oft wird davon gesprochen, dass die Linux-Containertechnologie und im speziellen Docker Virtuelle Maschinen überflüssig machen. Dieser Gedankengang ist jedoch falsch, da die beiden Technologien zwei unterschiedliche Anwendungsgebiete abdecken sich ergänzen, aber nicht ersetzen. Auf der einen Seite bieten VMs den Vorteil, dass sie Infrastrukturelastizität bieten, auf der anderen Seite hat Docker den Vorteil, dass die Software wie einzelne Bausteine zusammengesetzt ist. So schafft man es, moderne Architektur-Ansätze unveränderbar und verteilte zu implementieren. \cite{DockerVsVm}
