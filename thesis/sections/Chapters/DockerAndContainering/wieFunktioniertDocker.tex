\hthree{Wie funktioniert Docker}
\label{sec:dockerexplained}

Die Docker Technologie greift auf den Linux Kernel zurück, um die gewünschten Funktionen bereit zu stellen. Hierbei wird vom Kernel auf der einen Seite auf die sogenannten Kontrollgruppen zugegriffen, welche zum Beispiel für die CPU-Zeit, den Systemspeicher, oder die Netzwerkbandbreite und deren Aufteilung auf verschiedene Prozesse zuständig sind, als auch auf die "User Namespaces". So schafft es Docker, Prozesse und Applikationen getrennt und isoliert voneinander laufen zu lassen. \cite{Kontrollgruppen} \cite{DockerGrundlagen}

Docker ist ein imagebasiertes Bereitstellungsmodel und schafft es so Anwendungen und Pakete von Services gleichzeitig in unterschiedlichen Containern dem Anwender bereitzustellen. \cite{DockerGrundlagen}

Es ermöglicht die Bereitstellung von Container mit guter Performance und individueller Kontrolle der einzelnen Services und Versionen der Services. \cite{DockerGrundlagen}