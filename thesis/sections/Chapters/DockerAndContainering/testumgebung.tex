\hthree{Testumgebungen durch Containertechnologien}

Durch das Einsetzen von unterschiedlichsten Containertechnogien wie Docker, wird es Administratorinnen oder Administratoren leichter gemacht eine simple, aber vor Allem effiziente Testinfrastruktur und Testumgebung schnell und unkompliziert zu implementieren. Sollte zum Beispiel eine Version eines bestimmten Service einen Fehler beinhalten, können Container aushelfen. So können die Entwickler*innen denselben Service auf 2 unterschiedlichen Versionen in 2 separierten Containern laufen lassen und individuell prüfen ab welcher Version der Fehler auftritt. Außerdem können Container innerhalb kürzester Zeit gestartet und die Testumgebung angeworfen werden. Dies hat den großen Nutzen, dass kleine Infrastrukturänderungen, welche ab und an nötig sind, durch das Verwenden von Containertechnologie keine Probleme mehr darstellen.

Größere und komplexere Testszenarien können so durch die neu erlangte Flexibilität im Testen von Software und Systemen unterschiedlichster Art einfach und für den Administrator unkompliziert abgewickelt werden. Außerdem kann man das Testsystem, welches sich in einem Linux Container befindet, leichter transportieren als ein physisches System. \cite{TestenContainer}
