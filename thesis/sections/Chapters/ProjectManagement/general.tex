\hthree{Allgemeines Projektmanagement}
\sectionauthor{Richard Panzer}

Projektmanagement ist wichtig, um ein Projekt zu strukturieren, zu planen und im Endeffekt zeitgerecht fertig zu stellen. Ein Projekt im Allgemeinen ist ein Auftrag, welcher zeitlich begrenzt, risikobehaftet und daher auch von allen anderen Tagesgeschäften eines Unternehmens abgegrenzt. Da ein Projekt meistens recht komplex ist ist die Rolle eines Projektmanagers eingeführt worden um den gesamten Ablauf, welcher auch zwischen mehreren Abteilungen passieren kann, zu betreuen. \cite{Projectman.}

Im Laufe der Zeit entwickelten viele Leute unterschiedliche Ansätze im Projektmanagement um ein Projekt abzuwickeln und es im Endeffekt als erfolgreich abgehakt werden kann. Diese unterschiedlichen Ansätze gibt es, da ein Projekt nicht gleich ein Projekt ist und eine klassischere Art (siehe Kapitel \ref{sec:klassisch}) besser zu einem Projekt passt, während zu einem anderen Projekt die agilere Methode (siehe Kapitel \ref{sec:agile}) bei der Abwicklung besser funktioniert. \cite{Projectman.}

In den folgenden Kapiteln wird sowohl auf klassisches als auch auf agiles Management eingegangen und das Management in \ZELIA\ wird beleuchtet. \cite{Projectman.}


