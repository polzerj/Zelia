\hthree{Klassisches Projektmanagement}

\hfour{Grundlegendes}

Klassisches Projektmanagement zeichnet sich dadurch aus, dass zu Beginn des Projekts ein starrer Endzustand definiert wird, welcher durch viele Anforderungen belegt wird. Außerdem wird sehr schon im Vorhinein sehr viel in die detaillierte Planung von Prozessen und Zeitmanagement gesteckt. Zum Ende hin wird aber ein Zeitpuffer gelassen um auf mögliche Änderungen oder Verzögerungen vorbereitet zu sein und diese so abwickeln zu können, dass das Projekt trotzdem zeitgerecht fertiggestellt werden kann. Klassisches Projektmanagement hängt also vom Anfang bis zum Ende zusammen und greift ineinander. 

Somit kann gesagt werden, dass klassisches Projektmanagement eine hohe Planungssicherheit für das Unternehmen bietet. Jedoch ist die Planung zu Beginn im Bereich des Geldes oder der Zeit immer weniger wichtig wurde, da die Planung meistens nicht eingehalten werden konnte. Im Jahr 2015 wurde eine Studie durchgeführt, welche bestätigte, dass 71\% aller Projekte gar nicht, oder nur zu einem gewissen Teil fertig gestellt werden konnten. Das sagt nicht, dass klassisches Projektmanagement schlecht, oder nicht mehr zeitgemäß ist, sondern, dass sicher der Projektmanager im Vorhinein gut überlegen muss, welches Managementsystem am Besten für das Projekt geeignet ist.  \cite{Projectman.}

\hfour{Anwendungsgebiete}


