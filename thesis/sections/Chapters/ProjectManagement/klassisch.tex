\hthree{Klassisches Projektmanagement}

\hfour{Grundlegendes}

Klassisches Projektmanagement zeichnet sich dadurch aus, dass zu Beginn des Projekts ein starrer Endzustand definiert wird, welcher durch viele Anforderungen belegt wird. Außerdem wird sehr schon im Vorhinein sehr viel in die detaillierte Planung von Prozessen und Zeitmanagement gesteckt. Zum Ende hin wird aber ein Zeitpuffer gelassen um auf mögliche Änderungen oder Verzögerungen vorbereitet zu sein und diese so abwickeln zu können, dass das Projekt trotzdem zeitgerecht fertiggestellt werden kann. Klassisches Projektmanagement hängt also vom Anfang bis zum Ende zusammen und greift ineinander. 

Somit kann gesagt werden, dass klassisches Projektmanagement eine hohe Planungssicherheit für das Unternehmen bietet. Jedoch ist die Planung zu Beginn im Bereich des Geldes oder der Zeit immer weniger wichtig wurde, da die Planung meistens nicht eingehalten werden konnte. Im Jahr 2015 wurde eine Studie durchgeführt, welche bestätigte, dass 71\% aller Projekte gar nicht, oder nur zu einem gewissen Teil fertig gestellt werden konnten. Das sagt nicht, dass klassisches Projektmanagement schlecht, oder nicht mehr zeitgemäß ist, sondern, dass sicher der Projektmanager im Vorhinein gut überlegen muss, welches Managementsystem am Besten für das Projekt geeignet ist.  \cite{Projectman.}

\hfour{Anwendungsgebiete}

Klassisches Projektmanagement wird in Projekten verwendet, bei denen man weiß, dass mit wenigen Veränderungen zu rechnen ist. Es kann also von Beginn an gut und verlässlich in den Bereichen Personal, Kosten und Terminen geplant werden und es wird so eine hohe Planungssicherheit mitgebracht. Veränderungen in den vorher genannten Faktoren sind dann aber meistens mit hohen Kosten verbunden und werden so gut es geht vermieden.

Es empfiehlt sich Klassisches Projektmanagement zum Beispiel bei Projekten zu verwenden, welche in der Vergangenheit schon einmal durchgeführt wurden. Es gibt als schon Vorerfahrungen, welche in die fixe Planung des Projekts mit einfließen werden. Zusammengefasst kann aber gesagt werden, dass im Vorhinein viel Zeit in das Ausprobieren von unterschiedlichen Managementsystemen gesteckt werden soll, damit das am ehesten passende verwendet wird. \cite{Projectman.}

\hfour{"PRINCE2"}

"PRINICE2" ist auf der Welt die am häufigst verwendete Methode für klassisches Projektmanagement. "PRINCE" steht in diesem Fall für "Projects in Controlled Environments" und ist als Regierungsstandard für Projekte in der Informatik entwickelt worden. Jedoch wurde die Methode auch immer häufiger auch außerhalb der Informatik verwendet und wurde 1996 mit dem Namen "PRINCE2" als Methode für Projektmanagement im Allgemeinen vorgestellt.


