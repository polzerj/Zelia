\hthree{Mersed Kečo}
\sectionauthor{Mersed Kečo}

\begin{description}
    \item[20.09.2021-26.09.2021] Kennenlernen von Node.js und das Verständnis der WebUntis API. Desweiteren die ersten Schritte mithilfe von GitHub. Erste Inbetriebnahme der WebUntis API. 
    \item[27.09.2021-03.10.2021] Modellierung des Datenbanksschemas.Als Gruppe: Erstellen von Arbeitspaketen um einen Überblick zu behalten.
    \item[04.10.2021-10.10.2021] Verwaltung von Arbeitspaketen.Verwendung des WebUntis Wrapper's und alle dafür notwendigen Grundlagen. Umschreiben des Code von Node.js in TypeScript.
    \item[11.10.2021-17.10.2021] Helfen bei dem Designen des Frontends. Designen der ZELIA API. Planen der zu erreichenden Ziele und Arbeitspakete für die Alphaversion. Meeting mit den Diplomarbeitsbetreeung um die Ziele der Alphaversion festzuhalten. Code für die Beschaffung der Rauminformationen über WebUntis API schreiben. 
    \item[18.10.2021-24.10.2021]
    \item[25.10.2021-31.10.2021]
    \item[01.11.2021-07.11.2021]
    \item[08.11.2021-14.11.2021]
    \item[15.11.2021-21.11.2021]
    \item[22.11.2021-28.11.2021]
    \item[29.11.2021-05.12.2021]
    \item[06.12.2021-12.12.2021]
    \item[13.12.2021-19.12.2021]
    \item[20.12.2021-26.12.2021]
    \item[27.12.2021-02.01.2022]
    \item[03.01.2022-09.01.2022]
    \item[10.01.2022-16.01.2022]
    \item[17.01.2022-23.01.2022]
    \item[24.01.2022-30.01.2022]
    \item[31.01.2022-06.02.2022]
    \item[07.02.2022-13.02.2022]
    \item[14.02.2022-20.02.2022]
\end{description}