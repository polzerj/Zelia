\hthree{Mersed Kečo}
\sectionauthor{Mersed Kečo}

\begin{description}
    \item[20.09.2021-26.09.2021] Kennenlernen von Node.js und das Verständnis der WebUntis API. Desweiteren die ersten Schritte mithilfe von GitHub. Erste Inbetriebnahme der WebUntis API. 
    \item[27.09.2021-03.10.2021] Modellierung des Datenbanksschemas.Als Gruppe: Erstellen von Arbeitspaketen um einen Überblick zu behalten.
    \item[04.10.2021-10.10.2021] Verwaltung von Arbeitspaketen.Verwendung des WebUntis Wrapper's und alle dafür notwendigen Grundlagen. Umschreiben des Code von Node.js in TypeScript.
    \item[11.10.2021-17.10.2021] Helfen bei dem Designen des Frontends. Designen der ZELIA API. Planen der zu erreichenden Ziele und Arbeitspakete für die Alphaversion. Meeting mit den Diplomarbeitsbetreeung um die Ziele der Alphaversion festzuhalten. Code für die Beschaffung der Rauminformationen über WebUntis API schreiben. 
    \item[18.10.2021-24.10.2021] Implementieren des GetRoomInfo Controller. Error-Handling und Zurückliefern eines HTTP-Statuscodes beim Abfangen eines WebUntis-Fehlers. FInale Abgaben für die Alpha-Version. DA-Meeting mit Betreuern bezüglich Alpha-Version. Beginn der Dokumentation.
    \item[25.10.2021-31.10.2021] Kickoff der Planung für die Beta-Version und das Zusammenschreiben von Arbeitspaketen. Entfernen der Präfixe der Klassenräume. Weiterschreiben an der Dokumentation.
    \item[01.11.2021-07.11.2021] Den Ablauf der WebUntis-Session behandeln und den automatischen Re-Login einbauen. Diplomarbeitdokumentation erweitern. 
    \item[08.11.2021-14.11.2021] Meeting mit Diplomarbeitsetreuern und Feedback zum momentanen Stand
    \item[22.11.2021-28.11.2021] Meeting für die Beta-Version und Feedback der Diplomarbeitsbetreuer. Errorhandling für die ZELIA-API einbauen. Die grundlegenden Fehler der Middleware verstehen und die versuchen, abzufangen. Implementierung des Room Booking Controllers. 
    \item[29.11.2021-05.12.2021] Implementierung von wichtigen WebUntis Interfaces. Planen der noch zusätzlich erforderlichen Änderungen für die Beta-Version. Implementierung des Room-Information-Services. Beheben von Importing-Fehlern. Weiterschreiben an der Dokumentation.
    \item[06.12.2021-12.12.2021] Implementierung der AuthenticationMiddleware und des AdminRequestController für die ZELIA-API. Meeting für das Troubleshooting der Fehler in der Betaversion von ZELIA. Dokumentation weiterschreiben. 
    \item[13.12.2021-19.12.2021] Verbinden des AdminRequestController zur Datenbank. Den auslaufend WebUntis Token abfangen und erneuern. Präsentation der Veränderungen der Betaversion im Vergleich zum Meeting am 22.11.2021.
    \item[20.12.2021-26.12.2021] Diplomarbeitsdokumentation weiterschreiben.
    \item[10.01.2022-16.01.2022] Coronabedingt nicht anwesend/Diplomarbeitdokumentation weiterschreiben
    \item[17.01.2022-23.01.2022] Coronabedingt nicht anwesend/Diplomarbeitdokumentation weiterschreiben
    \item[24.01.2022-30.01.2022] Coronabedingt nicht anwesend/Diplomarbeitdokumentation weiterschreiben. Installieren von LaTeX und umschreiben der Dokumentation in LaTeX.
    \item[31.01.2022-06.02.2022] Erweiterung der Dokumentation, Proof-Reading der geschriebenen Kapitel
    \item[07.02.2022-13.02.2022] Erweiterung der Dokumentation
    \item[14.02.2022-20.02.2022] Erweiterung und Verbesserung der Dokumentation, Proof-Reading und Themenneufindung. Neuinstallation von LaTex. Designen des Werbesbanners für ZELIA. Projekttagebuch schreiben
\end{description}