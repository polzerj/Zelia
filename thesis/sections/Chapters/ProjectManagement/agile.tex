\hthree{Agiles Projektmanagement}

\hfour{Grundlegendes}

Grundlegend kann gesagt werden, dass beim agilen Projektmanagement das gesamte Team in den Bereichen Umfang, Zeit und Kosten über eine hohe Flexibilität verfügt. Außerdem wird der Kunde in den gesamten Prozess des Projekts miteinbezogen und immer auf dem Laufenden gehalten, in welchem Stand sich das Projekt befindet. Der Fokus liegt beim agilen Projektmanagement sehr stark auf dem Endprodukt und nicht auf Prozesse, welche im Vorhinein fix und starr definiert werden müssen. Also das Einhalten von Terminen, Kosten oder eines fixen Arbeitsaufwands wird eher oder ganz vernachlässigt. 

Dieses Dynamische und diese Flexibilität wird durch das "agil" zum Ausdruck gebracht. Änderungen im Ablauf des Projekts können unkompliziert eingearbeitet werden. Hierbei spricht man von sogenannten "Änderungsanträgen". Ein Änderungsantrag kann zum Beispiel einen der folgenden Punkte enthalten:

\begin{itemize}
    \item Verbesserungen für das Produkt
    \item Eine Terminverschiebung
    \item Eine Erweiterung des Projektteams
    \item Die Reduzierung von Kosten für das Projekt
\end{itemize}

Die Bedeutung von "Agil" hebt auch insbesondere die Vorteile von geringer Planungssicherheit hervor und besetzt diese positiv. \cite{agil} \cite{Aenderung}

\hfour{Anwendungsgebiete}

Die Verwendung von agilem Projektmanagement bringt entscheidende Vorteile mit sich, wenn das Projekt:

\begin{itemize}
    \item keine klaren und deutlich definierten Anforderungen hat.
    \item vielen und ständigen Veränderungen in der Planung ausgesetzt ist.
    \item komplexere Ziele hat.
    \item kein deutliches Endziel hat oder kein vordefiniertes Endprodukt ist.
    \item schnelle Ergebnisse liefern muss und keine Zeit für langwierige Planung bleibt.
\end{itemize}

\cite{AnwendungAgil}

\hfour{Scrum}


