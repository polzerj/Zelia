\hthree{Julian Kusternigg}
\sectionauthor{Julian Kusternigg}

\begin{description}
    \item[20.09.2021-26.09.2021]Beginn der Entwicklung des Komponentensystems und des Client-Side-Routers. Erste implementierung von Teseract OCR. (beides mit Johannes Polzer)
    \item[27.09.2021-03.10.2021] Vervollständigung der OCR implementierung und testen des Services. Als ganze Gruppe: Erstellen von Arbeitspaketen um einen Überblick zu behalten.
    \item[04.10.2021-10.10.2021] Erste Komponente "OCR" um Teseract weiter zu testen.
    \item[11.10.2021-17.10.2021] Weitere Komponenten für Raumeingabe, um die Nummer manuell eingeben zu können und Rauminformation, um später Daten anzuzeigen. Meeting und Planung der Alphaversion. Schnittstellen definern und statische Testwerte von der API aussenden um implementierte Rauminfo abfrage Testen zu können. Erster entwurf des Frontends. Verbesserung des Frontend-Routers um Pfadvariablen zu unterstüzen. Realisierung eines Loggers.
    \item[18.10.2021-24.10.2021] Erste implementierung von der Komponente um den Stundenplan anzuzeigen. Rauminfokomponente wurde überarbeitet. Unterstützung bei erstellung des Datenbankschemas. Fertigstellung und Nachbesprechung der Alphaversion.
    \item[25.10.2021-31.10.2021] Erste Notizen für die Dokumentation gemacht. Funktionalität der Raumeingabe verbessert. Pfadvaribalenfehler behoben und erneute überarbeitung des Designs im Frontend (mit Johannes Polzer). Gemeinsames planen für Herbstferien und Betaversion.
    \item[01.11.2021-07.11.2021] OCR Modul optimiert und Bildvorbearbeitung einfgeführt. Rechachieren über SSL Zertifikate und HTTP/S proxys mit Nginx (mit Johannes Polzer). Flexibles CSS ins Frontend eingeführt und Fehler bei Raumeingabe behoben.
    \item[08.11.2021-14.11.2021] Entwurf eines Logos. Meeting mit den Vorgesetzten. Implementierung der Debugkomponente um auf Mobilgeräten den Logger anzuzeigen. Überarbeitung des Regex Musters der OCR Ergebnis für alle Raumnummern (beides mit Johannes Polzer).
    \item[15.11.2021-21.11.2021] Iplementierung der Komponente um Meldungen zu tätigen zu können. Meeting für Betaversion und Nachbesprechung mit den Vorgesetzten. 
    \item[22.11.2021-28.11.2021] Experiment zur Fehler abarbeitung in der "Middleware" vom Server. Fertigstellung der Raummeldungskomponente (mit Johannes Polzer).
    \item[29.11.2021-05.12.2021] Unterstüzung bei der Implementierung des Raumbuchungs-Controllers für die API. Weitere Planungen für Betaversion.
    \item[06.12.2021-12.12.2021] Begin der Entwicklung der Komponenten für Admin-Login und Admin-Dashboard. Weitere Komponenten: "Report-Handle" und "Booking-Handle" um als Admin Meldungen und Buchungen abarbeiten zu können. Das Buchungsformular auf die Rauminfoseite hinzugefügt. Login Controller auf dem Server implementiert (mit Johannes Polzer). Betaversion Meeting und Troubleshooting.
    \item[13.12.2021-19.12.2021] Adminseite mit flexiblem CSS überarbeitet. Unterstützung bei API-DB Verbindung. DB-Docker Fehler gesucht und behoben (mit Johannes Polzer). Filter für Admin-Dashboard eingebaut (mit Johannes Polzer). Neuen Stand präsentieren.
    \item[27.12.2021-02.01.2022] Begin der Dokumentation über WebComponents.
    \item[03.01.2022-09.01.2022] Erweiterung der WebComponents-Dokumentation und begin der Doku für den Client-Side-Router.
    \item[17.01.2022-23.01.2022] Kleine erweiterungen der WebComponent und Client-Side-Router Dokumentation. Begin der "Optical Character Recognition" Doku und der generellen Einleitung in die Arbeit. 
    \item[24.01.2022-30.01.2022] Umstieg von Word auf LaTeX und einarbeitung des Feedbacks von Vorgesetzten (mit Johannes Polzer). 
    \item[31.01.2022-06.02.2022] Überarbeitung und erweiterung von eigenen Teilen der Dokumentation. 
    \item[07.02.2022-13.02.2022] Gestaltung eines Werbeplakates
    \item[14.02.2022-20.02.2022] Einabreitung von Messungen des OCR-Prozesses in die Doku und verbesserungen eigener Teile.
    \item[21.02.2022-27.02.2022] Schreiben des Benutzer*innenhandbuch. Überarbeitung der anderen Kapitel. Entwerfen von Skizzen für bessere Verständlichkeit.
\end{description}