\hthree{Julian Kusternigg}
\sectionauthor{Julian Kusternigg}

\begin{description}
    \item[20.09.2021-26.09.2021]Beginn der Entwicklung des "Komponentensystems" und des "Client-Side-Routers". Erste implementierung von Teseract OCR. 
    \item[27.09.2021-03.10.2021] Vervollständigung der OCR implementierung und testen des Services. Erstellen von Arbeitspaketen um einen Überblick zu behalten.
    \item[04.10.2021-10.10.2021] Erste Komponente "OCR" um Teseract weiter zu testen.
    \item[11.10.2021-17.10.2021] Weitere Komponenten für Raumeingabe, um die Nummer manuell eingeben zu können und Rauminformation, um später Daten anzuzeigen. Meeting und Planung der Alphaversion. Schnittstellen definern und statische Testwerte von der API aussenden um implementierte Rauminfo abfrage Testen zu können. Erster entwurf des Frontends. Verbesserung des Fronend-Routers um Pfadvariablen zu unterstüzen. Realisierung eines Loggers.
    \item[18.10.2021-24.10.2021] Erste implementierung von der Komponente um den Stundenplan anzuzeigen. Rauminfokomponente wurde überarbeitet. Unterstützung bei erstellung des Datenbankschemas. Fertigstellung und Nachbesprechung der Alphaversion.
    \item[25.10.2021-31.10.2021] Erste Notizen für die Dokumentation gemacht. Funktionalität der Raumeingabe verbessert. Pfadvaribalenfehler behoben. Überarbeitung des Designs im Frontend. Planung für Herbstferien und Betaversion.
    \item[01.11.2021-07.11.2021] OCR Modul optimiert und Bildvorbearbeitung einfgeführt. Rechachieren über SSL Zertifikate und HTTP/S proxys mit Nginx. Flexibles CSS ins Frontend eingeführt und Fehler bei Raumeingabe behoben.
    \item[08.11.2021-14.11.2021] Entwurf eines Logos. Meeting mit Vorgesetzten. Implementierung der Debugkomponente um auf Mobilgeräten den Logger anzuzeigen. Überarbeitung des Regex Musters für OCR Ergebnis.
    \item[15.11.2021-21.11.2021] Komponente um Meldungen zu tätigen. Meeting für Betaversion nachbesprechung mit Vorgesetzten. 
    \item[22.11.2021-28.11.2021] Experiment zur Fehler abarbeitung in der "Middleware" vom Server. Fertigstellung der Raummeldungskomponente.
    \item[29.11.2021-05.12.2021] Unterstüzung bei der Implementierung des Raumbuchungs-Controllers für die API. Weitere Planungen für Betaversion.
    \item[06.12.2021-12.12.2021] Begin der Entwicklung der Komponenten für Admin-Login und Admin-Dashboard. Weitere Komponenten: "Report-Handle" und "Booking-Handle" um als Admin Meldungen und Buchungen abarbeiten zu können. Das Buchungsformular auf die Rauminfoseite hinzugefügt. Login Controller auf dem Server implementiert. Betaversion Troubleshooting-Meeting
    \item[13.12.2021-19.12.2021] Adminseite mit flexiblem CSS überarbeitet. Unterstützung bei API-DB Verbindung. DB-Docker Fehler gesucht und behoben. Filter für Admin-Dashboard eingebaut. Presentierung des neuen Stands.
    \item[20.12.2021-26.12.2021] - 
    \item[27.12.2021-02.01.2022] Begin der Dokumentation über WebComponents.
    \item[03.01.2022-09.01.2022] Erweiterung der WebComponents-Dokumentation und begin der Doku für den Client-Side-Router.
    \item[10.01.2022-16.01.2022] - 
    \item[17.01.2022-23.01.2022] Kleine erweiterungen der WebComponent und Client-Side-Router Dokumentation. Begin der "Optical Character Recognition" Doku und der generellen Einleitung in die Arbeit. 
    \item[24.01.2022-30.01.2022] Umstieg von Word auf LaTeX. Einarbeitung des Feedbacks von Vorgesetzten. 
    \item[31.01.2022-06.02.2022] Überarbeitung und erweiterung von eigenen Teilen der Dokumentation. Einabreitung von Messungen des OCR-Prozesses in die Doku.
    \item[07.02.2022-13.02.2022]
    \item[14.02.2022-20.02.2022] 
\end{description}