\hthree{Julian Kusternigg}
\sectionauthor{Julian Kusternigg}

\begin{description}
    \item[20.09.2021-26.09.2021]Beginn der Entwicklung des Komponentensystems und des Client-Side-Routers. Erste Implementierung von Teseract OCR. Beides mit Johannes Polzer.
    \item[27.09.2021-03.10.2021] Vervollständigung der OCR Implementierung und testen des Services. Als ganze Gruppe: Erstellen von Arbeitspaketen um einen Überblick zu behalten.
    \item[04.10.2021-10.10.2021] Erste Komponente "OCR" um Tesseract weiter zu testen.
    \item[11.10.2021-17.10.2021] Weitere Komponente eingeführt, um die Nummer manuell eingeben zu können. Andere Komponente entwickelt, um später Rauminformationen anzuzeigen. Meeting und Planung der Alphaversion. Schnittstellen definieren und statische Testwerte von der API aussenden, um implementierte Rauminfoabfragen testen zu können. Erster Entwurf des Frontends. Verbesserung des Frontend-Routers um Pfadvariablen zu unterstützen. Realisierung eines Loggers.
    \item[18.10.2021-24.10.2021] Erste Implementierung von der Komponente um den Stundenplan anzuzeigen. Rauminfokomponente wurde überarbeitet. Unterstützung bei Erstellung des Datenbankschemas. Fertigstellung und Nachbesprechung der Alphaversion.
    \item[25.10.2021-31.10.2021] Erste Notizen für die Dokumentation gemacht. Funktionalität der Raumeingabe verbessert. Pfadvariablenfehler behoben und erneute Überarbeitung des Designs im Frontend (mit Johannes Polzer). Gemeinsames planen für Herbstferien und Betaversion.
    \item[01.11.2021-07.11.2021] OCR Modul optimiert und Bildvorbearbeitung eingeführt. Recherchieren über SSL Zertifikate und HTTP/S Proxys mit Nginx (mit Johannes Polzer). Flexibles CSS ins Frontend eingeführt und Fehler bei Raumeingabe behoben.
    \item[08.11.2021-14.11.2021] Entwurf eines Logos. Meeting mit den Vorgesetzten. Implementierung der Debug-Komponente um auf Mobilgeräten den Logger anzuzeigen. Überarbeitung des Regex Musters der OCR Ergebnisse für alle Raumnummern (beides mit Johannes Polzer).
    \item[15.11.2021-21.11.2021] Implementierung der Komponente um Meldungen tätigen zu können. Meeting für Betaversion und Nachbesprechung mit den Vorgesetzten. 
    \item[22.11.2021-28.11.2021] Experiment zur Fehlerabarbeitung in der "Middleware" vom Server. Fertigstellung der Raummeldungskomponente (mit Johannes Polzer).
    \item[29.11.2021-05.12.2021] Unterstützung bei der Implementierung des Raumbuchungs-Controllers für die API. Weitere Planungen für Betaversion.
    \item[06.12.2021-12.12.2021] Beginn der Entwicklung der Komponenten für Admin-Login und Admin-Dashboard. Weitere Komponenten: "Report-Handle" und "Booking-Handle" um als Admin Meldungen und Buchungen abarbeiten zu können. Das Buchungsformular auf die Rauminfoseite hinzugefügt. Login Controller auf dem Server implementiert (mit Johannes Polzer). Betaversion Meeting und Troubleshooting.
    \item[13.12.2021-19.12.2021] Admin-Seite mit flexiblem CSS überarbeitet. Unterstützung bei API-DB Verbindung. DB-Docker Fehler gesucht und behoben (mit Johannes Polzer). Filter für Admin-Dashboard eingebaut (mit Johannes Polzer). Neuen Stand präsentiert.
    \item[27.12.2021-02.01.2022] Beginn der Dokumentation über WebComponents.
    \item[03.01.2022-09.01.2022] Erweiterung der WebComponents-Dokumentation und Beginn der Doku für den Client-Side-Router.
    \item[17.01.2022-23.01.2022] Kleine Erweiterungen der WebComponents und Client-Side-Router Dokumentation. Beginn der "Optical Character Recognition" Doku und der generellen Einleitung in die Arbeit. 
    \item[24.01.2022-30.01.2022] Umstieg von Word auf LaTeX und Einarbeitung des Feedbacks von Vorgesetzten (mit Johannes Polzer). 
    \item[31.01.2022-06.02.2022] Überarbeitung und Erweiterung von eigenen Teilen der Dokumentation. 
    \item[07.02.2022-13.02.2022] Gestaltung eines Werbeplakates.
    \item[14.02.2022-20.02.2022] Einarbeitung von Messungen des OCR-Prozesses in die Doku und Verbesserungen eigener Teile.
    \item[21.02.2022-27.02.2022] Schreiben des Benutzer*innenhandbuchs. Überarbeitung der anderen Kapitel. Entwerfen von Skizzen für bessere Verständlichkeit.
\end{description}