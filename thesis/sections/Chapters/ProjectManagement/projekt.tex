\hthree{Projektmanagement der Diplomarbeit}\label{sec:ProManZELIA}

Das Projektmanagement während des Projektes war eine Mischform aus klassischem und agilen Management, da mit flexiblen Sprints, aber auch mit fixen Deadlines, gearbeitet wurde. Ein Großteil des Managements wurde mit der Versionierungssoftware GitHub (siehe Kapitel "Github" \ref{sec:github}, Seite \pageref{sec:github}) durchgeführt.

Zu Beginn wurde durch das Schreiben des Projektantrags die Anforderungen für die Software definiert, welche bis zum Ende der Diplomarbeit umgesetzt werden müssen. Außerdem wurden fünf Meilensteine festgelegt, an denen immer etwas mehr von der Funktionalität fertig sein musste.

Die einzelnen Arbeitspakete sind bei gemeinsamen Meetings mit allen Teammitgliedern aus den großen Meilensteinen heruntergebrochen worden und wurden in einem Tabellenprogramm niedergeschrieben. Außerdem wurden bereits zu Beginn Abhängigkeiten festgestellt und die Deadlines dementsprechend angesetzt. Weiters wurde die festgelegte Deadline auch direkt in die Tabelle eingetragen und daraus die verbleibenden Tage zum Arbeiten berechnet.  

Unter der Woche hat jedes Teammitglied an seinem Teil gearbeitet und weiter programmiert. Jede Woche gab es ein circa einstündiges Meeting, bei dem jeder davon berichtete, an was er im Moment arbeitet und ob es damit irgendwo Schwierigkeiten gibt, bei denen die Unterstützung der anderen Teammitglieder benötigt wird. So war für alle der Fortschritt des Projekts sichtbar und etwaige Schwierigkeiten konnten früh beseitigt werden. Wenn mehrere Entwickler an einen Arbeitspaket gearbeitet haben, wurde zwischen den Entwicklern auch außerhalb dieses wöchentlichen Meetings viel miteinander kommuniziert.

Sobald eine Anforderung implementiert wurde und der Entwickler der Ansicht war, dass es so passt, wurde mittels "Vier-Augen-Prinzip" auf den neu geschriebenen Code geschaut und ein Review getätigt (siehe Kapitel \ref{sec:github}, Seite \pageref{sec:github}). Dies dient dazu, vor dem Zusammenführen von neuem Code mit schon bestehendem eine gewisse Sicherheit zu haben, dass die Gesamtsoftware nach dem Zusammenführen auch noch funktioniert.

Vor eine größeren Zwischenabgabe setzte sich das gesamte Team am Tag davor zusammen und sprach über den vergangenen Sprint und es wurden gemeinsam Softwaretests durchgeführt, um festzustellen, ob für die Zwischenpräsentation am nächsten Tag alles ordnungsgemäß funktioniert. 
