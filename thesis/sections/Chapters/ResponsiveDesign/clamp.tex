\hfour{Clamp-Funktion}\label{sec:clamp}

Die Clamp-Funktion ermöglicht es, ein Element dynamisch zu skalieren, ohne dafür Media-Queries zu verwenden. 
Dazu übernimmt sie drei Parameter. 
Der Erste ist der Minimalwert, der in einer fixen Einheit, wie \zb\ Pixel ({\ttfamily px}), angegeben wird. 
Der zweite Parameter gibt den optimalen Wert an. 
Dieser wird in einer Einheit angegeben, welche abhängig von der Breite des Bildschirms ({\ttfamily vw}) bzw. des übergeordneten HTML-Elements ({\ttfamily \%}) ist. 
Der dritte Parameter ist die Maximalbreite, die so wie die Minimalbreite in einer fixen Einheit angegeben wird. 

Die Einheit {\ttfamily vw} steht für "Viewport Width" und gibt prozentuell die Breite relativ zum Viewport an. Der Viewport ist die Größe des Bereiches der Web-App, welcher am Bildschirm gleichzeitig angezeigt werden kann. 
Mit dem Prozentzeichen ({\ttfamily \%}) kann die Größe relativ zum übergeordneten Element angegeben werden.

Das folgende Element wird von mindestens 100 Pixel mit 80-prozentiger Bildschirmbreite bis zu 800px Breite skaliert.

\css{code/ResponsiveDesign/clamp.css}{Beispiel "clamp"-Funktion}
