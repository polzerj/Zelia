\hfour{Media Queries}\label{sec:mediaQueries}

"CSS-Media-Queries" ermöglichen es Webentwicklern, Styles für eine Webseite in Abhängigkeit von Medientyp, Ansichtsgröße und Ausrichtung zu schreiben. 
Die "View Size" ist die Größe des Inhalts, der auf einem Bildschirm gleichzeitig angezeigt werden kann.
Diese wird in Pixeln ({\ttfamily px}) gemessen. 
Es ist auch möglich, die Webseite anders zu gestalten, je nachdem, ob sie auf dem Bildschirm angezeigt oder ausgedruckt wird. 

\begin{minipage}{\textwidth}
    Jede Media Query hat folgende Struktur:
    
    \css{code/ResponsiveDesign/MediaQueries.css}{Beispiel Media-Queries}
\end{minipage}

Es gibt verschiedene Medien, auf die ein unterschiedliches Design angewandt werden kann: 
\begin{itemize}
    \item screen -- Style wird nur auf dem Bildschirm angewandt.
    \item print -- Style wird auf die Druckfunktion angewandt.
\end{itemize}

Zusätzlich gibt es die Option "all", welche beide "media-types" beeinflusst. \cite{w3MediaQueries}

Beispiele für "media-features" sind

\begin{itemize}
    \item max-width -- beeinflusst kleinere Bildschirme als die in Pixeln angegebene Größe
    \item min-width -- beeinflusst größere Bildschirme als die in Pixeln angegebene Größe
    \item orientation -- prüft, ob das Gerät im horizontalen oder vertikalen Modus verwendet wird
    \item prefers-color-scheme -- prüft, ob das Gerät ein helles oder dunkles Theme verwendet
\end{itemize}
