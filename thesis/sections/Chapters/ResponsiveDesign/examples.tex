\hthree{Beispiele}
\hfour{Vergleich: Light- und Dark-Theme}

Die \ZELIA\ App lässt sich sowohl im Light- als auch im Dark-Theme anzeigen (siehe Abbildung \ref{fig:theme}). Dafür werden Media Queries (siehe Kapitel \ref{sec:mediaQueries}) verwendet, welche das Theme des Browsers abfragen (siehe Code \ref{code:theme}). 
Die jeweiligen Farben werden in CSS-Variablen gespeichert. Dadurch müssen die Farben nur einmal in einer Media Query angepasst werden und können nun für jedes Element verwendet werden. 

\begin{figure}[H]
    \begin{subfigure}[c]{0.49\textwidth}
        \centering
        \includegraphics[width=\textwidth]{media/ResponsiveDesign/ZeliaDesktop.png}
        \caption{Helles Theme}
    \end{subfigure} \hfill
    \begin{subfigure}[c]{0.49\textwidth}
        \centering
        \includegraphics[width=\textwidth]{media/ResponsiveDesign/ZeliaDesktopDark.png}
        \caption{Dunkles Theme}
    \end{subfigure}
    \caption{Gegenüberstellung: helles - dunkles Theme}
    \label{fig:theme}
\end{figure}

\css[code:theme]{code/ResponsiveDesign/theme.css}{Anpassung der Farben anhand des Themes}

\clearpage
\hfour{Vergleich: Mobil und Desktop}

Da \ZELIA\ sowohl am Computer als auch am Smartphone oder Tablet genutzt werden soll, wird mithilfe der Clamp-Funktion (siehe Kapitel \ref{sec:clamp}) das Design für die verschiedenen Bildschirmgrößen angepasst (siehe Abbildung \ref{fig:info}).

\begin{figure}[H]
    \begin{subfigure}[c]{0.64\textwidth}
        \centering
        \includegraphics[width=\textwidth]{media/ResponsiveDesign/ZeliaDesktop.png}
        \caption{Rauminfo (Desktop)}
    \end{subfigure} \hfill
    \begin{subfigure}[c]{0.34\textwidth}
        \centering
        \includegraphics[width=\textwidth]{media/ResponsiveDesign/ZeliaMobile.png}
        \caption{Rauminfo (Mobil)}
    \end{subfigure}
    \caption{Vergleich von Desktop- und Mobile-Versionen der Rauminformation}
    \label{fig:info}
\end{figure}

Wenn \ZELIA\ auf einem Mobilgerät verwendet wird, dann wird der "Kamera öffnen"-Button (siehe Abbildung \ref{fig:homeMobile}) dafür angezeigt, damit die Raumnummer mittels OCR (siehe Kapitel \ref{sec:ocr}) eingelesen werden kann. Auf der Desktop-Version wird dieser Button nicht angezeigt (siehe Abbildung \ref{fig:homeDesktop}).

\begin{figure}[H]
    \begin{subfigure}[c]{0.64\textwidth}
        \centering
        \includegraphics[width=\textwidth]{media/ResponsiveDesign/ZeliaHome.png}
        \caption{Startseite (Desktop)}
        \label{fig:homeDesktop}
    \end{subfigure} \hfill
    \begin{subfigure}[c]{0.34\textwidth}
        \centering
        \includegraphics[width=\textwidth]{media/ResponsiveDesign/ZeliaHomeMobile.png}
        \caption{Startseite (Mobil)}
        \label{fig:homeMobile}
    \end{subfigure}
    \caption{Vergleich von Desktop- und Mobile-Version der Startseite}
    \label{fig:home}
\end{figure}