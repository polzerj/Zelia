\hthree{Allgemeines}

Immer mehr Menschen nutzen ihr Smartphone oder Tablet, um im Internet zu surfen. 
Daher ist es wichtig, dass das Design einer Webseite für verschiedene Gerätegrößen und Ausrichtungen angepasst wird. 
Diese Anpassungen werden "Responsive Design" genannt. 
Eine "responsive" Webseite ändert ihr Layout je nach Bildschirmgröße. \cite{MDNResponsiveDesign}

Beispiel: 
Bei den meisten Webseiten verkleinert sich die Navigationsleiste auf mobilen Geräten auf eine einzige Menüschaltfläche, damit mehr Platz für den Inhalt zur Verfügung steht.
Außerdem sind Knöpfe auf Mobilgeräten meist größer, um sie besser mit dem Finger antippen zu können. 

Da \ZELIA\ vor allem auf Mobilgeräten benutzt werden soll, ist es sehr wichtig, dass die Webseite für jedes Gerät optimiert ist. 

% TODO: Ausbauen mit Screenshots