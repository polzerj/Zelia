\htwo{TypeScript}
\sectionauthor{Mersed Kečo}

TypeScript ist eine von Microsoft entwickelte Programmiersprache, welche auf JavaScript beziehungsweiße, dem ECMAScript-6-Standard basiert. TypeScript unterstützt mithilfe von Modulen das Kapseln von Klassen, Interfaces, Funktionen und Variablen in eigene Namensräume(Namespaces). Dabei wird zwischen internen Modulen, welche sich an den Standard anlehnen und externen Modulen, welche eine JavaScript-Bibliothek wie AMD oder CommonJS verwenden, unterschieden. \cite{TypeScript}

\typescript{code/TypeScript/jsvsts.ts}{Js vs. Ts}


Aufgrund des Zieles von TypeScript, nur die Erweiterung von JavaScript zu erreichen und beide Sprachen ein und denselben Sprachkern haben, ist der Code derselbe und keine Unterschiede sind bemerkbar. Die Erweiterungen, die von TypeScript erbracht werden haben also Ähnlichkeiten zu Java. \cite{TypeScript}


\hthree{Problemlösungen mithilfe von TypeScript}
Um sicherzustellen, dass der geschrieben Code keine Fehler enthält, gibt es die Möglichkeit automatisierte Tests anzufertigen und mithilfe dieser zu überprüfen, ob der Code in dieser Art und Weise, wie er geschrieben ist,funktioniert.
Nicht viele Unternehmen haben die Unternehmensgröße von Microsoft, doch sind die meisten Probleme, die bei der Programmierung von Code mit JavaScript auftreten, diesselben. Große Codebasen bestehen aus mehreren Tausend Dateien. Bei der Änderung eines einzelnen Wertes in einer Datei, kann sich das Verhalten dieser und aber auch anderer Dateien vollkommen verändern.
Die Validierung von Verbindungen kann schnell zeitaufwendig werden. Eine Sprache wie TypeScript kann dies automatisch erledigen und liefert während der Entwicklung sofortiges Feedback. Mithilfe dieser Funktion, versucht TypeScript Entwicklern mehr Vertrauen in ihren Code zu geben und spart viel Zeit bei der Validierung, um sicherzustellen, dass die Entwickler das Projekt nicht versehentlich beschädigt haben oder dieser überhaupt nach dem Fertigschreiben lauffähig ist. \cite{ScriptWiki}
