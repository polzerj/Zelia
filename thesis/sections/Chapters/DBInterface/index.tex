\sectionauthor{Mario Naunović}
\htwo{DB-Interface}

\hthree{Allgemein}

Das Datenbank-Interface ermöglicht es Daten, welche verschiedenen Controller im Back\-end-Code verarbeiten, in die Datenbank zu schreiben. Diese Daten sind beispielsweise Buchungen von Räumen oder auch Beschwerden. Durch das Interface können Daten nicht nur in die Datenbank geschrieben, sondern auch ausgelesen werden.

\hthree{Auslesen von Rauminformationen}

\typescript{code/DBInterface/GetRooms.ts}{Datenbankschnittstelle um Rauminformationen auslesen}

Die in der vorrigen Abbildung beschriebeme Funktion ermöglicht es, Rauminformation zu einem bestimmten Raum auszulesen. Parameter der Funktion ist die Raumnummer als String. Falls die Rauminformation für den angegebenen Raum ausgelesen werden kann, wird diese Information in Form eines Arrays mit einem Raum-Objekt zurückgeliefert. Falls während dem Auslesen aus der Datenbank ein Fehler auftritt, wird eine \emph{DatabaseNotAvailableException} aufgerufen. Falls die ausgelesenen Daten die Länge 0 besitzen, wurde der Raum nicht gefunden. In diesem Fall wird eine \emph{RoomNotFoundException} aufgerufen.

\hthree{Auslesen von Reports}

\typescript{code/DBInterface/GetReportByID.ts}{Datenbankschnittstelle um bestimmte Meldungen auslesen}

Mit dieser Funktion können alle Reports für einen bestimmten Raum ausgelesen werden. Parameter der Funktion ist die Raumnummer als String. Wenn die Reports fehlerfrei aus der Datenbank ausgelesen werden können, wird ein Array bestehend aus RoomReport-Objekten zurückgegeben. Falls während dem Lesen ein Fehler auftritt, wird eine \emph{DatabaseNotAvailableException} geworfen. Falls die Länge des Arrays 0 beträgt, wird eine \emph{RoomNotFoundException} geworfen.

\typescript{code/DBInterface/GetAllRoomReports.ts}{Datenbankschnittstelle um alle Meldungen auslesen}

Damit alle Reports für alle Räume auslesen zu können, gibt es die \emph{getAllRoomReports()} Funktion. Sie nimmt keine Parameter an. Wenn die Daten erfolgreich aus der Datenbank ausgelesen wurden, liefert die Funktion ein Array mit Report-Objekten zurück. Wenn während dem Auslesen aus der Datenbank ein Fehler auftritt, wird eine \emph{Database\-Not\-Available\-Exception} geworfen. Wenn nach dem Auslesen das Array die Länge 0 hat, wurden keine Reports gefunden. Es wird somit eine \emph{NoRoomReportsFoundException} geworfen.

\hthree{Reports schreiben}

\typescript{code/DBInterface/SetReport.ts}{Datenbankschnittstelle um Meldungen zu erstellen}

Damit Reports in die Datenbank geschrieben werden, wird die Funktion \emph{setRoomReportDbService} verwendet. Als Parameter wird ein Report als Report-Objekt mitgegeben. Wenn der Report nicht in die Datenbank geschrieben werden kann, wird eine \emph{CouldNotInsertDataException} geworfen.

\hthree{Reports verändern}

\typescript{code/DBInterface/UpdateReport.ts}{Datenbankschnittstelle um Meldungen zu erstellen}

Um bereits vorhandene Reports zu verändern, wird die Funktion \dots Oje, jetzt hab ich Corona und gehe nach Hause.