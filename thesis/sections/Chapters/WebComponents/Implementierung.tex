\hthree{Implementierung}

\hfour{Einleitung}
Wie ein der Rest von ZELIA ist auch der Code im Frontend in Typescript geschrieben, um einen besseren Überblick zu behalten. Zusätzlich bringt es noch Vorteile bei der Implementierung des Komponentensystems. Wie schon gesagt ist das Komponentensystem, neben dem Client-Side-Router, der zweite Teil unserer Frontend Bibliothek, den wir selbst entworfen haben, um unsere Webseite dazustellen zu können. Somit müssen wir uns nicht auf andere Frameworks verlassen, die in den meisten Fällen nicht so performant sind, da unsere Bibliothek perfekt an die Bedürfnisse von ZELIA angepasst ist.

\hfour{Architektur}
Wie eben im Standard beschrieben muss es eine Klasse geben, welche von "HTMLElement" erbt, um als "Custom HTML Element" verwendet werden zu können. So gibt es in der ZELIA Implementierung die abstrakte Klasse "Component", von der alle anderen Komponenten erben.