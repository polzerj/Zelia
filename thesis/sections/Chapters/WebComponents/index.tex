\hone{Web Components}
\htwo{Allgemeines}

Wie schon erwähnt läuft das ZELIA-Frontend im Browser, damit man von jedem Gerät darauf zugreifen kann. Durch die "Web Component" Technologie ist es möglich eigene HTML-Elemente zu definieren. Diese HTML-Elemente können dann nach belieben, mit allen ihrer "child"-Elementen, in die Webseite hinzugefügt werden. Somit kann man HTML-Code zusammenfassen und nur angeigen wenn es notwendig ist.

\hthree{Der "Web Component" Standard}

Web Components ermöglichen es den gleichen Code des Clients an mehreren Stellen zu verwenden. Dazu wurden drei neue Technologien \cite{MDNWebComponents} eingeführt:

\begin{itemize}
    \item Custom Elements
    \item Shadow DOM
    \item HTML Templates
\end{itemize}


\hfour{Custom Elements}

\hfive{Erstellung eines Custom Elements}

Custom Elements sind ES6 Klassen, welche von der Klasse "HTMLElement" erben. "ES6", ausgeschrieben "ECMA Script 6", ist der Standard, welcher Klassen in Javascript hinzugefügt hat.\cite{MDNes6Classes}

\typescript{code/WebComponents/one.ts}{Erstellung einer Custom Element Klasse}

Diese Klasse kann nun als Custom Element registriert werden. Dabei muss der Name des Elements vergeben werden, welcher einen Bindestrich enthalten muss.

\typescript{code/WebComponents/two.ts}{Registrierung des Custom Elements}

Wenn das Custom Element registriert ist, kann es mit dem vergebenen Namen im HTML verwendet werden.

\html{code/WebComponents/three.html}{Verwendung des Custom Elements im HTML}

\hfive{Lifecycle-Methoden \cite{MDNWebCustomElements}}

Um den Lebenszyklus eines Custom Elements besser zu kontrollieren, gibt es sogenannte "lifecycle callbacks". Diese "lifecycle callbacks" sind spezielle Methoden in einer Klasse, welche von der API der Custom Elements aufgerufen werden.

Der "connectedCallback" wird aufgerufen, wenn das Element in das DOM eingefügt wird. Das DOM (Document Object Model) ist eine Programmierschnittstelle über die man auf die einzelnen HTML Elemente zugreifen kann. \cite{MDNDOM} Darin können zum Beispiel Event-Listeners registriert werden.

\typescript{code/WebComponents/four.ts}{"connectedCallback" Beispiel}
   
Der "disconnectedCallback" ist das Gegenstück zum "connectedCallback" und wird beim Verlassen des DOMs aufgerufen. Darin sollten etwaige "Event Listeners" entfernt werden, um die nicht mehr benötigten Ressourcen freizugeben und dadurch Speicher und CPU-Zeit zu sparen.  

\typescript{code/WebComponents/five.ts}{"disconnectedCallback" Beispiel}

Der "adoptedCallback" wird jedes Mal aufgerufen, wenn das Element auf eine neue Seite verschoben wird.

\typescript{code/WebComponents/six.ts}{"adoptedCallback" Beispiel}

Der "attributeChangedCallback" Lifecycle-Hook wird jedes Mal aufgerufen, wenn ein Attribut des Custom Elements geändert wird. Allerdings muss dafür eine Liste an Attributnamen von der statischen Eigenschaft "observedAttributes" zurückgegeben werden. 
    
Der "attributeChangedCallback" nimmt drei Parameter:
\begin{itemize}
    \item Name: Name des Attributes
    \item Old Value: Den Wert vor der Änderung
    \item New Value: Der neue, aktuelle Wert 
\end{itemize}

\typescript{code/WebComponents/seven.ts}{"attributeChangedCallback" Beispiel}

\pagebreak
\hfour{Shadow DOM}

Unter DOM (Document Object Model) versteht man die Programmierschnittstelle, mit der auf die Elemente einer Webseite zugegriffen werden kann, um deren Inhalt und Darstellung zu verändern. \cite{MDNDOM}

Ein Shadow DOM ist ein Ast im DOM-Baum innerhalb der Webseite, welcher separat von der restlichen Seite gerendert wird. Dadurch kann auch auf Custom Elements ein Style in Style-Tags erstellt werden, ohne das Design der restlichen Seite zu beeinflussen. \cite{MDNShadowRoot}

Um einen Shadow Root zu erstellen, muss lediglich die Methode "attachShadow" aufgerufen werden. \cite{MDNShadowRoot}

\typescript{code/WebComponents/eight.ts}{Erstellung eines Shadow Roots}

Der "Mode" muss auf "open" gesetzt werden, damit der "ShadowRoot" befüllt werden kann. Nun kann mit "this.shadowRoot" auf den "ShadowRoot" innerhalb des Custom Components zugegriffen werden.
\cite{MDNShadowRoot}

\pagebreak
\hfour{HTML Templates}

Um das Custom Element mit Inhalten zu füllen, gibt es in HTML den "template"-Tag. Dieser wird nicht im DOM gerendert, kann aber trotzdem mit Javascript referenziert werden. \cite{MDNtemplate}

\html{code/WebComponents/nine.html}{Erstellung eines HTML Template}

Damit das Template als Custom Element gerendert wird, muss der Content in das dieses, oder dessen Shadow DOM kopiert werden.

\typescript{code/WebComponents/ten.ts}{Kopieren des Template in das Custom Element}

Um den Inhalt der Components dynamisch zu machen, gibt es den HTML "Slot"-Tag. 
Dieser ermöglicht es Inhalte an die Components weiterzugeben. 
Dafür muss allerdings ein Shadow DOM verwendet werden. 
Ein Slot wird immer mit seinem Namen referenziert. 
Wenn bei einem Child-Node eines Custom Elements kein Slotname angegeben ist, dann wird dieses Node im Slot ohne Namen gerendert. 
Ein Node versteht sowohl ein HTML-Element, als auch ein Text-Node, welches direkt in das Element eingefügt wird. 
Wird für einen Slot kein Node gesetzt, dann wird der Inhalt des Slots als Standardwert verwendet. 
\cite{MDNSlots}

\pagebreak
\hfive{Beispiel eines Webcomponents mit Slots und Shadow DOM}

\typescript[code:WebComponentExampleJS]{code/WebComponents/eleven.ts}{Web Component Beispiel (Javascript)}


\html[code:WebComponentExampleHTML]{code/WebComponents/twelve.html}{Web Component Beispiel (HTML)}

Die Codebeispiele \ref{code:WebComponentExampleJS} und \ref{code:WebComponentExampleHTML} erzeugen folgende Ausgabe:

{\ttfamily Hi Universe!!}
\hthree{Implementierung}

\sectionauthor{Julian Kusternigg}

\hfour{Einleitung}
Wie ein der Rest von ZELIA ist auch der Code im Frontend in Typescript geschrieben, um einen besseren Überblick zu behalten. Zusätzlich bringt es noch Vorteile bei der Implementierung des Komponentensystems. Wie schon gesagt ist das Komponentensystem, neben dem Client-Side-Router, der zweite Teil unserer Frontend Bibliothek, den wir selbst entworfen haben, um unsere Webseite dazustellen zu können. Somit müssen wir uns nicht auf andere Frameworks verlassen, die in den meisten Fällen nicht so performant sind, da unsere Bibliothek perfekt an die Bedürfnisse von ZELIA angepasst ist.

\hfour{Architektur}
Wie eben im Standard beschrieben muss es eine Klasse geben, welche von "HTMLElement" erbt, um als "Custom HTML Element" verwendet werden zu können. So gibt es in der ZELIA Implementierung die abstrakte Klasse "Component", von der alle anderen Komponenten erben.

\typescript{code/WebComponents/BaseInterfaces.ts}{Schnittstellen der Basiskomponente}

Die jeweils dazugehörige HTML Quelle wird vom sogenannten "Component Loader", wenn die Website aufgerufen wird, geladen und zwischengespeichert. Dadurch muss nicht jede Komponente ihren Darstellungscode selbst herunterladen, wenn sie verwendet wird, wie es bei manchen Bibliotheken der Fall ist. Somit ist es kaum merkbar, wenn man eine Seite wechselt, denn die Seite kann schon angezeigt werden, während die nötigen Infos noch nachgeladen werden.

Mit Hilfe von "States" kann man Text dynamisch ersetzen, wie es auch in anderen Bibliotheken gemacht wird.

\html{code/WebComponents/State.html}{Ersetzungsmechanismus (HTML)}
\typescript{code/WebComponents/State.ts}{Ersetzungsmechanismus (Javascript)}

Dadurch können zum Beispiel nachgeladene Daten eingefügt werden, aber auch die Sprache der Seite kann schnell geändert werden.

Zusätzlich kann sich das Komponentensystem automatisch Referenzen auf HTML Elemente suchen. Dafür muss im Konstruktor von der Basiskomponente, von der alle Komponenten erben, nur definiert sein, wie ein Element gefunden wird. Um die Richtigkeit der Datentypen und die automatische Vervollständigung aufrecht zu erhalten, wird als generischen Typ der Komponente mitgegeben, welche Datentypen die Elemente haben, die automatisch gefunden werden.

\typescript{code/WebComponents/AutoRef.ts}{Erstellen einer Komponente und suchen einer Referenz}

In diesem Beispiel wird das Element "button" gesucht. Dieses Element ist vom Typ "HTMLButtonElement" und kann mit "\#btnInfo" gefunden werden. Die Raute im Suchfeld heißt, dass der nachfolgende Text die ID von dem HTML-Element ist. Somit würde der Suchtext für das folgende Beispiel zutreffen:

\html{code/WebComponents/IDButton.html}{Beispiel eines HTML Knopfes mit ID}

Im Endeffekt steht dann auf dem Knopf "Click Me". Das "Click" wird mithilfe der
"States" ersetzt und danach wird über die Elementreferenz ein "Me" angehängt. Das automatische Finden von Referenzen ist wichtig, um Events abzufangen oder Elemente zu modifizieren, wenn es notwendig ist.

Wie kurz erwähnt lädt der "Component Loader" die notwendigen HTML-Files und speichert sie. Verwendet werden die Daten dann von der Basis "Component"- Klasse, welche im Konstruktor einen Parameter bekommt, in dem der Name des HTML Elements steht. Über diesen Namen holt sich die Komponente, wenn sie erstellt, wird die HTML Ressource. Wenn sie schließlich angezeigt werden soll, ersetzt sie die "States" und fügt sich selbst in die Webseite hinzu.

\typescript{code/WebComponents/ComponentLoader.ts}{Laden einer Komponente}

Erst nachdem alle Ressourcen der Komponenten geladen wurden, wird der Client-Side-Router (siehe Kapitel Client-Side-Router) erstellt und somit die Seite gestartet. Dadurch wird sichergestellt, dass keine verzögerten Ladezeiten durch das Laden der Ressourcen auftreten.

\hfour{"Life-Cycle"}

Wenn nun ein eigenes Element erstellt wird, hat der "Loader" die HTML-Ressource, falls es eine gibt, bereits geladen. Während der Initialisierung des Elements, holt sich dieser die HTLM-Ressource als Text. Im Konstruktor der Basiskomponente gibt es neben dem "query"-Parameter, welcher für die automatische referenzieren verwendet wird, noch die "autoRender" und "useShaddowRoot" Argumente. Standardmäßig sind beide Werte auf Wahr gesetzt. Das heißt die Komponente wird, sobald sie auf die Seite hinzugefügt wird, ihr HTML anzeigen und ein einem "ShadowRoot" eingebunden werden. Durch das "ShadowRoot" wird es als eigenes Dokument angezeigt unabhängig von dem, wo es eingebunden wird, was Überschneidungen verschieden CSS-Styles behebt. Somit sieht die Komponente jedes Mal gleich aus, unabhängig von der restlichen Seite.

Bevor der HTML-Text dargestellt werden kann, werden die States ersetzt. Danach wird geschaut ob die Elemente die automatisch referenziert werden sollen, vorhanden sind. Falls sich ein "State" ändert, wird der Inhalt der Komponente automatisch neu erstellt und gerendert. Durch das neu erstellen der Komponente bekommen auch die referenzierten Elemente neue Referenzen. Das könnte ein Problem sein, wenn man Komponenten hat, bei denen es wichtig ist, dass die Elemente "dieselben" bleiben. Somit sollte man sich um das Modifizieren dieser Komponenten selbst kümmern oder keine States ändern, sodass nicht neu gerendert werden muss.

\hfour{Eingebaute Komponenten}

Insgesamt besteht das komplette Frontend aus 12 Komponenten, die auf 7 Seiten verteilt sind. Zusätzlich gibt es die "Debug"-Komponente, die uns während der Entwicklung einen Logger zu Verfügung gestellt hat, der vor allem praktisch war, um auf Mobilgeräten die Hintergrundaufgaben mitverfolgen zu können.

\hfive{Die Startseite}

Auf dem Root-Pfad ("/") liegt unsere Willkommensseite. Auf ihr findet man die Raum-\\eingabekomponente, die von der ZELIA-API alle möglichen Raumnummern abfragt und als Auswahlhilfe anzeigt. Auf Mobilgeräten, wie Smartphones oder Tablets, wird zusätzlich ein Knopf angezeigt, der es möglich macht, die Raumnummer mit der Kamera des Geräts einzulesen. Wenn man auf diesen Knopf drückt, wird man zur "OCR"-Seite weitergeleitet. Gibt man die Raumnummer selbst ein, so landet man auf der Rauminformationsseite. Während der manuellen Eingabe, werden alle bekannten Räume als "Dropdown" unter der Eingabebox angezeigt. Gibt man die ersten Buchstaben des Raumnamens oder Raumnummer an, so werden nur mehr Ergebnisse angezeigt, die dem angegebenen Muster entsprechen.

Beispiel:

\image{media/WebComponents/Startseite}{ZELIA Startseite}

\hfive{Die "OCR"-Seite}

Auf dieser Seite können alle Geräte, die eine Kamera haben, mittels "Optical Character Recognition" (siehe Kapitel Optical Character Recognition), die Raumnummer von abscannen, zum Beispiel von Raumschildern, die vor den Räumen hängen. Aufgebaut ist die Seite mit einer Komponente, welche anzeigt was die Kamera sieht und einer Zweiten die als Link dient, um zurück zur Startseite zu kommen. Die Komponente, welche die Kamera anzeigt, ist auch zuständig um den OCR-Service (Tesseract) zu starten und mit Hilfe von diesem den Text aus den Bildern auszulesen. Bevor das Bild an den OCR-Service übergeht, wird es auf 2 Farben komprimiert und so klein zusammengeschnitten, dass nur mehr der markierte Fokusteil der Kamera im Bild ist.

% TODO: Bild von der Seite

Meist liefert der OCR-Prozess gut erkennbare Ergebnisse. Allerdings kann es auch passieren, gerade wenn viele helle und dunkle Flecken im Hintergrund des Bildes sind, dass diese als Zeichen erkannt werden. Somit ist wichtig, dass nachdem das Bild in Text umgewandelt wurde, mittels "Regular Expressions" der wichtige Teil aus dem Text herausgefiltert wird. Dieser gesamte Prozess, Bild in Text umzuwandeln, wird einmal pro Sekunde gemacht. Dieser Wert ist einfach änderbar, aber wir haben uns nach längerem testen geeinigt, dass eine Sekunde völlig ausreichen ist, um schnell Raumnummern zu erfassen. Würde der Wert kleiner werden, also wenn mehrmals pro Sekunde Bild zu Text verarbeitet wird, so würden schwächere Smartphones schnell warm werden und mehr Energie verbrauchen. Wenn eine Raumnummer erfolgreich eingelesen wurde, wird man automatisch auf die Rauminformationsseite weitergeleitet.