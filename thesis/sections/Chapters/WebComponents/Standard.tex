\hthree{Der "Web Component" Standard}

Web Components ermöglichen es den gleichen Code des Clients an mehreren Stellen zu verwenden. Dazu wurden drei neue Technologien \cite{MDNWebComponents} eingeführt:

\begin{itemize}
    \item Custom Elements - Selbstdefinierte HTML-Elemente
    \item Shadow DOM - Ein unabhängig gerenderter DOM-Baum
    \item HTML Templates - Ein HTML-Tag, der nicht angezeigt wird, um dessen Inhalt von Javascript zu referenzieren
\end{itemize}


\hfour{Custom Elements}

\hfive{Erstellung eines Custom Elements}

Custom Elements sind ES6 Klassen, die von der Klasse "HTMLElement" erben. "ES6", ausgeschrieben "ECMA Script 6", ist der Standard, in dem Klassen in Javascript hinzugefügt wurden.\cite{MDNes6Classes}

\typescript{code/WebComponents/one.ts}{Erstellung einer Custom Element Klasse}

Diese Klasse kann nun als Custom Element registriert werden. Dabei muss der Name des Elements vergeben werden, welcher einen Bindestrich enthalten muss.

\typescript{code/WebComponents/two.ts}{Registrierung des Custom Elements}

Wenn das Custom Element registriert ist, kann es mit dem vergebenen Namen im HTML verwendet werden.

\html{code/WebComponents/three.html}{Verwendung des Custom Elements im HTML}

\hfive{Lifecycle-Methoden}
Zur besseren Kontrolle des Lebenszyklus eines Custom Elements werden sogenannte "lifecycle callbacks" verwendet.
Diese "lifecycle callbacks" sind spezielle Methoden in einer Klasse, welche von der API der Custom Elements aufgerufen werden. \cite{MDNWebCustomElements}

\typescript{code/WebComponents/LifecycleExamples.ts}{"connectedCallback" Beispiel}

Der "{\ttfamily connectedCallback}" ist eine der Lifecycle-Methoden und wird aufgerufen, wenn das Element in das DOM eingefügt wird. 
Das DOM (Document Object Model) ist eine Programmierschnittstelle, über die auf die einzelnen HTML Elemente zugegriffen werden kann. \cite{MDNDOM} Im "{\ttfamily connectedCallback}" können zum Beispiel Event-Listeners registriert werden. \cite{MDNWebCustomElements}

\typescriptsub{code/WebComponents/LifecycleExamples.ts}{"connectedCallback" Beispiel}{12}{14}

% todo: zum
Die Lifecycle-Methode "{\ttfamily disconnectedCallback}" ist das Gegenstück zu \linebreak "{\ttfamily connectedCallback}".
Diese wird beim Verlassen des DOMs aufgerufen. Darin sollten etwaige "Event Listener" entfernt werden, um die nicht mehr benötigten Ressourcen freizugeben und dadurch Speicher und CPU-Zeit zu sparen. 
Auf diese Art wird sichergestellt, dass die Anwendung auch nach längerer Nutzung performant funktioniert. \cite{MDNWebCustomElements}

\typescriptsub{code/WebComponents/LifecycleExamples.ts}{"disconnectedCallback" Beispiel}{16}{18}

Die Lifecycle-Methode "{\ttfamily adoptedCallback}" wird jedes Mal aufgerufen, wenn das Element mit der "{\ttfamily adoptNode}"-Methode in ein anderes Dokument - \zb\ eines "{\ttfamily iframe}"-Elements - verschoben wird \cite{MDNWebCustomElements}

\typescriptsub{code/WebComponents/LifecycleExamples.ts}{"adoptedCallback" Beispiel}{24}{26}

Der "{\ttfamily attributeChangedCallback}" Lifecycle-Callback wird jedes Mal aufgerufen, wenn ein Attribut des Custom Elements geändert wird. 
Damit dieser Callback ausgelöst wird, müssen die Namen der zu überwachenden Attribute von der statischen Eigenschaft "{\ttfamily observedAttributes}" zurückgegeben werden. \cite{MDNWebCustomElements} Beim Codebeispiel \ref{code:attributeChangedCallback} wird die Methode "{\ttfamily attributeChangedCallback}" nur aufgerufen, wenn die Attribute "from" und "to" geändert werden.
    
Der "{\ttfamily attributeChangedCallback}" nimmt drei Parameter\cite{MDNWebCustomElements}:
\begin{itemize}
    \item Name: Name des Attributes
    \item Old Value: Den Wert vor der Änderung
    \item New Value: Der neue, aktuelle Wert 
\end{itemize}

\typescriptsub[code:attributeChangedCallback]{code/WebComponents/LifecycleExamples.ts}{"attributeChangedCallback" Beispiel}{28}{34}

\pagebreak
\hfour{Shadow DOM}

Unter DOM (Document Object Model) versteht man die Programmierschnittstelle, mit der auf die Elemente einer Webseite zugegriffen werden kann, um deren Inhalt und Darstellung zu verändern. \cite{MDNDOM}

Ein Shadow DOM ist ein Ast im DOM-Baum innerhalb der Webseite, welcher separat von der restlichen Seite gerendert wird.
Dadurch kann auch auf Custom Elements ein Style in Style-Tags erstellt werden, ohne das Design der restlichen Seite zu beeinflussen. \cite{MDNShadowRoot}

Shadow DOMs bieten die Möglichkeit, einzelne Elemente des Webprojekts abgesondert vom Rest der Seite zu behandeln, ohne dafür "{\ttfamily iframe}"-Elemente zu verwenden. Dabei ermöglichen sie, die Komponente ganz unabhängig von dokumentenübergreifenden Styling-Anweisungen zu gestalten. \cite{IONOSShadowDom}

Damit ein "Shadow Root" erstellt werden kann, muss lediglich die "{\ttfamily attachShadow}"-Methode aufgerufen werden. \cite{MDNShadowRoot}

\typescript{code/WebComponents/eight.ts}{Erstellung eines Shadow Roots}

Der "Mode" muss auf "{\ttfamily open}" gesetzt werden, damit der "Shadow Root" befüllt werden kann. Nun kann mit "{\ttfamily this.shadowRoot}" auf den "ShadowRoot" innerhalb des Custom Components zugegriffen werden.
\cite{MDNShadowRoot}

\pagebreak
\hfour{HTML Templates}

Um ein Custom Element mit Inhalt zu füllen, gibt es in HTML den "template"-Tag. Dieser wird nicht im DOM gerendert, kann aber trotzdem mit Javascript referenziert werden. \cite{MDNtemplate}

\html{code/WebComponents/nine.html}{Erstellung eines HTML Template}

Damit das Template als Custom Element gerendert wird, muss der Content in dieses, oder dessen Shadow DOM kopiert werden.

\typescript{code/WebComponents/ten.ts}{Kopieren des Template in das Custom Element}

\hfive{Slots}

Um den Inhalt der Components dynamisch zu machen, gibt es den HTML "Slot"-Tag. 
Dieser ermöglicht es, Inhalte an die Components weiterzugeben. 
Dafür muss allerdings ein Shadow DOM verwendet werden. 
Ein Slot wird immer mit seinem Namen referenziert. 
Wenn bei einem Child-Node eines Custom Elements kein Slotname angegeben ist, dann wird dieses Node im Slot ohne Namen gerendert. 
Unter einem Node versteht man sowohl ein HTML-Element als auch ein sogenannter Text-Node. 
Letzteres ist ein Text, der direkt in das Element eingefügt wird. 
Wird für einen Slot kein Node mit dem jeweiligen "{\ttfamily slot}"-Attribut gesetzt, dann wird der Inhalt des Slots als Standardwert verwendet. 
\cite{MDNSlots}

\pagebreak
\hfive{Beispiel eines Webcomponents mit Slots und Shadow DOM}

\typescript[code:WebComponentExampleJS]{code/WebComponents/eleven.ts}{Web Component Beispiel (Javascript)}


\html[code:WebComponentExampleHTML]{code/WebComponents/twelve.html}{Web Component Beispiel (HTML)}

Die Codebeispiele \ref{code:WebComponentExampleJS} und \ref{code:WebComponentExampleHTML} erzeugen folgende Ausgabe:

{\ttfamily Hi Universe!!}