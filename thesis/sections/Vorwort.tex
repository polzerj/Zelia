\hnone{Vorwort}

Liebe Leserinnen, liebe Leser,

die folgende Diplomarbeit handelt von der von uns entwickelten Software \ZELIA\, welche ein Lehrrauminfosystem für Schulen ist. Im Zuge des Projektes beschäftigten wir uns mit unterschiedlichen Technologien, um das Produkt bestmöglich umzusetzen.

Diese Diplomarbeit verfassten wir als Abschluss unserer fünfjährigen HTL Ausbildung an der HTL Ungargasse mit der Fachrichtung Informations\-technologie-Netzwerk\-technik. Das Ziel unseres Projektes war es, einerseits ein innovatives Produkt zu entwickeln und andererseits Forschungen zu neuen technischen Mechanismen anzustellen.

Deshalb möchten wir uns bei unseren Diplomarbeitsbetreuern DI Michael Vogel und DI Johannes Neuhofer bedanken, welche uns während der Diplomarbeit immer unterstützend zur Seite gestanden sind. Außerdem wollen wir unsern Eltern, für ihre Unterstützung und das Korrekturlesen der Diplomarbeit unsern Dank aussprechen. Weiters wollen wir uns bei Ing. Michael Lichtblau-Früh für die Umsetzung des Logos der Diplomarbeit und bei OSR Mst. Ing. Dipl.-Päd. Hans Fürst für den Zugang zu WebUntis bedanken.

Wir wünschen Ihnen viel Freude beim Lesen der Diplomarbeit \ZELIA.

Mersed Keco, Julian Kusternigg, Mario Naunovic, Richard Panzer, Johannes Polzer

Wien, 15. März 2022
