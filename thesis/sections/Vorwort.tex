\hnone{Vorwort}

Liebe Leserinnen, liebe Leser,

die folgende Diplomarbeit handelt von der von uns entwickelten Software \ZELIA, welche ein Lehrrauminfosystem für Schulen ist. Im Zuge des Projektes beschäftigten wir uns mit unterschiedlichen Technologien, um das Produkt bestmöglich umzusetzen.

Diese Diplomarbeit verfassten wir als Abschluss unserer fünfjährigen HTL-Aus\-bildung an der HTL Ungargasse mit der Fachrichtung Informations\-technologie und Netzwerk\-tech\-nik. 
Das Ziel unseres Projektes war einerseits ein innovatives Produkt zu entwickeln und andererseits Forschungen zu neuen technischen Mechanismen anzustellen.

Deshalb möchten wir uns bei unseren Diplomarbeitsbetreuern, Herrn DI Michael Vogel und Herrn Mag. Johannes Neuhofer, bedanken, welche uns während der Diplomarbeit immer unterstützend zur Seite gestanden sind. 
Außerdem wollen wir unseren Eltern für ihre Unterstützung und das Korrekturlesen der Diplomarbeit unseren Dank aussprechen.  
Weiters wollen wir uns bei Herrn Ing. Michael Lichtblau-Früh für die Gestaltung des Logos der Diplomarbeit, bei Herrn OSR Mst. Ing. Dipl.-Päd. Hans Fürst für den Zugang zu WebUntis und bei Frau Mag. Elisabeth Schaludek-Paletschek für die Unterstützung beim Verfassen des Abstracts bedanken.

Wir wünschen Ihnen viel Freude beim Lesen der Diplomarbeit \ZELIA.

Mersed Kečo, Julian Kusternigg, Mario Naunović, Richard Panzer, Johannes Polzer

Wien, 25. März 2022
