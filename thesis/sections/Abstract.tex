\hnone{Abstract}
%\begin{otherlanguage}{english}
    Schools provide their a very good infrastructure for their teachers and students. 
    To ensure that this infrastructure is constantly improved, \ZELIA\ -- a real-time teaching room information system -- was developed.
    
    Information concerning the timetable of the room, the number of chairs, the con\-nec\-tion options to the projector etc. should be managed and displayed in a way that is easily accessible and understandable.
    Individual status information such as defective lamps, broken windows and vandalized desks in the rooms can be reported within a few clicks.
    Similarly, bookings for studying groups and private tutoring lessons can be made.
    
    \ZELIA\ can be accessed through any internet-enabled device with a web browser. 
    On the main page, the users can manually enter a room number or scan it with their cameras. 
    After entering this number, the information for the selected room will be displayed on the "Room Information Page". 
    From there, reservations can be made and issues reported.
    
    \ZELIA\ concists of the following parts, which were implement either individually or in a team of two:
    \begin{itemize}
        \item Server -- provides the information
        \item Database -- stores the room data, the report messages and booking entries
        \item Client -- allows end users to access \ZELIA\ through a web browser.
    \end{itemize}
    
    Each of these parts were implemented independently in small teams consisting of up to two people.
%\end{otherlanguage}