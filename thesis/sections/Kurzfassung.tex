\hnone{Kurzfassung}

Schulen bieten ihren Schüler*innen sowie Lehrer*innen eine sehr gute Infrastruktur. Im Zuge, diese ständig zu verbessern, wurde \ZELIA\ - ein Echtzeit-Lehr\-raum\-informations\-system - entwickelt. 

Daten über Räume wie der jeweilige Stundenplan, Anzahl der Sitzplätze, Anschlussmöglichkeiten an den Projektor usw. sollen damit abgewickelt werden. Zusätzlich können individuelle Statusinformationen wie Auffälligkeiten und Mängel in den Räumen gemeldet oder Buchungen für Lerngruppen und Nachhilfestunden getätigt werden.

Schüler*innen und Lehrer*innen können über jedes internetfähige Gerät mit einen Webbrowser auf \ZELIA\ zugreifen. Dort können sie eine Raumnummer mit der Kamera einscannen oder manuell eingeben. Nach Eingabe dieser Nummer findet man die Informationen, wie oben beschrieben, für den jeweiligen Raum vor. Von dort aus können Endbenutzer*innen Meldungen und Buchungen tätigen.

Aufteilen kann man \ZELIA\ in drei Teile:
\begin{itemize}
    \item Den Server -- Welcher die Informationen zur Verfügung stellt.
    \item Die Datenbank -- Wo die Raumdaten, sowie Meldungen und Buchungen, gespeichert sind.
    \item Den Client -- Auf den die Endbenutzer*innen über ihren Webbrowser zugreifen können um \ZELIA\ zu verwenden.
\end{itemize}

Jeder dieser Teile wurde in kleineren Gruppen innerhalb des \ZELIA-Teams entwickelt.
