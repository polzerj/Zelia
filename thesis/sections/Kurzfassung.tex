\hnone{Kurzfassung}

Schulen bieten ihren Schülerinnen und Schülern sowie Lehrerinnen und Lehrern eine sehr gute Infrastruktur. 
Im Zuge, diese ständig zu verbessern, wurde \ZELIA\ -- ein Echtzeit-Lehr\-raum\-informations\-system -- entwickelt. 

Informationen über Räume wie der jeweilige Stundenplan, die Anzahl der Sitzplätze, Anschlussmöglichkeiten an den Projektor usw. sollen damit übersichtlich verwaltet und dargestellt werden.
Zusätzlich können individuelle Statusinformationen wie beispielsweise defekte Lampen in den Räumen gemeldet oder Buchungen für Lerngruppen und Nachhilfestunden getätigt werden.

Schüler*innen und Lehrer*innen können über jedes internetfähige Gerät mit einem Webbrowser auf \ZELIA\ zugreifen. 
Dort können die Nutzer*innen eine Raumnummer eingeben oder mit der Kamera einscannen. 
Nach Eingabe dieser Nummer werden die Informationen für den ausgewählten Raum, auf der "Rauminformationsseite", angezeigt. 
Von dort aus können Nutzer*innen von \ZELIA\ Meldungen und Buchungen tätigen.

\ZELIA\ gliedert sich in folgende Teile:
\begin{itemize}
    \item Server -- stellt die Informationen zur Verfügung
    \item Datenbank -- speichert die Raumdaten sowie Meldungen und Buchungen
    \item Client -- ermöglicht Endbenutzerinnen und Endbenutzern den Zugriff auf \ZELIA\ über einen Webbrowser
\end{itemize}

Jeder dieser Teile wurde einzeln oder in einem Team bestehend aus zwei Personen realisiert.
